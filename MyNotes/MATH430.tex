\documentclass[12pt]{article}

%\documentclass{amsart}
%\documentclass{scrartcl}
%\usepackage{changepage}
%\usepackage{scrextend}

\usepackage{amssymb,amsmath,amsthm}
% amssymb has empty set symbo
\usepackage{scrextend} % for \begin{addmargin}[0.55cm]{0cm} text \end{margin}

\usepackage{mathrsfs} % for \mathscr{P}
\usepackage{float}
\usepackage{enumitem}
\usepackage{hanging}
\usepackage{verbatim}
\usepackage{mathtools}
\usepackage{pst-node}%

\newcommand{\n}{ \noindent }
\newcommand{\F}{\mathcal{F}}
\newcommand{\G}{\mathcal{G}}
\newcommand{\U}{\mathcal{U}}

\usepackage{hyperref}
\hypersetup{
    colorlinks,
    citecolor=black,
    filecolor=black,
    linkcolor=black,
    urlcolor=black
}

\newcommand{\pwset}{\mathcal{P}}
%\newcommand{\pwset}{\mathscr{P}}
\DeclareMathOperator{\LO}{\mathcal{L}}

\DeclareMathOperator{\proj}{proj}
\newcommand{\vct}{\mathbf}
\newcommand{\vctproj}[2][]{\proj_{\vct{#1}}\vct{#2}}


%https://tex.stackexchange.com/questions/22252/how-to-typeset-function-restrictions
\newcommand\restr[2]{{% we make the whole thing an ordinary symbol
  \left.\kern-\nulldelimiterspace % automatically resize the bar with \right
  #1 % the function
  \vphantom{\big|} % pretend it's a little taller at normal size
  \right|_{#2} % this is the delimiter
  }}

% from https://tex.stackexchange.com/questions/644238/drawing-the-phase-portrait-of-two-differential-equations
% \usepackage{pgfplots}
% \pgfplotsset{compat=1.8}
% \usepackage{amsmath}
% \usepackage{derivative}
\usepackage{pgfplots}
\pgfplotsset{compat=1.8}
\usetikzlibrary{decorations.markings}
\usepackage{amsmath}
\usepackage{derivative}


%\DeclareFontFamily{U}{MnSymbolC}{}
%\DeclareSymbolFont{MnSyC}{U}{MnSymbolC}{m}{n}
%\DeclareFontShape{U}{MnSymbolC}{m}{n}{
%  <-6>    MnSymbolC5
%  <6-7>   MnSymbolC6
%  <7-8>   MnSymbolC7
%  <8-9>   MnSymbolC8
%  <9-10>  MnSymbolC9
%  <10-12> MnSymbolC10
%  <12->   MnSymbolC12%
%}{}
%\DeclareMathSymbol{\powerset}{\mathord}{MnSyC}{180}

\newcommand{\overbar}[1]{\mkern 1.5mu\overline{\mkern-1.5mu#1\mkern-1.5mu}\mkern 1.5mu}
\newcommand{\overtilde}[1]{\mkern 1.5mu\widetilde{\mkern-1.5mu#1\mkern-1.5mu}\mkern 1.5mu}

%\usepackage[left=2cm,right=2cm,top=2cm,bottom=2cm]{geometry}
\usepackage[left=1.2in, right=1.2in, top=1in, bottom=1in]{geometry}

\newtheorem*{proposition}{Proposition}
\newtheorem*{lemma}{Lemma}

\theoremstyle{definition}
\newtheorem*{definition}{Definition}
\newtheorem{conj}{Conjecture}[section]
\newtheorem*{example}{Example}
\newtheorem{theorem}{Theorem}[section]  % This enables \begin{theorem}
\newtheorem*{corollary}{Corollary}
\newtheorem*{claim}{Claim}
\newtheorem*{remark}{Remark}

\newcommand{\innerproduct}[2]{\langle #1, #2 \rangle}
\DeclareMathOperator{\Tr}{Tr} 

\setcounter{MaxMatrixCols}{20}

\title{STAT 430 - Notes \\
Matrix Analysis}

\begin{document}
\maketitle
\tableofcontents

\section{Matrix}


An $m \times n$ matrix $A$ can be written as

\[
A =
\begin{bmatrix}
a_{11} & a_{12} & \cdots & a_{1n} \\
a_{21} & a_{22} & \cdots & a_{2n} \\
\vdots & \\
a_{m1} & a_{m2} & \cdots & a_{mn}
\end{bmatrix}
\]

\section{Notation}

\begin{itemize}
\item $a_{ij}$ is the $(i, j)$ component of A, so $a_{ij}$ is in
the i\textsuperscript{th} row and j\textsuperscript{th} column.

\item each horizontal line of $A$ is a \textbf{row} of $A$
\item each vertical line of $A$ is a \textbf{column} of $A$
\item $m \times n$ is the \textbf{size} of $A$, i.e., $A$ has
$m$ rows and $n$ columns.
\end{itemize}

A shorthand notation for $A$ is $A = [a_{ij}]$.

\section{Other formulations of a matrix}

\begin{enumerate}
\item \textbf{Column form} We can write $A$ as 
$A = [A_{\bullet 1} \quad A_{\bullet 2} \quad \cdots \quad A_{\bullet n}]$
where $A_{\bullet k}$ represents the k\textsuperscript{th} column of $A$, i.e.,

\[
A_{\bullet k} =
\begin{bmatrix}
a_{1 k} \\
a_{2 k} \\
\vdots \\
a_{m k}
\end{bmatrix}
\]

\begin{remark}
If $A$ is an $m \times 1$ matrix then $A$ is called an m-column vector or simply
an m-vector.
\end{remark}

\item \textbf{Row form} We can write $A$ as

\[
A =
\begin{bmatrix}
A_{1 \bullet} \\
A_{2 \bullet} \\
\vdots \\
A_{m \bullet}
\end{bmatrix}
\]

where $A_{\ell \bullet}$ represents the $\ell$\textsuperscript{th} row of $A$, i.e.,
$A_{\ell \bullet} = [a_{\ell 1} \quad a_{\ell 2} \quad \cdots \quad a_{\ell n}]$.
\end{enumerate}


\section{Examples}

\begin{enumerate}
\item Let $A$ be an $m \times n$ matrix. If $m = n$, then $A$ is a \textbf{square matrix}.

\item Given $A = [a_{ij}]$ such that $a_{ij} = 0$, $\forall i, j$. Then $A$ is called
the \textbf{zero matrix}, i.e., $A = 0$.
\end{enumerate}

\section{Equivalence of matrices}

Given two matrices $A = [a_{ij}]$ and $B = [b_{ij}]$ we say that $A = B$ if

\begin{enumerate}
\item $A$ and $B$ are the same size,
\item $a_{ij} = b_{ij}$ for all $i,j$.
\end{enumerate}

\section{Basic Matrix Operations}

\begin{enumerate}
\item \textbf{Addition}: Let $A$ and $B$ be $m \times n$ matrices. Then

\[
A + B = [a_{ij} + b_{ij}].
\]

\item \textbf{Scalar multiplication}: Let $c$ be a scalar, i.e., $c$ is either a real or complex
number. Given $A = [a_{ij}]$ then

\[
cA = [c \cdot a_{ij}].
\]
\end{enumerate}

\subsubsection{Examples}

Let 
$
A =
\begin{bmatrix}
2 & 1 \\
0 & 1 \\
1 & 0
\end{bmatrix}_{3 \times 2}
$ and
$
B =
\begin{bmatrix}
-1 & 1 \\
0 & 0 \\
0 & -1
\end{bmatrix}_{3 \times 2}.
$

Find $2A + B$.

\textit{Solution:} First we find $2A$ as
\[
2A =
2
\begin{bmatrix}
2 & 1 \\
0 & 1 \\
1 & 0
\end{bmatrix}
=
\begin{bmatrix}
4 & 2 \\
0 & 2 \\
2 & 0
\end{bmatrix}.
\]
Then we find $2A + B$ as

\[
2A + B =
\begin{bmatrix}
4 & 2 \\
0 & 2 \\
2 & 0
\end{bmatrix}
+
\begin{bmatrix}
-1 & 1 \\
0 & 0 \\
0 & -1
\end{bmatrix}
=
\begin{bmatrix}
3 & 3 \\
0 & 2 \\
2 & -1
\end{bmatrix}.
\]

\begin{definition}
Let $A$ be an $m \times n$ matrix. Then the \textbf{additive inverse} of A is defined as

\[
-A = (-1)A = [-a_{ij}].
\]
\end{definition}

\begin{definition}
Let $A$ and $B$ be $m \times n$ matrices. Then the \textbf{difference} of $A$ and $B$ is
defined as

\begin{align*}
A - B  &= A + (-B) \\
&= [a_{ij}] + [-b_{ij}] \\
&= [a_{ij} - b_{ij}].
\end{align*}
\end{definition}

\begin{enumerate}
\setcounter{enumi}{2}
\item \textbf{Transpose} Let $A$ be an $m \times n$ matrix and $A = [a_{ij}]$. Let $A^T$
be the transpose of $A$ where

\[
\left[ A^T \right]_{ij} = \left[ A \right]_{ji} = a_{ij}.
\]

Then $A^T$ is an $m \times n$ matrix, where the k\textsuperscript{th} row is the
k\textsuperscript{th} column of $A$.

\begin{remark}
For a square matrix, the transpose operation does not change the size of the matrix.
\end{remark}

\item \textbf{Conjugate and conjugate transpose}

Let $z = a + bi$ be a complex number where $a, b \in \mathbb{R}$, and $i = \sqrt{-1}$ is
the imaginary unit. We call $a$ the real part of $z$, i.e., $a = Re(z)$ and $b$ is
the imaginary part of $z$, i.e., $b = Im(z)$.

The \textbf{conjugate} of $z$ is

\[
\overbar{z} = \overbar{a + bi} = a - bi.
\]

Particularly, if $z$ is a real number, i.e., $z = a$, then $\overbar{z} = z$.

Suppose $A = [a_{ij}]$ is a complex matrix, i.e., $a_{ij}$ is a complex number. Therefore,

$$ \bar{A} = \left[ \bar{a_{ij}} \right]. $$


The \textbf{conjugate transpose} of $A$ is 

$$ A^{\star} = ( \bar{A} )^T. $$


\begin{example}
Let 
$ A =
\begin{bmatrix}
1 & 1 + i \\
i & 3 \\
0 & 2
\end{bmatrix}.
$

Then to calculate the conjugate transpose we first calculate the conjugate of $A$ as
$
\bar{A} =
\begin{bmatrix}
1 & 1 - i \\
-i & 3 \\
0 & 2
\end{bmatrix}.
$
Then we calculate the conjugate transpose as

$
(\bar{A})^T =
\begin{bmatrix}
1 & -i & 0 \\
1-i & 3 & 2
\end{bmatrix}
= A^{\star}
$.

\begin{remark}
The conjugate and transpose operations are commutative. Thus
$A^{\star} = \overline{(A^T)}$.
\end{remark}

\end{example}

\end{enumerate}


\section{Properties of matrix operations}

\begin{enumerate}[label=\Roman*)]
\item \textbf{Addition}

\begin{enumerate}
\item Commutative Property $A + B = B + A$

\item Associative Property
$(A + B) + C = A + (B + C)$.

\item $A + 0 = A$

\item Let $-A$ be the additive inverse of $A$. Then $A + (-A) = 0$.
\end{enumerate}

\item \textbf{Scalar multiplication}

Let $\alpha$ and $\beta$ be two scalars.

\begin{enumerate}
\item $(\alpha \cdot \beta)A = \alpha(\beta A) = \beta (\alpha A)$

\item $(\alpha + \beta)A = \alpha A + \beta A$.

To see this we have
\begin{align*}
(\alpha + \beta)A &= [(\alpha + \beta)a_{ij}] \\
&= [\alpha a_{ij} + \beta a_{ij}] \\
&= [\alpha a_{ij}] + [\beta a_{ij}] \\
&= \alpha A + \beta A.
\end{align*}

\item $1 \cdot A = A$
\end{enumerate}


\item Transpose and conjugate transpose

\begin{enumerate}
\item $(A + B)^T = A^T + B^T$
\item $(cA)^T = c \cdot A^T$
\item $(A^T)^T = A$
\item $(A + B)^{*} = A^* + B^*$
\item $(c \cdot A)^* = \bar{c} \cdot A^*$ where $\bar{c}$ is the conjugate of
complex number $c$
\item $(A^*)^* = A$
\end{enumerate}

\begin{remark}
Below we use the fact that given 2 complex numbers $z$ and $w$, i.e.
$z, w \in \mathbb{C}$, then

$$ \overline{z \cdot w} = \bar{z} \cdot \bar{w}. $$
\end{remark}

\begin{lemma}
Given a complex number $c$ and a complex matrix $A$ we have

$$ (c \cdot A)^* = \bar{c} \cdot A^*. $$
\end{lemma}

\begin{proof}
Recall that $A^* = (\bar{A})^T$. Now using the remark above we can write

\begin{align*}
[\overline{c \cdot A}]_{ij} &= \overline{c \cdot a_{ij}} \\
&= \overline{c} \cdot \overline{a_{ij}} \\
&= \overline{c} \cdot [\overline{A}]_{ij}, \quad \forall i, j. 
\end{align*}

Thus, $\overline{c \cdot A} = \bar{c} \cdot \bar{A}$. Therefore

\begin{align*}
(c \cdot A)^* &= (\overline{c \dot A})^T \\
&= (\overline{c} \cdot \overline{A})^T \\
&= \overline{c} \cdot (\overline{A})^T \\
&= \overline{c} \cdot A^*.
\end{align*}
\end{proof}

\end{enumerate}


\begin{remark}
Let $A : m \times n$ matrix. If $A$ and $A^T$ have the same size, then $m = n$,
i.e., $A$ is a square matrix.
\end{remark}

\begin{definition}
Let $A$ be a square matrix, i.e., $A : n \times n = [a_{ij}]$.

\begin{enumerate}
\item If $A = A^T$, then $A$ is \textbf{symmetric}, i.e., 
$a_{ij} = a_{ji}, \quad \forall i, j$.

\item If $A = -A^T$, then $A$ is \textbf{skew symmetric}, i.e.,
$a_{ij} = -a_{ji}, \quad \forall i, j$.

\item If $A = A^*$, then $A$ is \textbf{Hermitian}, i.e., 
$a_{ij} = \overline{a_{ij}}, \quad \forall i, j$.

\item If $A = -A^*$, then $A$ is \textbf{skew Hermitian}, i.e.,
$a_{ij} = -\overline{a_{ji}}, \quad \forall i, j$.
\end{enumerate}

\end{definition}

Using the above definitions and properties we can prove the following statements:

\begin{enumerate}
\item Let $A$ be a skew symmetric matrix. Then $a_{ii} = 0$ for all $i$ and $j$.

\begin{proof}
Since $A$ is skew symmetric and so $a_{ij} = -a_{ji}$ for all $i$ and $j$. Letting $j = i$,
we have $a_{ii} = -a_{ii}$ and after adding $a_{ii}$ to both sides we have $2 a_{ii} = 0$. Since
$2 \neq 0$, then we must have $a_{ii} = 0$.
\end{proof}

\item If $A$ is skew Hermitian the $a_{ii}$ are pure imaginary numbers.

\begin{proof}
Need to complete.
\end{proof}

\item Let $A$ be a square matrix. Then $A + A^T$ is symmetric.

\begin{proof}
Let $B = A + A^T$. We want to show $B = B^T$. So

\begin{align*}
B^T &= (A + A^T)^T \\
&= A^T + (A^T)^T \\
&= A^T + A \\
&= A + A^T \\
&= B.
\end{align*}

We have shown $B^T = B$ and therefore $B$ is symmetric.
\end{proof}
\end{enumerate}


\section{Diagonal entries and diagonal matrices}

\begin{definition}
Let $A$ be an $n \times n$ matrix and we can write $A$ as

\[
\begin{bmatrix}
a_{11} & a_{12} & \cdots & a_{1n} \\
a_{21} & a_{22} & \cdots & a_{2n} \\
\vdots & & & \\
a_{n1} & a_{n2} & \cdots & a_{nn}
\end{bmatrix}_{n \times n}.
\]


\begin{itemize}[label= ]
\item The entries $a_{ii}$ are called the \textbf{diagonal} entries of $A$ and the other
entries are \textbf{non-diagonal} entries.
 
\item A \textbf{diagonal} matrix $A$ is a matrix whose non-diagonal entries are all zero.

\item An \textbf{upper triangular} matrix is a square matrix whose $a_{ij} = 0$, where $i > j$. 

\item A \textbf{lower triangular} matrix is a square matrix whose $a_{ij} = 0$, where $i < j$.
\end{itemize}

\end{definition}

The following statements follow from the definition of matrix operations and the above 
definitions:

\begin{enumerate}
\item If $A$ is upper triangular then $A^T$ is lower triangular.
\item If $A$ is lower triangular, then $A^T$ is upper triangular.
\item $A$ is lower and upper triangular if and only if $A$ is diagonal.
\end{enumerate}

\begin{example}
Let the $n \times n$ matrix $I$ be defined as

\[
I =
\begin{bmatrix}
1 & 0 & 0 & 0\\
0 & 1 & 0 & 0\\
0 & 0 & \ddots & 0 \\
0& 0 & 0 & 1
\end{bmatrix}_{n \times n}.
\]

Then $I$ is called the \textbf{Identity matrix}.
\end{example}

\section{Matrix-vector product}

\begin{definition}
Let $A$ be an $m \times n$ matrix, i.e., 
$A = [A_{\bullet 1} \quad A_{\bullet 2} \quad \ldots A_{\bullet n}]$, which is 
the column form of $A$. Let $b = \begin{bmatrix} b_1 \\ b_2 \\ \vdots \\ b_n \end{bmatrix}$
be an n-(column) vector. Then the \textbf{matrix-vector product} $Ab$ is defined as

\[
A \cdot b = (b_1 \cdot A_{\bullet 1}) + (b_2 \cdot A_{\bullet 2}) + \cdots + (b_n \cdot A_{\bullet n}).
\]

Note that by the definition of scalar multiplication, $b_1 \cdot A_{\bullet k}$ for 
$k = 1, 2, \ldots, n$ is an m-vector (with $m$ the number of rows in $A$). Then by the definition
of vector addition, $A \cdot b$ is also an m-vector or an $m \times 1$ matrix.
\end{definition}

\subsubsection{Properties of matrix-vector products}

Let $A: m \times n$ matrix and let $u, v$: n-vector and let $c$ be a scalar. Then

\begin{enumerate}
\item $A(u + v) = Au + Av$
\begin{proof}
Need to fill in.
\end{proof}

\item $A(c \cdot u) = c \cdot (A \cdot u)$
\begin{proof}
Need to fill in.
\end{proof}
\end{enumerate}

\begin{remark}
Recall that the identity matrix $I$ can be written as

\[
I =
\begin{bmatrix}
1 & 0 & 0 & 0\\
0 & 1 & 0 & 0\\
0 & 0 & \ddots & 0 \\
0& 0 & 0 & 1
\end{bmatrix}_{n \times n}.
\]

We can also write $I$ as $I = [e_1 e_2 \cdots e_n]$ where $e_k$ is the k\textsuperscript{th}
column of $I$.
\end{remark}

\begin{example}
Let $A: n \times n$ matrix. Then

\[
A \cdot e_1 = A_{\bullet 1} \cdot 1 + A_{\bullet 2} \cdot 0 + \cdots + A_{\bullet n} \cdot 0 = A_{\bullet 1}
\]

where $e_1 = \begin{bmatrix} 1 \\ 0 \\ \vdots \\ 0 \end{bmatrix}$. Likewise,
$A \cdot e_2 = A_{\bullet 2} , \ldots A \cdot e_n = A_{\bullet n}$.
\end{example}

\section{Matrix multiplication}

\begin{definition}
Two matrices $A_{m \times n}$ and $B_{r \times p}$ are \textbf{conformable} if $n = r$, i.e., $B_{n \times p}$.
\end{definition}

It can be shown that if $A$ and $B$ are two square matrices of the same size then they are conformable.

\subsubsection{Matrix product}

Given $A_{m \times n}$ and $B_{n \times p}$ where $B = [B_{\bullet 1} B_{\bullet 2} \cdots B_{\bullet p}]$ and
each $B_{\bullet k}$ is an n-vector, then

\[
A \cdot B = [A \cdot B_{\bullet 1} \quad A \cdot B_{\bullet 2} \quad \cdots \quad A \cdot B_{\bullet p}].
\]

By the definition of a matrix-vector product, $A \cdot B_{\bullet k}$ for $k = 1, 2, \ldots, p$ is an m-vector,
or an $m \times 1$ matrix. Therefore, $A \cdot B$ has $m$ rows and $p$ columns and so $A \cdot B_{m \times p}$
matrix.

Another definition of matrix product is:

\[
\left[ A \cdot B \right]_{ij} = A_{i \bullet} \cdot B_{\bullet j} = \sum_{k = 1}^{n} a_{ik} \cdot b_{kj}.
\]

\subsubsection{Caution and examples}

\begin{enumerate}
\item Even if $A \cdot B$ is well defined, $B \cdot A$ may not. For example, given $A: m \times n$ and $B: n \times p$,
the product $A \cdot B$ is well defined, but if $m \neq p$ then $B \cdot A$ is not defined.

\item Let $A: m \times n$ and $B: n \times m$. In this case $A \cdot B$ and $B \cdot A$ are well defined. But,
$A \cdot B: m \times m$ and $B \cdot A: n \times n$. If $m \neq n$, then $AB \neq BA$.

\item Let $A: n \times n$ and $B: n \times n$. In this case, $A \cdot B$ and $B \cdot A$ are both well defined and
of the same size. However, $AB \neq BA$ in general. For example, let 
$A = \begin{bmatrix} 1 & 1 \\ 0 & 1 \end{bmatrix}_{2 \times 2}$ and
$B = \begin{bmatrix} 0 & 1 \\ 1 & 0 \end{bmatrix}_{2 \times 2}$. Then
$AB = \begin{bmatrix} 1 & 1 \\ 1 & 0 \end{bmatrix}$ and
$BA = \begin{bmatrix} 0 & 1 \\ 1 & 1 \end{bmatrix}$. Therefore, $A \cdot B \neq B \cdot A$. 


\begin{definition}
Let $A, B: n \times n$ matrices. If $A \cdot B = B \cdot A$, then we say that $A$ and $B$ \textbf{commute}, otherwise
they do not commute.
\end{definition}


\item (No cancellation law in general)

Consider $A = \begin{bmatrix} 1 & -1 \\ 2 & -2 \end{bmatrix}$ and $B = \begin{bmatrix} 1 & 3 \\ 1 & 3 \end{bmatrix}$.
Both $A$ and $B$ are non-zero, but $AB = \begin{bmatrix} 0 & 0 \\ 0 & 0 \end{bmatrix}$.

Let $C$ be the $2 \times 2$ zero matrix. Then $A \cdot C = 0$ and therefore $A \cdot B = A \cdot C$ but $B \neq C$.

\end{enumerate}

\section{Product of matrix and identity matrix}

Recall that the $n \times n$ identity matrix

\[
I =
\begin{bmatrix}
1 & 0 & 0 & 0\\
0 & 1 & 0 & 0\\
0 & 0 & \ddots & 0 \\
0& 0 & 0 & 1
\end{bmatrix}_{n \times n}
=
[e_1 \quad e_2 \quad \ldots \quad e_n]
\]

where $e_i$ is the ith column of $I_n$.

Given an $m \times n$ matrix $A$,

\[
A \cdot I_n = A \cdot [e_1 \quad e_2 \quad \ldots \quad e_n] = [Ae_1 \quad Ae_2 \quad \ldots \quad Ae_n].
\]

Since $A \cdot e_i = A_{\bullet i}$, the ith column of $A$, then 

\[
A \cdot I_n = [A_{\bullet 1 } \quad A_{\bullet 2} \quad \ldots \quad A_{\bullet n}] = A.
\]

Similarly, let $I_n$ be the $m \times n$ identity matrix. Then $I_n \cdot A = A$.

\begin{example}
Suppose $A : m \times n$ matrix, such that $A \cdot A$ is well defined.

\textbf{Question:} What can be said about $m$ and $n$?

\textbf{Answer:} Since $A \cdot A$ is well defined then $A$ and $A$ are conformable, then
the number of columns in $A$ must equal the number of rows in $A$. Thus, $m = n$.
\end{example}

\begin{remark}
For the next few sections let $A$ be a $n \times n$ square matrix.
\end{remark}

\section{k\textsuperscript{th} power of a matrix}

Since $A$ is a square matrix then $A \cdot A$ is well defined and $A \cdot A = A^2$. Similarly,

\begin{align*}
A^3 &= A \cdot A \cdot A \\
&\vdots \\
A^k &= A \cdot A \cdots A \quad\quad k \in \mathbb{N}.
\end{align*}

By convention, $A^0 = I_n$.

\section{Properties of matrix products}

\begin{enumerate}
\item $A \cdot (B + C) = A \cdot B + A \cdot C$
\item $(D + E) \cdot F = D \cdot F + E \cdot F$
\item $(A + B) \cdot C = A \cdot (B \cdot C)$
\item Let $\alpha$ be a scalar, then $A(\alpha \cdot B) = \alpha \cdot (A \cdot B) = (\alpha \cdot A) \cdot B$
\item $(A \cdot B)^T = B^T \cdot A^T$, $(A \cdot B)^* = B^* \cdot A^*$
\end{enumerate}

\begin{proof}
(Proof of 5) Consider the pure transpose case only. Let $A : m \times n$ and $B : n \times p$. First we show that
$(A \cdot B)^T$ and $B^T \cdot A^T$ are the same size. Since $A \cdot B$ is $m \times p$ then by the definition of
transpose, $(A \cdot B)^T$ is $p \times m$. Now, $B^T : p \times n$ and $A^T : n \times m$ so the number of 
columns in $B^T$ is $n$ and the number of columns in $A^T$ is $n$, thus $B^T \cdot A^T$ is well defined and
is a $p \times m$ matrix.

Next, we want to show that the corresponding entries of $(A \cdot B)^T$ and $B^T \cdot A^T$ are equal. Consider
$i = 1, \ldots, p$ and $j = 1, \ldots, m$. Then,

\begin{align*}
[(B^T \cdot A^T)]_{ij} &= (B^T)_{i \bullet} \cdot (A^T)_{\bullet j} \\
&= \sum^{n}_{k=1} [B^T]_{ik} \cdot [A^T]_{kj} \\
&= \sum^{n}_{k=1} [B]_{ki} \cdot [A]_{jk} \\
&= \sum^{n}_{k=1} [A]_{jk} \cdot [B]_{ki}.
\end{align*}

Also,

\begin{align*}
[(A \cdot B)^T]_{ij} &= [A \cdot B]_{ji} \\
&= A_{j \bullet} \quad B_{\bullet i} \\
&= \sum_{k = 1}^{n} [A]_{jk} \cdot [B]_{ki}.
\end{align*}

Therefore, $[(AB)^T]_{ij} = [(B^T \cdot A^T)]_{ij}$. Thus $(A \cdot B)^T = B^T \cdot A^T$.

\end{proof}

\begin{example}
Let $A : m \times n$ matrix. Show that $A \cdot A^T$ and $A^T \cdot A$ are well defined and symmetric.

\begin{proof}
Since $A : m \times n$ then $A^T : n \times m$. Therefore $A \cdot A^T$ is well defined and is $m \times m$.
Also, $A^T$ has $m$ columns and $A$ has $m$ rows and so $A^T \cdot A$ is well defined and is an
$n \times n$ matrix.

Let $B = A \cdot A^T$, then

\begin{align*}
B^T &= (A \cdot A^T)^T \\
&= (A^T)^T \cdot A^T \\
&= A \cdot A^T \\
&= B.
\end{align*}

Thus, $B^T = B$, so $B$ is symmetric. Likewise we can show that $A^T \cdot A$ is symmetric.
\end{proof}
\end{example}

\begin{definition}
Let $A$ be an $n \times n$ matrix. The trace of $A$ is the summation of the diagonal entries, i.e.,

\[
trace(A) = \sum^{n}_{i = 1} a_{ii},
\]

where $A = [a_{ij}]$.
\end{definition}

\begin{example}
Let $A = \begin{bmatrix} 1 & 2 \\ 3 & -4 \end{bmatrix}_{2 \times 2}$. Then $trace(A) = 1 + (-4) = -3$.
\end{example}

\begin{example}
$trace(I_n) = n$
\end{example}

\begin{lemma}
Let $A : m \times n$ and $B : n \times m$ such that $AB$ and $BA$ are well defined. Note that $AB$ and $BA$
must be square matrices. Then $trace(A \cdot B) = trace(B \cdot A)$.
\end{lemma}

\begin{proof}
\begin{align*}
trace(A \cdot B) &= \sum_{i = 1}^{m} [A \cdot B]_{ii} \\
&= \sum_{i = 1}^{m} ( A_{i \bullet} B_{\bullet i} ) \\
&= \sum_{i = 1}^{m} \left( \sum_{k = 1}^{n} [A]_{ik} [B]_{ki} \right) \\
&= \sum_{k = 1}^{n} \left( \sum_{i = 1}^{m} [A]_{ik} [B]_{ki} \right) \\
&= \sum_{k = 1}^{n} \left( \sum_{i = 1}^{m} [B]_{ki} [A]_{ik} \right) \quad \quad (\text{in parathenases is } B_{k \bullet} A_{\bullet k}) \\
&= \sum_{k = 1}^{n} \left( B_{k \bullet} A_{\bullet k} \right) \\
&= \sum_{k = 1}^{n} [BA]_{kk} \\
&= trace(BA).
\end{align*}
\end{proof}

\paragraph{Word of caution}

\begin{itemize}
\item $trace(A \cdot B) \neq trace(A) \cdot trace(B)$.
Counterexample: Let $A = B = I_n$. Then $trace(A) = trace(B) = n$, but $A \cdot B = I_n$ and so $trace(AB) = n$.
Therefore, $trace(A \cdot B) = n$ and $trace(A) \cdot trace(B) = n^2$.
\end{itemize}

\paragraph{Cycling property of trace}
Let $A, B, C : n \times n$ matrices. It can be shown that $trace(A \cdot B \cdot C) = trace(C \cdot A \cdot B)$.

\section{Partition of a matrix}

Let $A : m \times n$ matrix.

\[
A = \left[
\begin{array}{c|c}
A_{11} & A_{12} \\ \hline
A_{21} & A_{22}
\end{array}
\right]
\]

where $A_{ij}$ is the $(i, j)$-block of $A$.

\begin{example}
\[
A =
\left[
\begin{array}{cc|c}
0 & 1 & 1 \\
1 & 0 & 1 \\ \hline
-2 & 3 & 0
\end{array}
\right]
\]

In the above matrix, $A_{11} = \begin{bmatrix} 0 & 1 \\ 1 & 0 \end{bmatrix}$, $A_{12} = \begin{bmatrix}
1 \\ 1 \end{bmatrix}$, $A_{21} = \begin{bmatrix} -2 & 3 \end{bmatrix}$, and
$A_{22} = \begin{bmatrix} 0 \end{bmatrix}$.
\end{example}

\section{Block matrix multiplication}

\begin{example}
Let $A = \begin{bmatrix} A_{11} & A_{12} \\ A_{21} & A_{22} \end{bmatrix}$ and
$B = \begin{bmatrix} B_{11} & B_{12} & B_{13} \\ B_{21} & B_{22} & B_{23} \end{bmatrix}$. Assume that 
$(A_{ik} , B_{kj})$ is conformable and so $A_{ik} \cdot B_{kj}$ is well defined and so $A \cdot B$ is
well defined. Then

\[
A \cdot B =
\begin{bmatrix}
( A_{11} B_{11} + A_{12} B_{21} ) & ( A_{11} B_{12} + A_{12} B_{22} ) & ( A_{11} B_{13} + A_{12} B_{23} )\\
( A_{21} B_{11} + A_{22} B_{21} ) & ( A_{21} B_{12} + A_{22} B_{22} ) & ( A_{21} B_{13} + A_{22} B_{23}  )
\end{bmatrix}.
\]
\end{example}

\begin{example}
Let $A = \begin{bmatrix} C & I \\ I & 0 \end{bmatrix}$ and $B = \begin{bmatrix} I & 0 \\ D & E \end{bmatrix}$. Then

\[
AB = 
\begin{bmatrix}
( CI + ID ) & ( C0 + IE ) \\
( II + 0D ) & ( I0 + 0E )
\end{bmatrix}
=
\begin{bmatrix}
C + D & E \\
I & 0
\end{bmatrix}.
\]
\end{example}

\begin{example}
The product of upper block triangular matrices is also block upper triangular. Let
$A = \begin{bmatrix} A_{11} & A_{12} \\ 0 & A_{22} \end{bmatrix}$ and 
$B = \begin{bmatrix} B_{11} & B_{12} \\ 0 & B_{22} \end{bmatrix}$. Then

\[
AB =
\begin{bmatrix}
A_{11} B_{11} & ( A_{11} B_{12} + A_{12} B_{22} ) \\
0 & A_{22} B_{22}
\end{bmatrix}.
\]
\end{example}

\section{Linear functions}

Let $V$ and $W$ be two sets (i.e., vector spaces) with the same field, like $\mathbb{R}$ or $\mathbb{C}$,
and two basic operations on $V$ and $W$:

\begin{enumerate}

\item vector addition:
\begin{itemize}
\item[] $u + w \in V \quad \quad \forall u, v \in V$
\item[] $x + y \in W \quad \quad \forall x, y \in W$
\end{itemize}

\item scalar multiplication:
\begin{itemize}
\item[] $\alpha \cdot u \in V \quad \quad \forall u \in V, \alpha \in \mathbb{R} \text{ or } \mathbb{C}$
\item[] $\alpha \cdot x \in W \quad \quad \forall x \in W, \alpha \in \mathbb{R} \text{ or } \mathbb{C}$
\end{itemize}

\end{enumerate}

\begin{example}

\begin{enumerate}
\item $V = \mathbb{R}^n$ : n-dimensional Euclidean space and $W = \mathbb{R}^m$ is m-dimensional Euclidean space.
\item $V = \mathbb{C}^n$ : n-dimensional complex space and $W = \mathbb{C}^m$ is m-dimensional complex space.
\item $V = \mathbb{R}^{m \times n}$: the space of all $m \times n$ real matrices.
\end{enumerate}
\end{example}

\begin{definition}
A function $f : V \rightarrow W$ is \textbf{linear} if:
\begin{enumerate}
\item $f(u + v) = f(u) + f(v)$, $\forall u, v \in V$. This is sometimes called the superposition property.
\item $f(\alpha \cdot u) = \alpha \cdot f(u)$, $\forall u \in V$ and scalar 
$\alpha \in \mathbb{R} \text{ or } \mathbb{C}$. This is sometimes called the scalar property.
\end{enumerate}
\end{definition}

\begin{example}
Given an $m \times n$ real matrix $A$, define the function $f : \mathbb{R}^n \rightarrow \mathbb{R}^m$ as
$f(x) = A \cdot x$ for $x \in \mathbb{R}^n$ where $x = \begin{bmatrix} x_1 \\ x_2 \\ \vdots \\ x_n \end{bmatrix} \in \mathbb{R}^n$.

\begin{claim}
$f : \mathbb{R}^n \rightarrow \mathbb{R}^m$ is linear.
\end{claim}

\begin{proof}
By question 3 from homework 1,
$f( u + v ) = A (u + v) = Au + Av = f(u) + f(v)$ for all $u, v \in \mathbb{R}^n$. Also, for scalar $\alpha$ and
$u \in \mathbb{R}^n$, $f(\alpha \cdot u) = A ( \alpha \cdot u ) = \alpha (A \cdot u) = \alpha f(u)$.
\end{proof}

\end{example}

\begin{example}
Consider the trace function $trace : \mathbb{R}^{n \times n} \rightarrow \mathbb{R}$ where 
$\mathbb{R}^{n \times n}$ is the space of all $n \times n$ matrices.

\textbf{Show}: $trace(\cdot)$ is linear. 

\begin{proof}
First we show $trace(\cdot)$ satisfies the superposition property. Given $A = [a_{ij}]$ and
$B = [b_{ij}]$ in $\mathbb{R}^{n \times n}$ then

\begin{align*}
trace(A + B) &= \sum^{n}_{i = 1} [A + B]_{ii} \\
&= \sum^{n}_{i = 1} [A]_{ii} + [B]_{ii} \\
&= \sum^{n}_{i = 1} [A]_{ii} + \sum^{n}_{i = 1} [B]_{ii} \\
&= trace(A) + trace(B).
\end{align*}

Now we show $trace(\cdot)$ satisfies the scalar property. Let $\alpha \in \mathbb{R}$ and 
$A = [a_{ij}] \in \mathbb{R}^{n \times n}$. Then

\begin{align*}
trace(\alpha A) &= \sum^{n}_{i = 1} [\alpha \cdot A]_{ii} \\
&= \sum^{n}_{i = 1} \alpha [A]_{ii} \\
&= \alpha \sum^{n}_{i = 1} [A]_{ii} \\
&= \alpha \cdot trace(A).
\end{align*}

Therefore, $trace(\cdot)$ is linear.
\end{proof}
\end{example}

\begin{proposition}
A function $f : V \rightarrow W$ is linear if and only if 
$$f(\alpha \cdot u + v) = \alpha \cdot f(u) + f(v), \quad \forall u, v \in V, \alpha \in
\mathbb{R} \text{ or } \mathbb{C}.$$
\end{proposition}

\begin{proof}
$(\Rightarrow)$ Suppose $f$ is linear. Then for any $u, v \in V$ and scalar .$\alpha$,

$$f(\alpha \cdot u + v) = f(\alpha \cdot u) + f(v) = \alpha \cdot f(u) + f(v)$$

where the first equality follows from the superposition property and the second from
the scalar property of linear functions.

$(\Leftarrow)$ For MATH 603
\end{proof}


\section{Matrix Inverse}

\begin{definition}
A $n \times n$ matrix $A$ is invertible (i.e., non-singular) if (and only if) there exists
an $n \times n$ matrix $B$ such that

$$ A \cdot B = I_n \text{ and } B \cdot A = I_n.$$

Here $B$ is called an inverse of $A$, denoted $A^{-1}$.
\end{definition}

\begin{claim}
If $A$ is invertible then its inverse is unique.
\end{claim}

\begin{proof}
Since $A$ is invertible, there exists a matrix $B$ such that $A \cdot B = B \cdot A = I_n$. Now
suppose there exists a matrix $B^{\prime}$ such that $A \cdot B^{\prime} = B^{\prime} \cdot A = I_n$.
Then since $A \cdot B^{\prime}$,

\begin{align*}
B = B \cdot I_n &= B \cdot (A \cdot B^{\prime}) \\
&= (B \cdot A) \cdot A^{\prime} \\
&= I_n B^{\prime} \\
&= B^{\prime}.
\end{align*}

Thus $B = B^{\prime}$. Therefore, the inverse of $A$ is unique.
\end{proof}

\begin{proposition}
Let $A$ be an $n \times n$ matrix. The following are equivalent:
\begin{enumerate}[label = (\arabic*)]
\addtocounter{enumi}{-1}

\item $A$ is invertible.

\item $A$ is row equivalent to $I_n$.

\item The equation $Ax = 0$ has the solution $x = 0$ only.

\item The equation $Ax = b$ has a solution for any n-vector b.

\item $rank(A) = n$.

\item There exists an $n \times n$ matrix $B$ such that $B \cdot A = I_n$.

\item There exists an $n \times n$ matrix $C$ such that $A \cdot C = I_n$.
\end{enumerate}

\end{proposition}

\begin{proof}
In the following we assume as facts (2) and (3) to prove $5) \Rightarrow 0)$ and
$6) \Rightarrow 0)$.

\begin{itemize}
\item[] \underline{$5) \Rightarrow 0)$}. By 5) there exists a matrix such that $B \cdot A = I_n$. It is 
sufficient to show that the equation $Ax = 0$ has the solution $x = 0$ only to prove that
$A$ is invertible.

Let $u$ be a solution to $Ax = 0$, i.e., $Au = 0$. If we right multiply the latter equation
by $B$ we have $B(Au) = B \cdot 0 = 0$. Now consider

$$B(A\cdot u) = (B \cdot A) u = I_n \cdot u = u.$$

Thus $u = 0$. Therefore, 2) holds and since 2) implies 0) then 5) implies 0).

\item[] \underline{$6) \Rightarrow 0)$} By 6) there exists a matrix $C$ such that $AC = I_n$.

\textbf{Claim:} For any n-vector $b$, $Ax = b$ has a solution.

Since there exists a matrix $C$ such that $A \cdot C = I_n$ then given any n-vector $b$,
$(A \cdot C) \cdot b = A \cdot (C \cdot b) = I_n \cdot n = b$. Thus 
$A \cdot (C \cdot b) = b$ implies that $x = C \cdot b$ is a solution to the equation $Ax = b$.
Thus, 3) holds $\Rightarrow$ 0) holds.
\end{itemize}
\end{proof}

\section{Properties of an invertible matrix}

Suppose $A$ is invertible.

\begin{enumerate}[label = (\arabic*)]

\item $A^{-1}$ is invertible, and $(A^{-1})^{-1}$.
\begin{proof}
$A \cdot A^{-1} = A^{-1} \cdot A = I_n$. Fill in details.
\end{proof}

\item $A^T$ and $A^*$ are invertible, and $(A^T)^{-1} = (A^{-1})^T$ and $(A^*)^{-1} = (A^{-1})^*$
\begin{proof}
Since $A$ is invertible then

\begin{align*}
A \cdot A^{-1} = A^{-1} \cdot A = I_n
&\Rightarrow (A \cdot A^{-1})^T = (A^{-1} \cdot A)^T = (I_n)^T = I_n \\
&\Rightarrow (A^{-1})^T \cdot A^T = A^T \cdot (A^{-1})^T = I_n
\end{align*}

Therefore $A^T$ is invertible and $(A^T)^{-1} = (A^{-1})^T$.
\end{proof}

\item Let $B$ be another invertible matrix. Then $A \cdot B$ is invertible, and its inverse
$(A \cdot B)^{-1} = B^{-1} \cdot A^{-1}$.
\begin{proof}
\begin{align*}
A \cdot B \cdot (B^{-1} \cdot A^{-1}) &= A \cdot (B \cdot B^{-1}) \cdot A^{-1} \\
&= A \cdot A^{-1} \\
&= I_n.
\end{align*}
Therefore $(A \cdot B)^{-1} = B^{-1} \cdot A^{-1}$.
\end{proof}
\end{enumerate}

\begin{corollary}
This is an extension of 3 above. Let $A_1, \ldots, A_k$ be invertible matrices. Then
$A_1 \cdot A_2 \cdots A_k$ is invertible, and 
$(A_1 \cdot A_2 \cdots A_k)^{-1} = A^{-1}_k \cdots A^{-1}_2 \cdot A^{-1}_1$. Can use
proof by induction on $k$.
\end{corollary}

\section{Examples}

\begin{example}
Let $A$ be a square matrix such that $I - A$ is invertible. Show:
\[
A \cdot (I - A)^{-1} = (I - A)^{-1} \cdot A \quad \text{ (commutative) }.
\]
\begin{proof}
Compute
\begin{align*}
A \cdot (I - A) &= A \cdot I - AA = A - A^2 \\
&= I \cdot A - A \cdot A \\
&= (I - A) \cdot A.
\end{align*}
Then 
\begin{align*}
A \cdot (I - A)(I - A)^{-1} = (I - A) \cdot A \cdot (I - A)^{-1} &\Rightarrow
A = (I - A) \cdot A \cdot (I - A)^{-1} \\
&\Rightarrow (I - A)^{-1} \cdot = A \cdot (I - A)^{-1}.
\end{align*}
\end{proof}
\end{example}

\begin{example}
We will use the following preliminary results in this example.

\begin{enumerate}[label = (\arabic*)]
\item  Given $x, y \in \mathbb{R}^n$ (i.e., $x, y \in \mathbb{R}^{n \times 1}$ are n-column vectors).
The following is a scalar:

\[
x^T \cdot y = (x^T \cdot y)^T = y^T \cdot x \quad \text{ (standard inner product) }.
\]



\item Given $x \in \mathbb{R}^n$ such that $x^T x = 0$, then $x = 0$.

\begin{proof}
Write this as 
$\displaystyle
\begin{bmatrix}
x_1 \\ x_2 \\ \vdots \\ x_n
\end{bmatrix}
\in \mathbb{R}^n.
$

Then
\begin{align*}
x^T \cdot x &= 
\begin{bmatrix}
x_1 & x_2 & \cdots & x_n
\end{bmatrix}
\begin{bmatrix}
x_1 \\
x_2 \\
\vdots \\
x_n
\end{bmatrix} \\
&=
x_1^2 + x_2^2 + \cdots + x_n^2 \\
&= 0.
\end{align*}

Since $x^Tx = 0$, then $x_1^2 + x_2^2 + \cdots + x^2_n = 0$.

If $x_i \neq 0$ for some $i$ then $x^Tx > 0$. This shows that each $x_i = 0$ or $x = 0$.
\end{proof}

\item Show $x^T S x = 0$, $\forall x \in \mathbb{R}^n$.
\begin{proof}
For any $x \in \mathbb{R}^n$

\begin{align*}
x^TSx &= x^T \cdot (Sx) \\
&= (S \cdot x)^T \cdot x \quad \text{ (by preliminary result 1) }\\
&= (x^T \cdot S^T) \cdot x \\
&= (x^T \cdot -S) \cdot x \quad \text{ (S is skew symmetric) }\\
&= x^T \cdot (-1)\cdot S \cdot x \\
&= (-1) \cdot x^TS \cdot x.
\end{align*}

Therefore, 
\begin{align*}
x^TSx = -x^TSx &\Rightarrow 2 \cdot x^T S x = 0 \\
&\Rightarrow x^TSx = 0.
\end{align*}
\end{proof}
\end{enumerate}

Let $S$ be a real skew symmetric matrix, i.e., $S = -S^T$ or $S^T = -S$.

Show that $\forall \alpha \in \mathbb{R}$, $I + \alpha \cdot S$ is invertible.

\begin{proof}
Let $\alpha \in \mathbb{R}$ be arbitrary. To show $I + \alpha \cdot S$ is invertible
it suffices to show that 

\[
(I + \alpha S) \cdot x = 0
\]

has the solution $x = 0$ only.

Suppose $\hat{x}$ is an arbitrary solution to $(I + \alpha S) \cdot x = 0$. Therefore,

\begin{align*}
(I + \alpha S) \cdot \hat{x} = 0 &\Rightarrow \hat{x}^T (I + \alpha S) \hat{x} = \hat{x}^T \cdot 0 \\
&\Rightarrow \hat{x}^T \hat{x} + \alpha \hat{x}^T S \hat{x} = 0 \\
&\Rightarrow \hat{x}^T \hat{x} = 0 \\
&\Rightarrow \text{ By 2), } \hat{x} = 0.
\end{align*}

Therefore $(I + \alpha S) \cdot x = 0$ has the solution $x = 0$ only. Thus $I + \alpha \cdot S$ is
invertible.
\end{proof}

\end{example}


\section{Vector Space}

\begin{definition}
A \textbf{vector space} is a set $V$, along with a field $F$ (which is $\mathbb{R}$ or $\mathbb{C}$),
and two basic operations


\begin{enumerate}[label = (\arabic*)]
\item vector addition ``+'' and
\item scalar multiplication `` $\cdot$ ''
\end{enumerate}

which satisfy:

\begin{itemize}
\item [A1):] $V$ is closed under vector addition, i.e., $x, y \in V$ then $x + y \in V$.
\item [A2):] For any $x, y \in V$, $(x + y) + z = x + (y + z)$ (associative)
\item [A3):] For any $, y \in V$, $x + y = y + x$ (commutative)
\item [A4):] There exists a zero vector in $V$, denoted by 0, such that $x + 0 = x$ and
$0 + x = 0$, $\forall x \in V$.
\item [A5):] For any $x \in V$, there exists a vector in $V$, denoted by $-x$ such that
$x + (-x) = 0$. Here $-x$ is additive inverse of $x$.
\item [M1):] $V$ is closed under scalar multiplication, i.e., for any $x \in V$ and scalar
$\alpha$, $\alpha \cdot x \in V$.
\item [M2):] For any scalar $\alpha$, $\beta$ and $x \in V$,
\[
(\alpha \cdot beta) \cdot x = \alpha (\beta \cdot x).
\]
\item [M3):] For any scalar $\alpha$, and $x, y \in V$,
\[
\alpha \cdot (x + y) = \alpha \cdot x + \alpha \cdot y.
\]
\item [M4):] For any scalars $\alpha$, $\beta$ and $x \in V$
\[
(\alpha + \beta) x = \alpha \cdot x + \beta \cdot x. 
\]
\item [M5):] When $\alpha = 1$, $\alpha \cdot x = x$, $\forall x \in V$.
\end{itemize}

Any $x \in V$ is called a \textbf{vector} in $V$.
\end{definition}

\begin{example} Vector Spaces
\begin{enumerate}[label = (\arabic*)]
\item $\mathbb{R}^n$: the n-dimensional Euclidean space. Likewise, $\mathbb{C}^n$ is also a
vector space.
\item $\mathbb{R}^{m \times n}$: the set of all $m \times n$ real matrices. Likewise, 
$\mathbb{C}^{m \times n}$ is also a vector space.
\begin{itemize}
\item [($2^{\prime}$)] The set of all $n \times n$ upper triangular real matrices is a 
vector space, which is a subspace of $\mathbb{R}^{n \times n}$.
\end{itemize} 
\item The set of all real sequences
\begin{align*}
X + Y &= (x_1 + y_1, x_2 + y_2, \dots), \\
\alpha \cdot X &= (\alpha x_1, \alpha x_2, \ldots).
\end{align*}
This set is a vector space.
\begin{itemize}
\item [($3^{\prime}$)] The set of \textbf{convergent} real sequences is a vector space and is a subspace
of the above vector space.
\item [($3^{\prime\prime}$)] The set of \textbf{bounded} real sequences is also a vector space.
\end{itemize}
\item The set of all real-valued continuous functions on $\mathbb{R}$ (or $[a,b] \subseteq \mathbb{R}$).
\item The set of all \textbf{divergent} real sequences is \textbf{NOT} a vector space.

This is because:
\begin{enumerate}[label = \arabic*)]
\item It has no "zero vector"
\item The closure under scalar multiplication fails. If we multiply any element by scalar 0, then the
sequence will be convergent.
\item The closure under vector addition fails. Add $-x$ to $x$ and will be zero vector, which is
convergent. 
\end{enumerate}
\end{enumerate}
\end{example}

\section{Vector Spaces: Important Facts}

\begin{enumerate}
\item $V$ has the unique zero vector 0. (Related to A4)
\item For any $x \in V$, there exists the unique additive inverse $-x$, and $-x = (-1) \cdot x$.
\item For any $x \in V$, $0 \cdot x = 0 \in V$.
\item For any scalar $\alpha$, $\alpha \cdot 0 = 0 \in V$.
\item A finite sum of vectors in $V$ belongs to $V$. Can use induction on the number of vectors in
sum to prove.
\end{enumerate}

\section{Subspace}

\begin{definition}
Let $(V, \mathbb{F}, +, \cdot)$ be a vector space. If a subset $S$ of $V$ is a vector space (along with
the same field $\mathbb{F}$ and basic operations of $+$ and $\cdot$ of $V$) then $S$ is a \textbf{subspace}
of $V$.
\end{definition}

\begin{example}
Let $(V, \mathbb{F}, +, \cdot)$ be a vector space.
\begin{itemize}
\item $V$ is a subspace of itself. So every vector space has at least one subspace.
\item $S = \{0\}$, the singleton set of the zero vector in $V$ is a subspace of $V$ and is the zero
or trivial subspace. Provide details of proof.
\end{itemize}
\end{example}

\begin{remark}
From the above example we can conclude that every vector space has at least 2 subspaces.
\end{remark}

\begin{proposition}
Let $S$ be a subset of a vector space $(V, \mathbb{F}, +, \cdot)$. Then $S$ is a subspace of $V$
if and only if
\begin{enumerate}
\item $S$ is closed under ``$+$''; (A1) and
\item $S$ is closed under ``$\cdot$'' (M1).
\end{enumerate}
\end{proposition}

\begin{proof}
$(\Leftarrow)$ Suppose $S$ is a subspace of $V$. Since $S$ is a subspace of $V$, then $S$ is
a subspace and satisfies A1-A5 and M1-M5. Thus, A1 and M1 hold.

$(\Rightarrow)$ Suppose the subset $S$ satisfies A1 and M1.

\vspace{0.2cm}

We want to show: $S$ satisfies A2-A5 and M2-M5.

\vspace{0.2cm}

Since $S \subseteq V$ and $V$ and $V$ is a vector space then A.2, A.3, and M.2-M.5 hold on $V$ and
therefore hold on $S$ as well.

\vspace{0.2cm}

Then it suffices to show that $S$ satisfies A.4 and A.5.

\begin{itemize}

\item [A.4] Pick an arbitrary $x \in S$. Then $0 \cdot x = 0$ for the scalar $0$ and any vector $x$, and since
$S$ is closed under ``$\cdot$'', then the zero vector is in $S$. Also, $0 + x = x + 0 = x$ for all $x \in S$.

\item [A.5] For any $x \in S$, recall that its additive inverse $-x = (-1) \cdot x$ by Fact 2. Since $S$ 
is closed under ``$\cdot$'', then $(-1) \cdot x \in S \implies -x \in S$. Thus, A.5 holds.

\end{itemize}

Therefore, we have shown that $S$ is a vector space and so $S$ is a subspace of $V$.

\end{proof}


\begin{example}

Examples of sets that are subspaces.

\begin{enumerate}[label = (\arabic*)]

\item Let $V = \mathbb{R}^3$ and let 
$\displaystyle S = \left\{ \begin{bmatrix} x_1 \\ 0 \\ 0 \end{bmatrix} :
x_1 \in \mathbb{R} \right\} $. Show that $S$ is a subspace of $\mathbb{R}^3$.

\begin{proof}
Let $x, y \in S$. Then

\[
x = 
\begin{bmatrix}
x_1 \\
0 \\
0
\end{bmatrix},
\quad
y =
\begin{bmatrix}
y_1 \\
0 \\
0
\end{bmatrix}
.
\]

The sum of $x$ and $y$ is

\[
x + y = 
\begin{bmatrix}
x_1 + y_1 \\
0 \\
0
\end{bmatrix}.
\]

Therefore, $S$ is closed under ``$+$''. Given any $\alpha \in \mathbb{R}$ and
$x = \begin{bmatrix} x_1 \\ 0 \\ 0 \end{bmatrix} \in S$,

\[
\alpha \cdot x =
\begin{bmatrix}
\alpha \cdot x_1 \\
0 \\
0
\end{bmatrix}
\in S.
\]

Therefore, $S$ is closed under scalar multiplication. Thus $S$ is a subspace of $V$.
\end{proof}

\item Let $V = \mathbb{R}^{n \times n}$.

\begin{enumerate}[label = (\roman*)]
\item Let $S_1$ be the set of all $n \times n$ diagonal matrices in $\mathbb{R}^{n \times n}$.
Since $S$ is closed under standard matrix addition and scalar multiplication then $S$ is a
subspace of $V$.

\item Let $S_2$ be the set of all symmetric matrices in $\mathbb{R}^{n \times n}$.

Let $A, B \in S_2$, and therefore $A = A^T$ and $B = B^T$. Now consider the transpose
of $A + B$,

\[
(A + B)^T = A^T + B^T = A + B,
\]

which shows that $(A + B)$ is symmetric and therefore $(A + B) \in S_2$. 

Now let $\alpha \in \mathbb{R}$ and $A \in S_2$ and consider their product

\[ (\alpha A)^T = \alpha A^T = \alpha T, \] 

which shows that $\alpha A$ is
symmetric and therefore in $S_2$. Since $S_2$ is closed under vector addition
and scalar multiplication then $S_2$ is a subspace of $V$.

\end{enumerate}

\end{enumerate}

\end{example}


\begin{example}

Examples of sets that are \textbf{not} subspaces.

\begin{enumerate}[label = (\arabic*)]

\item Let $V = \mathbb{R}^2$ and 
$S = \left\{ \begin{bmatrix} x_1 \\ x_2 \end{bmatrix} : x_1 \leq 0, x_2 \geq 0 \right\}$.
Let 

\[ x = \begin{bmatrix} x_1 \\ x_2 \end{bmatrix}, y = \begin{bmatrix} y_1 \\ y_2 \end{bmatrix} \in S. \]

Therefore, $x_1 \leq 0$, $y_1 \leq 0$ and $x_2 \geq 0$, $y_2 \geq 0$. The sum of $x$ and $y$ is

\[
x + y =
\begin{bmatrix}
x_1 + y_2 \\
x_2 + y_2
\end{bmatrix}
\]

where $x_1 + y_1 \leq 0$ and $x_2 + y_2 \leq 0$. Thus $x + y \in S$ and so $S$ is closed under
vector addition.

Given an arbitrary $\alpha \in \mathbb{R}$ and $x = \begin{bmatrix} x_1 \\ x_2 \end{bmatrix} \in S$.
For example, $x = \begin{bmatrix} -1 \\ 2 \end{bmatrix} \in S$ and $\alpha = -1$. Then the product
$\alpha \cdot x = \begin{bmatrix} 1 \\ -2 \end{bmatrix} \notin S$ and therefore $S$ is \textbf{not}
closed under vector multiplication. Thus, $S$ is not a subspace of $V$.

\item Let $V = \mathbb{R}^{2 \times 2}$ and $S$ is the subset of all singular (i.e., not
invertible) matrices in $\mathbb{R}^{2 \times 2}$. Let $A \in S$ and $\alpha \in \mathbb{R}$. If
$\alpha = 0$, then $\alpha \cdot A = 0$ is singular. If $\alpha \neq 0$, then $\alpha \cdot A$ is
singular because otherwise $\alpha A \cdot (1/\alpha \cdot B) = I$ and so $A \cdot B = I$. 
This contradicts our assumption that $A$ is not invertible. Therefore, $\alpha A$ is singular.
Thus, $S$ is closed under vector multiplication.

\textbf{Claim:} $S$ is not closed under ``+''. 

Let $A = \begin{bmatrix} 1 & 0 \\ 0 & 0 \end{bmatrix}$, and 
$B = \begin{bmatrix} 0 & 0 \\ 0 & 1 \end{bmatrix}$. Both $A$ and $B$ are because if a matrix
is in reduced row echelon form and any diagonal element is 0 then the matrix is singular. Thus
$A , B \in S$. However, $A + B = \begin{bmatrix} 1 & 0 \\ 0 & 1 \end{bmatrix}$, which is
the identity matrix and thus is invertible or nonsingular. Therefore, $A + B \notin S$ and
so $S$ is not closed under ``+''.

\end{enumerate}
\end{example}

\section{Linear Combination}

\begin{definition}
Let $\{ v_1, v_2, \ldots, v_p \}$ be a finite set of vectors in a vector space $V$. Given 
scalars $\alpha_1, \alpha_2, \ldots, \alpha_p$. The \textbf{linear combination} of
$v_1, v_2, \ldots, v_p$ using $\alpha_1, \alpha_2, \dots, \alpha_P$ is

\[
\alpha_1 v_1 + \alpha_2 v_2 + \cdots + \alpha_p v_p
\]

where the $\alpha_i$ for $i = 1, 2, \ldots, p$ are called the weights or coefficients.
\end{definition}

\begin{remark}
\begin{itemize}
\item By definition, $\alpha \cdot x \in V$ so $\alpha_1 v_1 \in V, \ldots, \alpha_p v_p \in V$.
\item Any finite sum of vectors from $V$ is in $V$.
\end{itemize}
\end{remark}

\begin{definition}
Given a finite set $S = \{v_1, \ldots, v_p \}$ in $V$, the set spanned by $v_1, \ldots, v_p$
(or $S$) is

\begin{align*}
span(S) &= \{ \alpha_1 v_1 + \cdots + \alpha_2 v_2 + \cdots + \alpha_p v_p : \text{ scalars  }
\alpha_i, i = 1, \ldots, p \} \\
&= \text{ the set of all linear combinations of } v_1, \dots, v_p.
\end{align*}

Clearly, $span(S) \subseteq V$, where $S$ is a spanning set of $span(S)$.
\end{definition}

\begin{proposition}
The set $span(S)$ is a subspace of $V$.
\end{proposition}

\begin{proof}
First we show that $span(S)$ is closed under ``+''. Let $x, y$ be two vectors in $span(S)$.
Therefore, there exists scalars $\alpha_1, \ldots, \alpha_p$ such that

\[ x = \alpha_1 v_1 + \cdots + \alpha_p v_p. \]

Similarly, there exists scalars $\beta_1, \ldots, \beta_p$ such that 

\[ y = \beta_1 v_1 + \cdots + \beta_p v_p. \]

Then the sum of $x$ and $y$ is

\begin{align*}
x + y &= (\alpha_1 v_1 + \cdots + \alpha_p v_p) + (\beta_1 v_1 + \cdots + \beta_p v_p) \\
&= (\alpha_1 v_1 + \beta_1 v_1) + (\alpha_2 v_2 + \beta_2 v_2) + \cdots + (\alpha_p v_p + \beta_p v_p) \\
&= (\alpha_1 + \beta_1) v_1 + (\alpha_2 + \beta_2) v_2 + \cdots + (\alpha_p + \beta_p) v_p. 
\end{align*}

Therefore, $x + y$ is a linear combination of $v_1, \ldots, v_p$. Thus $x + y \in span(S)$ and so
$span(S)$ is closed under ``+''.

Now we show that $span(S)$ is closed under ``$\cdot$''. Let $x \in span(S)$ and $\gamma$ be a scalar.
Therefore the exists scalars $\alpha_1, \ldots, \alpha_p$ such that

\[
x = \alpha_1 v_1 + \cdots + \alpha_p v_p.
\]

Then the product of $\gamma$ and $x$ is

\begin{align*}
\gamma \cdot x &= \gamma \cdot (\alpha_1 v_1 + \cdots + \alpha_p v_p) \\
&= (\gamma \cdot \alpha_1) v_1 + \cdots + (\gamma \cdot \alpha_p) v_p.
\end{align*}

Therefore, $\gamma \cdot x$ is a linear combination of $v_1, \dots, v_p$ and so
$\gamma \cdot x \in span(S)$. Thus $span(S)$ is closed under ``$\cdot$''. Thus,
$span(S)$ is a subspace.
\end{proof}

Here $span(S)$ is a \underline{subspace spanned} by $v_1, \dots, v_p$ (or $S$).

\begin{proposition}
Let $S = \{ v_1, \ldots, v_p \}$. Then $S \subseteq span(S)$.
\end{proposition}

\begin{proof}
For each $v_i$,

\[
v_i = 0 \cdot v_1 + \cdots + 0 \cdot v_{i-1} + \underline{1 \cdot v_i} + 0 \cdot 
v_{i + 1} + \cdots + 0 \cdots v_p = 1 \cdot v_i.
\]

Then $v_i \in span(S)$ and therefore $S \subseteq span(S)$.
\end{proof}

\begin{remark}
In the previous proof, $S$ is generally not a subspace while $span(S)$ is a subspace.
\end{remark}

\begin{proposition}
$span(S)$ is the smallest subset containing $S$.
\end{proposition}

\begin{proof}
Need to fill in
\end{proof}

\begin{definition}
Given a subspace $W$ of $V$, if a finite set $S$ spans $W$, i.e., $W = span(S)$, then
$W$ is \textbf{spanned by} $S$ or $S$ \textbf{spans} $W$.
\end{definition}

\begin{example}
\begin{enumerate}[label = (\arabic*)]

\item Show that $\{ e_1, e_2 \}$ spans $\mathbb{R}^2$ or $\mathbb{R}^2 = span\{e_1, e_2 \}$
where $e_1 = \begin{bmatrix} 1 \\ 0 \end{bmatrix}$ and
$e_2 = \begin{bmatrix} 0 \\ 1 \end{bmatrix}$.

\begin{proof}
It suffices to show that any $x = \begin{bmatrix} x_1 \\ x_2 \end{bmatrix}$ is a linear
combination of $e_1$ and $e_2$. Then

\[
x =
\begin{bmatrix}
x_1 \\
x_2
\end{bmatrix}
=
x_1
\begin{bmatrix}
1 \\
0
\end{bmatrix}
+
x_2
\begin{bmatrix}
0 \\
1
\end{bmatrix}
=
\begin{bmatrix}
x_1 \\
0
\end{bmatrix}
+
\begin{bmatrix}
0 \\
x_2
\end{bmatrix}.
\]

Therefore, $\{ e_1, e_2 \}$ spans $\mathbb{R}^2$.
\end{proof}

\item Show $\left\{ e_1, e_2, \begin{bmatrix} 1 \\ 1 \end{bmatrix} \right\}$ spans $\mathbb{R}^2$.

\begin{proof}
It suffices to show any $x \in \mathbb{R}^2$ is a linear combination of $e_1$, $e_2$ and
$\begin{bmatrix} 1 \\ 1 \end{bmatrix}$. For any 
$x = \begin{bmatrix} x_1 \\ x_2 \end{bmatrix} \in \mathbb{R}^2$,

\[
x =
\begin{bmatrix}
x_1 \\
x_2
\end{bmatrix}
=
x_1
\begin{bmatrix}
1 \\
0
\end{bmatrix}
+
x_2
\begin{bmatrix}
0 \\
1
\end{bmatrix}
+
0
\cdot
\begin{bmatrix}
1 \\
1
\end{bmatrix}.
\]
\end{proof}

Therefore, $\{e_1, e_2\}$ spans $\mathbb{R}^2$.

\item Does $\begin{bmatrix} 1 \\ 0 \end{bmatrix}, \begin{bmatrix} 1 \\ 1 \end{bmatrix}$
spans $\mathbb{R}^2$?

\vspace{0.2cm}

\textbf{Answer:} Yes

\vspace{0.2cm}

For any $x = \begin{bmatrix} x_1 \\ x_2 \end{bmatrix} \in \mathbb{R}^2$, we need to
show that there exists $\alpha_1, \alpha_2 \in \mathbb{R}$ such that

\[
\begin{bmatrix}
x_1 \\
x_2
\end{bmatrix}
=
\alpha_1
\begin{bmatrix}
1 \\
0
\end{bmatrix}
+
\alpha_2
\begin{bmatrix}
1 \\
1
\end{bmatrix}
= 
\begin{bmatrix}
\alpha_1 \\
0
\end{bmatrix}
+
\begin{bmatrix}
\alpha_1 \\
\alpha_2
\end{bmatrix}
=
\begin{bmatrix}
\alpha_1 + \alpha_2 \\
\alpha_2
\end{bmatrix}.
\]

This leads to the linear equation with unknowns $\alpha_1$ and $\alpha_2$:


\begin{align*}
\begin{cases}
\alpha_1 + \alpha_2 &= x_1 \\
\alpha_2 &= x_2
\end{cases}
\end{align*}

which implies

\begin{align*}
\begin{cases}
\alpha_2 &= x_2 \\
\alpha_1 &= x_1 - x_2
\end{cases}.
\end{align*}

Therefore, $\left\{ \begin{bmatrix} 1 \\ 0 \end{bmatrix}, \begin{bmatrix} 1 \\ 1 \end{bmatrix} \right\}$
spans $\mathbb{R}^2$.

\item Show that $\{ E^{11}, E^{12}, E^{21}, E^{22} \}$ spans $\mathbb{R}^{2 \times 2}$ where
\[
E^{11} = \begin{bmatrix} 1 & 0 \\ 0 & 0 \end{bmatrix},
E^{12} = \begin{bmatrix} 0 & 1 \\ 0 & 0 \end{bmatrix},
E^{21} = \begin{bmatrix} 0 & 0 \\ 1 & 0 \end{bmatrix},
\text{ and } E^{22} = \begin{bmatrix} 0 & 0 \\ 0 & 1 \end{bmatrix}.
\]
\end{enumerate}

For any matrix $A = \begin{bmatrix} a_{11} & a_{12} \\ a_{21} & a_{22} \end{bmatrix} \in \mathbb{R}^{2 \times 2}$,

\[
A = a_{11} \cdot E^{11} + a_{12} \cdot E^{12} + a_{21} \cdot E^{21} + a_{22} \cdot E^{22}.
\]

\item Let $W$ be the subspace of all $2 \times 2$ symmetric matrices in $\mathbb{R}^{2 \times 2}$. Show

\[
\left\{
\begin{bmatrix}
1 & 0 \\
0 & 0
\end{bmatrix},
\begin{bmatrix}
0 & 0 \\
0 & 1
\end{bmatrix},
\begin{bmatrix}
0 & 1 \\
1 & 0 
\end{bmatrix}
\right\}
\]

spans $W$.

All matrices in $W$ is symmetric.

\begin{proof}
For any $A \in W$, $A = \begin{bmatrix} a_{11} & a_{12} \\ a_{21} & a_{22} \end{bmatrix}$
such that $a_{12} = a_{21}$. Then

\begin{align*}
\begin{bmatrix}
a_{11} & a_{12} \\
a_{12} & a_{22}
\end{bmatrix}
&=
\begin{bmatrix}
a_{11} & 0 \\
0 & 0
\end{bmatrix}
+
\begin{bmatrix}
0 & a_{12} \\
a_{12} & 0
\end{bmatrix}
+
\begin{bmatrix}
0 & 0 \\
0 & a_{22}
\end{bmatrix} \\
&=
a_{11}
\begin{bmatrix}
1 & 0 \\
0 & 0
\end{bmatrix}
+
a_{12}
\begin{bmatrix}
0 & 1 \\
1 & 0 
\end{bmatrix}
+
a_{22}
\begin{bmatrix}
0 & 0 \\
0 & 1
\end{bmatrix}.
\end{align*}

Therefore, $A$ is a linear combination of $\begin{bmatrix} 1 & 0 \\ 0 & 0 \end{bmatrix}$,
$\begin{bmatrix} 0 & 1 \\ 1 & 0 \end{bmatrix}$, $\begin{bmatrix} 0 & 0 \\ 0 & 1 \end{bmatrix}$.
Thus, $W$ is spanned by $\begin{bmatrix} 1 & 0 \\ 0 & 0 \end{bmatrix}$,
$\begin{bmatrix} 0 & 1 \\ 1 & 0 \end{bmatrix}$, $\begin{bmatrix} 0 & 0 \\ 0 & 1 \end{bmatrix}$
or $W = span\left\{ \begin{bmatrix} 1 & 0 \\ 0 & 0 \end{bmatrix},
\begin{bmatrix} 0 & 1 \\ 1 & 0 \end{bmatrix}, \begin{bmatrix} 0 & 0 \\ 0 & 1 \end{bmatrix}
\right\}$.
\end{proof}

\end{example}

\section{Set operations on subspaces}

Let $U$ and $W$ be two subspaces of $V$.


\begin{enumerate}[label = (\arabic*)]
\item $U \cap W$ is a subspace of $V$.

\begin{proof}
Since $U$ and $W$ are subspaces of $V$, then $0 \in U$ and $0 \in W$ and therefore
$U \cap W$ is nonempty. First we show that $U \cap W$ is closed under ``+''. For any
$x, y \in U \cap W$, $x, y \in U$ and $x, y \in W$. Then $x + y \in U$ and
$x + y \in W$ by the closure property of the subspaces. Therefore, $U \cap W$ is
closed under ``+''.

Similarly, $U \cap W$ is closed under ``$\cdot$'' and therefore $U \cap W$ is a subspace.
\end{proof}

\item The union $U$ and $W$ is \textbf{not} a subspace in general.

Consider the example where 
$U = span \left\{ \begin{bmatrix} 1 \\ 0 \end{bmatrix} \right\} \subseteq \mathbb{R}^2$, and
$W = span \left\{ \begin{bmatrix} 0 \\ 1 \end{bmatrix} \right\} \subseteq \mathbb{R}^2$.
Therefore, $U \cup W = span(e_1) \cup span(e_2)$.

\vspace{0.2cm}

\textbf{Claim:} $U \cup W$ is \underline{not} closed under ``+''. This is because 
$e_1 \in U$, $e_2 \in W$ and therefore $e_1 + e_2 = \begin{bmatrix} 1 \\ 1 \end{bmatrix} \notin U \cup W$.

\item The (algebraic) sum of $U$ and $W$,

\[U + W = \{u + w : u \in U \text{ and } w \in W \}. \]

Example: When $U = span \left\{ \begin{bmatrix} 0 \\ 1 \end{bmatrix} \right\}$ and
$W = span \left\{ \begin{bmatrix} 0 \\ 1 \end{bmatrix} \right\}$. Then $U + W \in \mathbb{R}^2$.

\begin{proposition}
Let $U$ and $W$ be two subspaces of $V$. Then the (algebraic) sum of $U + W$ is a subspace, where
$U + W = \{u + w : u \in U, w \in W\}$.
\end{proposition}

\begin{proof}
First we show that $U + W$ is closed under ``+''. Let $x, y \in U + W$. Then there exists
$u \in U$ and $w \in W$ such that $x = u + w$. Also, there exists $u^{\prime} \in U$ and
$w^{\prime} \in W$ such that $y = u^{\prime} + w^{\prime}$. Then

\begin{align*}
x + y &= (u + w) + (u^{\prime} + w^{\prime} \\
&= (u + u^{\prime}) + (w + w^{\prime}).
\end{align*}

Now $U$ is a subspace, and $u, u^{\prime} \in U$, therefore $u + u^{\prime} \in U$ because
$U$ is a subspace and closed under vector addition. Likewise $w + w^{\prime} \in W$. Then
$x + y \in U + W$.

Now we will show that $U + W$ is closed under ``$\cdot$''. Let $x \in U + W$ and $\alpha$
be an arbitrary scalar. Then $x = u + w$ for some $u \in U$ and $w \in W$. Then
$\alpha \cdot x = \alpha (u + w) = \alpha u + \alpha w$ where $\alpha u \in U$ and
$\alpha w \in W$ and therefore $\alpha \cdot x \in U + W$.

Thus $U + W$ is a subspace.
\end{proof}

\begin{remark}
If $S_U$ is a spanning set of $U$ and $S_W$ is a spanning set of $W$, then $S_W \cup S_U$
is a spanning set of $U + W$.

Sketch of the proof: Let $S_U = \{u_1, \dots, u_p\}$ and $S_W = \{w_1, \ldots, w_r\}$.
Consider an arbitrary $x \in U + W$. Then $x = u + w$ for some $u \in U$ and $w \in W$.
Since $u \in U$ and $S_U$ spans $U$, there exists scalars $\alpha_1, \ldots, \alpha_p$
such that $u = \alpha_1 u_1 + \cdots + \alpha_p u_p$. Likewise, there exists
$\beta_1, \ldots, \beta_r$ such that $w = \beta_1 w_1 + \ldots + \beta_r w_r$. Then

\[
x = (\alpha_1 u_1 + \cdots + \alpha_p u_p) + (\beta_1 w_1 + \cdots + \beta_r w_r).
\]

So $x$ is a linear combination of $u_1, \ldots, u_p$ and $w_1, \ldots, w_r$ and so
$x \in span(S_U \cup S_W)$. Therefore $U + W \subseteq span(S_U \cup S_W)$. Similarly,
you can show that $span(S_U \cup S_W) \subseteq U + W$. Thus $U + W = span(S_U \cup S_W)$.
\end{remark}
\end{enumerate}

\section{Range and null space}

\begin{definition}
Let $f : V \rightarrow W$ be a linear function where $V$ and $W$ are vector spaces over
the same field.

\begin{itemize}

\item [] The \textbf{range} of $f$, denoted by $range(f)$, is

\[ range(f) = \{f(x) : x \in V\} \subseteq W \]

which is the set of all images of $x$ under $f$.

\item [] The \textbf{null space} of $F$, denoted by $Null(f)$ is

\[ Null(f) = \{x \in V : f(x) = 0\} \subseteq V. \]
\end{itemize}
\end{definition}

\begin{proposition}
Let $f: V \rightarrow W$ be a linear function. Then
\begin{enumerate}[label = (\arabic*)]
\item The range of $f$ is a subspace of $W$.
\item $Null(f)$ is a subspace of $V$.
\end{enumerate}
\end{proposition}

\begin{proof}
\begin{enumerate}[label = (\arabic*)]
\item We need to show that $range(f)$ is closed under ``+'' and ``$\cdot$''. We will only
show the former. Let $y, z \in range(f)$. Since $y \in range(f)$ then there exists a $x$
such that $y = f(x)$. Likewise, there exists $x^{\prime}$ in $V$ such that $z = f(x^{\prime})$.
Consider $y + z = f(x) + f(x^{\prime})$. Since $f$ is linear, by superposition property,
$f(x) + f(x^{\prime}) = f(x + x^{\prime})$. This shows $y + z = f(x + x^{\prime})$. Since
$x, x^{\prime} \in V$ then $x + x^{\prime} \in V$ and therefore $y + z \in range(f)$.

\item $Null(f)$ is closed under ``+'' and ``$\cdot$''. We will show the former. Let
$x, x^{\prime} \in Null(f)$. Then $f(x) = 0$ and $f(x^{\prime}) = 0$. Since $f$ is
linear function; $f(x + x^{\prime}) = f(x) + f(x^{\prime}) = 0 + 0 = 0$. Therefore,
$x + x^{\prime} \in Null(f)$ and so $Null(f)$ is closed under ``+''. We can also show that
$Null(f)$ is closed under ``$\cdot$''. Thus $Null(f)$ is a subspace.
\end{enumerate}
\end{proof}

\section{Matrix defined linear functions}

Let $A \in \mathbb{R}^{m \times n}$. Consider the function 
$f : \mathbb{R}^n \rightarrow \mathbb{R}^m$ given by $f(x) = A \cdot x$, $x \in \mathbb{R}^n$.
It's shown that $f$ is a linear function.
\begin{itemize}
\item Then the  $range(f) = range(A) = r(A) = \{ Ax : x \in \mathbb{R}^n \}$, 
which is a subspace of $\mathbb{R}^m$.

\item The $Null(f) = Null(A) = N(A) = \{ x \in \mathbb{R}^n : Ax = 0 \}$, which is a
subspace of $\mathbb{R}^n$.

\item Similarly we have $R(A^T)$ which is a subspace of $\mathbb{R}^n$.

\item Also, $N(A^T)$ is a subspace of $\mathbb{R}^m$.
\end{itemize}

The spaces $R(A)$, $N(A)$, $R(A^T)$, and $N(A^T)$ are the four \textbf{fundamental subspaces}
associated with $A$.

\section{Spanning Sets of $R(A)$}

Write the matrix $A$ in the column form:


\[ A = [A_{1} \quad A_{2} \quad \cdots \quad A_{n}]. \]


Therefore, for any $x \in \begin{bmatrix} x_1 \\ \vdots \\ x_n \end{bmatrix} \in \mathbb{R}^n$

\[ Ax = x_1 \cdot A_{\bullet 1} + x_2 \cdot A_{\bullet 2} + \cdots + x_n \cdot A_{\bullet n},
\]

which is a linear combination of $A_{\bullet 1} , \ldots, A_{\bullet n}$. Thus, $R(A)$ is
the set of all linear combinations of $A_{\bullet 1} , \ldots, A_{\bullet n}$. Therefore,
$R(A) = span\{A_{\bullet 1}, \ldots, A_{\bullet n}\}$. Likewise, $R(A^T)$ is spanned by the
columns of $A^T$.

\section{Linear Independence}

\begin{proposition}
Let $S = \{ v_1, \ldots, v_r, v_{r+1}\}$ span a subspace $U$. Suppose $v_{r+1}$ is a linear
combination of $v_1, \ldots, v_r$, then $U = span\{v_1, \ldots, v_r \}$.
\end{proposition}

\begin{proof}
Clearly, $U = span\{v_1, \ldots, v_r, v_{r+1} \}$.

\vspace{0.5cm}

\textbf{Claim 1:} $span\{v_1, \ldots, v_r\} \subseteq U$.

\vspace{0.5cm}

Consider an arbitrary vector $x \in span\{v_1, \ldots, v_r\}$. Therefore, $x$ is a linear
combination of $v_1, \ldots, v_r$ so there exist scalars $\alpha_1, \dots, \alpha_r$
such that

\begin{align*}
x &= \alpha_1 v_1 + \alpha_2 v_2 + \cdots + \alpha_r v_r \\
&= \alpha_1 v_1 + \alpha_2 v_2 + \cdots + \alpha_r v_r + 0 \cdot v_{r+1}.
\end{align*}

So $x \in U$.

\vspace{0.2cm}

\textbf{Claim: 2:} $U \subseteq span\{v_1, \ldots, v_r\}$.

\vspace{0.2cm}

Since $v_{r+1}$ is a linear combination of $v_1, \ldots, v_r$ then there exists scalars
$\beta_1, \ldots, \beta_r$ such that 
$v_{r+1} = \beta_1 v_1 + \beta_2 v_2 + \cdots \beta_r v_r$.

For any $x \in U$, $x$ is a linear combination of $v_1, v_2, \ldots, v_r, v_{r+1}$ so
there exists scalars $\alpha_1, \alpha_2, \ldots, \alpha_r, \alpha_{r+1}$ such that

\begin{align*}
x &= \alpha_1 v_1 + \alpha_2 v_2 + \cdots + \alpha_r v_r + \alpha_{r+1} v_{r+1} \\
&= \alpha_1 v_1 + \cdots + \alpha_r v_r + \alpha_{r+1} (\beta_1 v_1 + \cdots + \beta_r v_r) \\
&= (\alpha_1 + \alpha_{r+1} \cdot \beta_1) v_1 + \cdots + (\alpha_r + \alpha_{r+1} \cdot \beta_r) v_r.
\end{align*}

Therefore, $x \in span\{v_1, \ldots, v_r\}$.

Thus $U = span\{v_1, \ldots, v_r\}$.
\end{proof}

\begin{definition}
A set $\{v_1, \ldots, v_p\}$ in a vector space $V$ is \textbf{linearly dependent} if there
exist scalars $\alpha_1, \ldots, \alpha_p$, not all zero, such that

\[ \alpha_1 v_1 + \alpha_2 v_2 + \cdots + \alpha_p v_p = 0. \]

Otherwise, $\{v_1, \ldots, v_p\}$ is \textbf{linearly independent}. That is, if
$\alpha_1 v_1 + \cdots + \alpha_p v_p = 0$, then 
$\alpha_1 = \alpha_2 = \cdots = \alpha_p = 0$. ($\Leftarrow$ is trivial).
\end{definition}

\begin{example} $ $
\begin{enumerate}[label = (\arabic*)]
\item $\{ 0 \}$: linearly dependent because for any nonzero scalar $\alpha$, $\alpha \cdot 0 = 0$.

Let $v \neq 0$, then $\{ v \}$ is linearly independent.

\begin{proof}
Let $\alpha$ be a scalar such that $\alpha \cdot v = 0$. For sake of contradiction, assume
$\alpha \neq 0$, which implies that $\frac{1}{\alpha}$ exists. Then
$\displaystyle \frac{1}{\alpha} (\alpha \cdot v) = \frac{1}{\alpha} \cdot 0$, which implies
that $v = 0$. However this contradicts the assumption that $v \neq 0$. So $\alpha = 0$,
then $\{ v \}$ is linearly independent.
\end{proof}

\textbf{Summary:} $\{ v \}$ is linearly independent $\iff$ $v \neq 0$.


\item $\{ u, v \}$ is linearly independent if and only if one of $u$ and $v$ is a 
multiple of the other.

\begin{proof}
``if'': Suppose $u = \alpha \cdot v$ (WLOG) for some scalar $\alpha$. Then $u - \alpha v = 0$
if and only if $u + (-\alpha) v = 1 \cdot u + (-\alpha) \cdot v = 0$. Since not every $\alpha$
is 0, then $\{ u, v \}$ is linearly dependent. 

Not every $\alpha$ is 0, therefore $\{ u, v \}$ is linearly dependent.

``only if'': Suppose $\{ u, v \}$ is linearly independent. Then there exists scalars
$\alpha_1, \alpha_2$ not both zero such that 

\[ \alpha_1 u + \alpha_2 v = 0. \] 

WLOG, assume $\alpha_1 \neq 0$, which implies $\displaystyle \frac{1}{\alpha_1}$ exists. Then
multiplying the previous equation on both sides by $\displaystyle \frac{1}{\alpha_1}$ and 
rearranging we have

\[ u = -\frac{\alpha_2}{\alpha_2} v. \]

Therefore, $u$ is a multiple of $v$. Likewise, if $\alpha_2 \neq 0$, then $v$ is a
multiple of $u$.
\end{proof}

\end{enumerate}
\end{example}

\begin{lemma}
A finite set $S = \{ v_1, \ldots, v_p \}$ is linearly dependent if and only if one 
of the vectors in $S$ is a linear combination of the rest of the vectors in $S$.
\end{lemma}

\begin{proof}
``if'': WLOG, assume that $v_1$ is a linear combination of $v_2, \ldots, v_p$. Then 
there exists scalars such that $\alpha_2, \ldots, \alpha_p$ such that

\[ v_1 = \alpha_2 v_2 + \alpha_3 v_3 + \cdots + \alpha_p v_p, \] 

which implies that

\[ (1)v_1 + (-\alpha_2)v_2 + \cdots + (-alpha_p)v_p = 0. \]

Since the above linear combination is equal to 0, but the first scalar is 1, then
$S$ is linearly dependent.

``only if'': Suppose $S$ is linearly dependent. Then there exists scalars
$\alpha_1, \ldots, \alpha_p$, not all zero, such that

\[ \alpha_1 v_1 + \cdots + \alpha_p v_p = 0. \] 

WLOG, assume that $\alpha_1 \neq 0$, which implies $\displaystyle \frac{1}{\alpha}$
exists. Then multiplying the previous equation by $\displaystyle \frac{1}{\alpha}$
and rearranging we have

\[ v_1 = -\frac{\alpha_2}{\alpha_1} v2 - \cdots - \frac{\alpha_p}{\alpha_1}v_p. \]

Therefore, $v_1$ is a linear combination of $v_2, \ldots, v_p$.
\end{proof}


\begin{proposition}
A finite set $S = \{v_1, \ldots, v_p \}$ is linearly independent if and only if none
of the vectors in $S$ is a linear combination of the rest of the vectors in $S$.
\end{proposition}

\begin{proof}
Contrapositive of the preceding lemma.
\end{proof}

\section{Minimal Spanning Set}

\begin{definition}
A spanning set $S$ of a vector space $V$ (i.e., $V = span(S)$) is \textbf{minimal} if
none of the vectors in $S$ is a linear combination of the rest of the vectors in $S$.
\end{definition}

\noindent \textbf{Fact:} A spanning set $S$ is minimal if and only if it is linearly independent.

\begin{lemma}
Let $S = \{v_1, \ldots, v_p \}$ be a minimal spanning set for a vector space $V$. If we
remove any vector from $S$, then the resulting set does \textbf{not} span $V$.
\end{lemma}

\begin{proof}
WLOG, assume we remove $v_1$ from $S$ such that the resulting set is 
$S^{\prime} = \{ v_2, \ldots, v_p \}$.

\textbf{Claim:} $S^{\prime}$ does not span $V$.

From the given we know that $S = \{s_1, s_2, \ldots, s_p\}$ spans $V$. Therefore, every
$v_i \in S$ belongs to $V$, which implies $v_1 \in V$. On the other hand, $S$ is minimal
and therefore $v_1$ is not a linear combination of $v_2, \ldots, v_p$. Thus, 
$v_1 \notin span(S^{\prime})$ and so $S^{\prime}$ does not span $V$.

\end{proof}


\begin{remark} 
In the above proof we found a vector $v_1 \in V$ that is not in 
$span(S^{\prime})$ so $span(S^{\prime}) \neq V$.
\end{remark}


\section{Maximal linearly independent set}

\begin{definition}
A linearly independent set $S$ in a vector space $V$ is maximal if for any
$z \in V$, $S \cup \{z\}$ is linearly dependent.
\end{definition}

\begin{lemma}
Let $S = \{v_1, \ldots, v_p \}$ be a linearly independent set in $V$. Then for any
given $z \in V$, $z \in span(S)$ if and only if $S \cup \{ z \}$ is linearly
dependent.
\end{lemma}

\begin{proof}
``Only if'': Suppose $z \in span(S)$. Therefore, $z$ is a linear combination of
$v_1, \ldots, v_p$. Then $S \cup \{ z \} = \{v_1, \ldots, v_p, z\}$ where $z$
is a linear combination of $v_1, \ldots, v_p$. Thus $S \cup \{z\}$ is linearly
dependent.

``If'': Suppose $S \cup \{ x \}$ is linearly dependent, which means there
exists scalars $\alpha_1, \ldots, \alpha_p$ and $\beta$, not all zero, such
that

\[ \alpha_1 v_1 + \cdots + \alpha_p v_p + \beta z = 0. \]

\textbf{Claim:} The scalar $\beta \neq 0$.

For the sake of contradiction that $\beta = 0$, which implies $\beta \cdot z = 0$.
Then the above equation reduces to

\[ \alpha_1 v_1 + \cdots + \alpha_p v_p = 0. \]

Since $S = \{ v_1, \ldots, v_p \}$ is linearly independent, then
$\alpha_1 = \alpha_2 = \cdots = \alpha_p = 0$. (and $\beta = 0$). However,
this is a contradiction. Therefore, $\beta \neq 0$.

Now since $\beta \neq 0$ then we can multiply the equation

\[ \alpha_1 v_1 + \cdots + \alpha_p v_p + \beta z = 0 \]

through by $\beta$ and then rearranging we have

\[z = - \frac{\alpha_1}{\beta} v_1 - \cdots - \frac{\alpha_p}{\beta} v_p. \]

Therefore, $z \in span(S)$.
\end{proof}

\begin{theorem} $ $
\begin{enumerate}[label = (\arabic*)]
\item A maximal linearly independent set $S$ in $V$ spans $V$, i.e., $V = span(S)$.

\item A linearly independent set is maximal if and only if it is a minimal
spanning set.
\end{enumerate}
\end{theorem}

\begin{proof}
\begin{enumerate}[label = (\arabic*)]
\item Suppose $S$ is a \textbf{maximal} linearly independent set. Then for any
$z \in V$, $S \cup \{z\}$ is linearly dependent. By the preceding lemma,
$z \in span(S)$. Since $z \in span(S)$ for any $z \in V$, then $span(S) = V$ or
$S$ spans $V$.
\item ``only if'': Let $S$ be a maximal linear independent set. By 1), $S$ spans $V$.
Therefore, $S$ is linearly independent and so $S$ is a minimal spanning set.

``if'': Let $S$ be a minimal spanning set. Then $S$ is linearly independent and so
$S$ spans $V$. Thus, for any vector $z \in V$, $z \in span(S)$. By the preceding
lemma, $S \cup \{z\}$ is linearly dependent. Therefore, $S$ is a maximal linear
independent set.
\end{enumerate}
\end{proof}

\section{More on linearly independent sets}

\textbf{Fact:} Let $S = \{ v_1, \ldots, v_p \}$ be a linearly independent set. Then any
(non-empty) subset of $S$ is linearly independent.

\begin{proof}
WLOG, suppose $S^{\prime} = \{ v_1, \ldots, v_r \} \subseteq S$, where $r < p$. Now
we want to show that $S^{\prime}$ is linearly independent.

Let $\alpha_1, \ldots, \alpha_r$ be such that 
$\alpha_1 v_1 + \alpha_2 v_2 + \cdots + \alpha_r v_r = 0$. Then we can add in the 
$v_{r+1}$ to $v_p$ vectors to the previous linear combination by taking 
$\alpha_{r+1} = \cdots = \alpha_p = 0$,

\[ \alpha_1 v_1 + \alpha_2 v_2 + \cdots + \alpha_r v_r + 0 \cdot v_{r+1} + \cdots + 0 \cdots v_p = 0. \]

Now since $S$ is linearly independent then it must be the case that 
$\alpha_1 = 0, \alpha_2 = 0, \ldots, \alpha_r = 0$ and therefore $\{v_1, \ldots, v_r \}$ is
linearly independent.
\end{proof}

Let $A$ be an $m \times n$ matrix whose column form is

\[ A = [A_{\bullet 1} \quad A_{\bullet 2} \quad \cdots \quad A_{\bullet n} ] \]

where each column $A_{\bullet i}$ is in $\mathbb{R}^m$ or $\mathbb{C}^m$. If

\[ (A \cdot x = 0 \iff x = 0) \iff N(A) = \{ 0 \} \]

then the columsn of $A$ are linearly independent.

\begin{proposition}
Let $\{ v_1, v_2, \ldots, v_r \}$ be a linearly independent set in $\mathbb{R}^n$ and
$P \in \mathbb{R}^{n \times n}$ be invertible. Then
$\{ P \cdot v_1, P \cdot v_2, \ldots, P \cdot v_r \}$ is linearly independent.
\end{proposition} 

\begin{proof}
Let $\alpha_1, \ldots, \alpha_r$ be scalars in $\mathbb{R}$ such that

\[ \alpha_1 P \cdot v_1 + \alpha2 \cdot P \cdot v_2 + \cdots + \alpha_r P \cdot v_r = 0. \]

Then we can rearrange each of the terms and remove $P$ from the left so that we have

\[ P [\alpha_1 v_1 + \alpha_2 v_2 + \cdots + \alpha_r v_r ] = 0 \]

and since $P$ is invertible we can left multiply both sides by $P^{-1}$ to obtain

\[ \alpha_1 v_1 + \alpha_2 v_2 + \cdots + \alpha_r v_r  = P^{-1} \dot 0 = 0. \]

Therefore, $\{P v_1, \ldots, P v_r \}$ are linearly independent.

\end{proof}

Let $A$ be a matrix in $\mathbb{R}^{m \times n}$ or $\mathbb{C}^{m \times n}$. Then
the range of $A$, $R(A)$ is the \textbf{column space} of $A$, which is a subset of
$\mathbb{R}^m$ or $\mathbb{C}^m$.

We can write $A$ in column form and use row reduction to get in an equivalent form
$P \cdot U$

\begin{align*}
A &= [A_{\bullet 1} \quad A_{\bullet 2} \quad \cdots \quad A_{\bullet n} ] \\
&\vdots \\
&= P \cdot U
\end{align*} 

where $P$ is an $m \times m$ invertible matrix and $U$ is an $m \times n$ matrix
that is the echelon form of $A$,

\[ U = [U_{\bullet 1} \quad U_{\bullet 2} \quad \cdots \quad U_{\bullet r} \quad
U_{\bullet r+1} \quad \cdots \quad U_{\bullet n} ], \]

where $U_{bullet 1}, \ldots, U_{\bullet r}$ are the pivot columns and
$U_{\bullet r+1}, \ldots, U_{\bullet n}$ are the non-pivot columns.

\vspace{0.2cm}

\textbf{Facts:}

\begin{enumerate}[label = (\arabic*)]
\item The pivot columns of $U$ are linearly independent.
\item A non-pivot column of $U$ is a linear combination of pivot columns of $U$.
\end{enumerate}

\vspace{0.2cm}

We can write $A$ as

\[ A = [A_{\bullet 1} \quad \cdots \quad A_{\bullet r} \quad
A_{\bullet r+1} \quad \cdots \quad A_{\bullet n} ], \]

and then $PA_{\bullet 1}, \ldots, PA_{\bullet r}$ are linearly because multiplying
a set of linearly independent vectors by an ivertible matrix preserves linear
independence.

Therefore the pivot columns of $A$ are linearly independent and each non-pivot column
of $A$ is a linear combination of the pivot columns of $A$. Therefore

\[ R(A) = span\{A_{\bullet 1}, \ldots, A_{\bullet n}\} = span\{A_{\bullet 1}, \ldots, A_{\bullet r} \} \]

where $span\{A_{\bullet 1}, \ldots, A_{\bullet r} \}$ is a minimal spanning set. So to
find a minimal spanning set, find the pivot columns of $A$.

\section{Basis and Dimension}

\noindent \textbf{Assumption:} $V$ is a vector space spannded by a finite set.

\begin{definition}
If a set $B \in V$ spans $V$ and is linearly independent then $B$ is a \textbf{basis} for $V$.
\end{definition}

$B$ is a basis $\iff$ $V = span(B)$ and $B$ is linearly independent.

\vspace{0.5cm}

\noindent \textbf{Fact:} 

\begin{align*}
B \text{ is a basis } &\iff B \text{ is a minimal spanning set } \\
&\iff B \text{ is maximinal linearly independent } 
\end{align*}

\begin{example}
\begin{enumerate}[label = (\arabic*)]
\item Let $V = \mathbb{R}^2$.
\begin{itemize}
\item [] Consider $B = \{e_1, e_2 \} = \left\{ \begin{bmatrix} 1 \\ 0 \end{bmatrix},
\begin{bmatrix} 0 \\ 1 \end{bmatrix} \right\}.$ Clearly, $B$ spans $V$, and is
linearly independent. Therefore, $B$ is a basis (i.e., it is the \textbf{standard} basis).

\item Consider $B^{\prime} = \left\{ \begin{bmatrix} 1 \\ 0 \end{bmatrix}, 
\begin{bmatrix} 1 \\ 1 \end{bmatrix} \right\}$. In a previous example we have shown
that $B^{\prime}$ spans $\mathbb{R}^2$ and none of $\begin{bmatrix} 1 \\ 0 \end{bmatrix}$ and
$\begin{bmatrix} 1 \\ 1 \end{bmatrix}$ is a multiple of the other. Therefore, $B^{\prime}$ is
linearly independent and is a basis.
\end{itemize}

\item Let $V = \mathbb{R}^n$. Let $A \in \mathbb{R}^{n \times n}$ be an invertible matrix. Then
$B = \{A_{\bullet 1}, \ldots, A_{\bullet n} \}$ is a basis for $V$.

\begin{proof}
First we prove the spanning property. Let $y \in \mathbb{R}^n$ be arbitrary. Therefore, $y$
is a linear combination of the columns of $A$ and so there exists and $x \in \mathbb{R}^n$
such that $y = A \cdot x$. Since $A$ is invertible then the equation $Ax = y$ has the unique
solution given by $A^{-1} \cdot y$. Therefore, $B$ spans $\mathbb{R}^n$.

Now we prove that $B$ is linearly independent. To show $B$ is linearly independent it suffices to
show that $N(A) = 0$. Since $A$ is invertible then $N(A) = \{ 0 \}$, which implies that
$B$ is linearly independent. Thus $B$ is a basis.
\end{proof}

\item Let $W$ be the subspace of all $2 \times 2$ upper triangular matrices in $\mathbb{R}^{2 \times 2}$.
Consider

\[ B =  \{ E^{11}, E^{12}, E^{22} \} = \left\{ \begin{bmatrix} 1 & 0 \\ 0 & 0 \end{bmatrix},
\begin{bmatrix} 0 & 1 \\ 0 & 0 \end{bmatrix}, \begin{bmatrix} 0 & 0 \\ 0 & 1 \end{bmatrix} \right\}. \]

We claim that $B$ is a basis for $W$.

\begin{proof}
We first prove the spanning property. We can write every $A \in W$ as

\[ A =
\begin{bmatrix}
a_{11} & a_{12} \\
0 & a_{22}
\end{bmatrix}
=
a_{11} \cdot E^{11} + a_{12} \cdot E^{12} + a_{22} \cdot E^{22}.
\]


Therefore, $B$ spans $W$. Now to show linear independence, suppose 
$\alpha_1, \alpha_2, \alpha_3 \in \mathbb{R}$ are such that
$\alpha_1 E^{11} + \alpha_2 E^{12} + \alpha_3 E^{22} = 0$. Then if

\[
\begin{bmatrix}
\alpha_1 & \alpha_2 \\
0 & \alpha_2
\end{bmatrix}
= 
\begin{bmatrix}
0 & 0 \\
0 & 0
\end{bmatrix}
\]
we must $\alpha_1 = \alpha_2 = \alpha_3 = 0$. Therefore, $B$ is linearly independent and
so $B$ is a basis.

\end{proof}
\end{enumerate}
\end{example}

\begin{lemma}
Let $B = \{v_1, \ldots, v_n\}$ be a basis for a vector space $V$. Then for any $x \in V$,
there exist \textbf{unique} scalars $\alpha_1, \ldots, \alpha_n$ such that

\[x = \alpha_1 v_1 + \cdots + \alpha_n v_n. \]
\end{lemma}

\begin{proof}
Let $x$ be an arbitrary vector in $V$. Since $B$ is a basis then $B$ spans $V$ and so
every vector in $V$ is a linear combination of $v_1, \ldots, v_n$. Then there exist
scalars $\alpha_1, \ldots, \alpha_n$ such that

\[ x = \alpha_1 v_1 + \cdots + \alpha_n v_n. \]

To show uniqueness of $\alpha_1, \ldots, \alpha_n$ let $\beta_1, \dots, \beta_n$ be scalars
such that

\[ x = \beta_1 v_1 + \cdots + \beta_n v_n. \]

Now if we subtract $x = \alpha_1 v_1 + \cdots + \alpha_n v_n$ and the previous equation we
get

\[ 0 = (\alpha_1 - \beta_1) v_1 + \cdots + (\alpha_n - \beta_n) v_n. \]

Since $B$ is linearly independent then it must be the case that
$\alpha_1 - \beta_1 = 0, \alpha_2 - \beta_2 = 0, \ldots, \alpha_n - \beta_n = 0$. Therefore,
$\beta_1 = \alpha_1, \beta_2 = \alpha_2, \ldots, \beta_n = \alpha_n$. This shows the 
uniqueness of the $\alpha_i$'s. 
\end{proof}


\begin{definition}
Let $B = \{v_1, \ldots, v_n \}$ be a basis for $V$. Then for each $x \in V$, $(\alpha_1, \dots, \alpha_n)$
is the coordinate of $x$ with respect to the basis $B$ denoted by $[x]_{\beta}$. Therefore

\[ [x]_{\beta} = \begin{bmatrix} \alpha_1 \\ \vdots \\ \alpha_n \end{bmatrix} \in \mathbb{R}^n. \]
\end{definition}

\begin{definition}
The \textbf{dimension} of a vector space $V$ is the number of vectors in a basis of $V$,
and it is denoted by $\dim V$.
\end{definition}

\begin{example}
\begin{enumerate}[label = (\arabic*)]
\item $\dim \mathbb{R}^n = n$.
\item $\dim \{ 0 \} = 0$.
\end{enumerate}
\end{example}

\begin{lemma}
Let $A$ be an $m \times n$ matrix. Suppose $n > m$. Then the columns of $A$ are linearly dependent.
\end{lemma}

\begin{theorem}
Let $B_1$ and $B_2$ be two bases for a vector space $V$. Then $\#(B_1) = \#(B_2)$, where $\#(B_i)$
is the number of vectors in $B_i$ for $i = 1,2$.
\end{theorem}

\begin{proof}
Let $B_1 = \{ v_1, \ldots, v_p \}$ and $B_2 = \{u_1, \ldots, u_r \}$ be two bases for $V$. Since
$B_1$ is a basis for $V$ then $B_1$ spans $V$. Also, each $u_i \in B_2$ is in $V$ and therefore each
$u_i$ is a linear combination of the vectors in $B_1$, $v_1, \ldots, v_p$, i.e.,

\[ u_i = \alpha_{i1} v_1 + \alpha_{i2} v_2 + \cdots + \alpha_{ip} v_p \]

for $i = 1, \ldots, r$.

Therefore, we can write $B_2$ as

\[ B_2 = [u_1 \quad \cdots \quad u_r ] = [v_1 \quad \cdots \quad v_p ] \cdot
\begin{bmatrix}
\alpha_{11} & \alpha_{21} & \cdots & \alpha_{r1} \\
\alpha_{12} & \alpha_{22} & \cdots & \alpha_{r2} \\
\vdots & \vdots & & \vdots \\
\alpha_{1p} & \alpha_{2p} & & \alpha_{rp}
\end{bmatrix}_{p \times r}
=
B_1 \cdot A
\]

where $A$ is a $p \times r$ matrix.

\vspace{0.5cm}

\textbf{Claim:} $r \leq p$.

\vspace{0.5cm}


Suppose otherwise; i.e., $r > p$. Then by the preceding lemma, the columns of $A$ are linearly
dependent. Then there exists a vector $w \neq 0$ such that $Aw = 0$. If we let multiply the
above equation by $w$ then we have

\[ [u_1 \quad \cdots \quad u_r] \cdot w = [v_1 \quad \cdots \quad v_p] \cdot A \cdot w = 0.\]

Now let $w = \begin{bmatrix} w_1 \\ \vdots \\ w_r \end{bmatrix} \neq 0$. Then

\[ w_1 u_1 + w_2 u_2 + \cdots + w_r u_r = 0, \]

which implies that $\{u_1, \ldots, u_r \}$ is linearly dependent. However, this contradicts
our assumption that $B_2 = \{u_1, \ldots, u_r \}$ is a basis. Thus it must be the case
that 

\[r \leq p \iff \#(B_2) \leq \#(B_1) .\]

By interchanging the roles of $B_1$ and $B_2$ we obtain $p \leq r \iff \#(B_1) \leq \#(B_2)$.
Thus, $\#(B_1) = \#(B_2)$.
\end{proof}

\begin{proposition}
Let $V$ be an n-dimensional vector space, i.e., $dim V = n$. Then
\begin{enumerate}[label = (\arabic*)]
\item A maximal linearly independent set in $V$ has $n$ vectors.
\item A minimal spanning set in $V$ has $n$ vectors.
\item A spanning set for $V$ has at least $n$ vectors.
\item A linearly independent set in $V$ has at most $n$ vectors.
\end{enumerate}
\end{proposition}

\begin{proof}
(1) and (2): Recall that a maximal linearly independent set and a minimal spanning set
are bases for $V$. Since $\dim V = n$ then each basis has exactly $n$ vectors. Thus,
a maximal linearly independent or a minimal spanning set contains $n$ vectors.

(3) Let $S$ be a spanning set for $V$. If $S$ is minimal, then by 2) $\#(S) = n$. If $S$
is not minimal, then it can be reduced to a minimal spanning set $S^{\prime}$, where
$n = \# (S^{\prime}) < \# (S)$. Thus, $S$ has at least $n$ vectors.
\end{proof}

\begin{proposition}
Let $V$ be an n-dimensional vector space, and $W$ is a subspace of $V$.
\begin{enumerate}[label = (\arabic*)]
\item $\dim W \leq \dim V$
\item If $\dim W = \dim V$, then $W = V$.
\end{enumerate}
\end{proposition}

\begin{proof}
\begin{enumerate}[label = (\arabic*)]
\item Let $B$ be a basis for the subspace $W$. Then $B$ is linearly independent and
$B \subseteq W \subseteq V$. Then

\[\dim W = \#(B) \leq n = \dim V. \]

Therefore, $\dim W \leq \dim V$.

\item Let $\dim W = \dim V$. Suppose for the sake of contradiction that $W \neq V$.
Since $W \subseteq V$ and $W \neq V$ then there exists a vector $z \in V$ but
$z \notin W$. Let $B$ be a basis for $W$, i.e., $W = span(B)$. Since $z \notin W$,
then $z \notin span(B)$. Since $B$ is a basis then $B$ is linearly independent and
so $B \cup \{z\}$ is linearly independent. Therefore $\#(B \cup \{ z \} = \#(B) + 1$.
Since $\dim V = \dim W = n$ and therefore $\#(B) = n$. But $\#(B \cup \{ z \}) = n + 1$
and therefore $B \cup \{ z \}$ is a linear independent set in $V$ with $n + 1$ vectors.
However by a previous proposition, this is a contradiction. Therefore, our initial
assumption that $W \neq V$ must be false. Thus, $W = V$.
\end{enumerate}
\end{proof}

\begin{theorem}
Let $U$ and $W$ be two subspaces of a vector space $V$ (of finite dimension). Then

\[ \dim(U + W) = \dim(U) + \dim(W) - \dim (U \cap W).\]

\end{theorem}

\begin{proof}
Not given in class.
\end{proof}

\begin{example}
Let $V = \mathbb{R}^3$ and let $U = span\{e_1\}$ and $W = span\{ e_1, e_2 \}$. Since
$U$ is a subspace of $W$ then $U \subseteq V$ and $U + W = W$ so $U \cap W = U$.
Therefore, $\dim U = 1$ and $\dim W = 2$. Then $\dim (U + W) = \dim(W) = 2$. On the
right we have $\dim U + \dim W - \dim(U \cap W) = 1 + 2 = 2$.
\end{example}

\section{Dimension of four fundamental subspaces associated with an $m \times n$ matrix}

Without loss of generality let $A \in \mathbb{R}^{m \times n}$. Then

\begin{itemize}
\item $R(A)$: range or the column space of $A$ is a subspace of $\mathbb{R}^m$.
\item $N(A)$: null space of $A$ is a subspace of $\mathbb{R}^n$.
\item $R(A^T)$: range or column space of $A^T$ is a subspace of $\mathbb{R}^n$.
\item $N(A^T)$: null space of $A^T$ is a subspace of $\mathbb{R}^m$.
\end{itemize}

\begin{definition}
The dimension of $R(A)$ is called the \textbf{rank} of $A$, i.e., $\dim[R(A)] = rank(A)$.

The dimension of the null space $N(A)$ is the \textbf{nullity} of $A$.
\end{definition}

\begin{proposition}
$rank(A) = 0$ if and only if $A = 0$, i.e., $A$ is the zero matrix.
\end{proposition}

\begin{proof}
If $A = 0$ then $R(A) = \{ 0 \}$, which implies $\dim[R(A)] = 0$. If $rank(A) = 0$, then
$\dim[R(A)] = 0$, which implies $R(A) = \{ 0 \}$ and so $A = 0$.
\end{proof}

\noindent Let $A$ be an $m \times n$ real matrix.


\begin{enumerate}[label = (\arabic*)]
\item The pivot columns of $A$ form a basis for $R(A)$.
\item Let $r$ be the number of pivots in $A$. Then

\[\dim[R(A)] = r = rank(A). \]

\item $\dim[N(A)] = \text{ the number of non-pivot columns of } A = n - r$.
\end{enumerate}

\begin{theorem}
(Rank-nullity theorem) Let $A$ be an $m \times n$ matrix. Then $rank(A) + \dim[N(A)] = n$.
\end{theorem}


\noindent \textbf{Fact:}
\begin{itemize}
\item $rank(A) = \dim[R(A^T)] = \text{ number of pivots in } A = r$.
\item $rank(A^T) + \dim[N(A^T)] = m$.
\end{itemize}

\section{Rank of a matrix product}

\begin{lemma}
(Special Case) Let $A: m \times n$ matrix, and $P: m \times m$ invertible. Then $rank(P \cdot A) = rank(A)$.
\end{lemma}

\begin{proof}
Need to show: $dim(R(P \cdot A)) = dim(R(A))$.

Let $A_{\bullet 1}$, $A_{\bullet 2}$, $\ldots$, $A_{\bullet r}$ be all pivot columns of $A$ such that
$ \{A_{\bullet 1}, A_{\bullet 2}, \ldots, A_{\bullet r} \} $ is a basis for $R(A)$.

\textbf{Claim}: $\{PA_{\bullet 1}, PA_{\bullet 2}, \dots, PA_{\bullet r} \}$ is a basis for $R(P \cdot A)$.

\begin{itemize}
\item Spanning Property: Since $\{ A_{\bullet 1}, A_{\bullet 2}, \ldots, A_{\bullet r} \}$ is a basis
for $R(A)$ then $\{ A_{\bullet 1}, A_{\bullet 2}, \ldots, A_{\bullet r} \}$ spans $R(A)$. Therefore,
$\{ PA_{\bullet 1}, PA_{\bullet 2}, \ldots, PA_{\bullet r} \}$.

\item Linear Independence: Let $\alpha_1, \alpha_2, \ldots, \alpha_r$ be such that
$\alpha_1 PA_{\bullet 1} + \cdots + \alpha_r PA_{\bullet r} = 0$. Then
$P (\alpha_1 A_{\bullet 1} + \cdots + \alpha_r A_{\bullet r}) = 0$. Since $P$ is invertible,
we can right multiply both sides of the previous equation by $P^{-1}$ and then we have
$\alpha A_{\bullet 1} + \cdots + \alpha_r A_{\bullet r} = 0$. Since
$\{A_{\bullet 1}, \ldots, A_{\bullet r} \}$ is a basis, then the set is linear
independent and so $\alpha_1 = 0, \alpha_2 = 0, \ldots, \alpha_r = 0$. Thus,
$\{ PA_{\bullet 1}, \ldots, PA_{\bullet r} \}$ is linearly independent and so
$\{ PA_{\bullet 1}, \ldots, PA_{\bullet r} \}$ is a basis for $R(PA)$. Now since
$dim(R(A)) = r = dim(R(PA))$ then $rank(A) = rank(P\cdot A)$.


\end{itemize}
\end{proof}

Now we will prove a more general theorem, but first we need some remarks and 
a preliminary lemma. The statement of the general theorem is: Let $A$ be an $m \times n$
matrix and $B$ be a $n \times p$ matrix. Then $rank(A \cdot B) = rank(B) - 
dim(N(A) \cap R(B))$.

\begin{remark} $ $
\begin{enumerate}[label = (\arabic*)]
\item Since $A$ is $m \times n$ then $N(A)$ is a subspace of $\mathbb{R}^n$. Since $B$ is
$n \times p$ then $R(B)$ is a subspace of $\mathbb{R}^n$. Therefore, $N(A) \cap R(B)$ is
a subspace of $\mathbb{R}^n$.

\item For any $z \in (N(A) \cap R(B))$, $z \in N(A)$ so that $Az = 0$ and $z \in R(B)$ so that
$z = B\cdot w$ for some $w$. Therefore, $Az = ABw = 0$ and so any $z \in N(A) \cap R(B)$ does
not ``contribute'' to $dim(R(A \cdot B))$.
\end{enumerate}
\end{remark}

\begin{lemma}
Let $V$ be an n-dimensional vector space and $W$ be a r-dimensional space of $V$ with $r < n$.
Suppose $\{v_1, \ldots, v_r \}$ is a basis for $W$. Then there exists 
$v_{r +1}, v_{r + 2}, \ldots, v_n \in V$ such that
$\{v_1, \ldots, v_r, v_{r + 1}, \ldots, v_n \}$ is a basis for $V$.
\end{lemma}

\begin{proof}
Need to show $\{v_1, \ldots, v_r, v_{r + 1}, \ldots, v_n \}$ is maximally linearly independent,
which implies that $\{v_1, \ldots, v_r, v_{r + 1}, \ldots, v_n \}$ is a basis.
\end{proof}

\begin{theorem}
Let $A$ be an $m \times n$ matrix and $B$ be a $n \times p$ matrix. Then $rank(A \cdot B) = rank(B) - 
dim(N(A) \cap R(B))$.
\end{theorem}

\begin{proof}
Since $N(A) \cap R(B)$ is a subspace of $\mathbb{R}^n$. Let $\{v_1, \ldots, v_r \}$ be a basis for
$N(A) \cap R(B)$. Then $dim(N(A) \cap R(B)) = r$.

\vspace{0.1cm}

Now consider 2 cases:
\begin{enumerate}[label = (\arabic*)]
\item $r = n$. We know $N(A) \cap R(B)$ is a subspace of $N(A)$ and a subspace of $R(B)$.
Since $r = n$, then $n = dim(N(A) \cap R(B)) \leq dim(R(B)) \leq n$, where the last
inequality follows because $R(B)$ is a subspace of $\mathbb{R}^n$. Therefore,
$dim(R(B)) = n$ and so $rank(B) = n$. Also, since
$r = n$, then $n = dim(N(A) \cap R(B)) \leq dim(N(A)) \leq n$. The last inequality
holds again because $N(A)$ is a subspace of $\mathbb{R}^n$. Therefore, $dim(N(A)) = n$
and by the rank-nullity theorem we have

\begin{align*}
rank(A) &= n - dim(N(A)) \\
&= n - n \\
&= 0.
\end{align*}

Since $rank(A) = 0$, then $A = 0$. Now since $rank(B) = 0$ and $A = 0$ then $(A \cdot B) = 0$.
Thus

\[ rank(A \cdot B) = 0 = rank(B) - dim(N(A) \cap R(B)). \]


\item $r < n$. In this case, we use the preliminary lemma to obtain
$z_1, z_2, \ldots, z_t \in R(B)$ such that
$\{ v_1, \ldots, v_r, z_1, \ldots, z_t \}$ is a basis for $R(B)$. Therefore
$rank(B) = dim(R(B)) = r + t$. Then $\{ v_1, \ldots, v_r, z_1, \ldots, z_t \}$
spans $R(B)$ and so 
$\{ Av_1, \ldots, Av_r, Az_1, \ldots, Az_t \}$ spans $R(B)$.

Recall that $\{v_1, v_r\}$ is a basis for $N(A) \cap R(B)$. Therefore, for each
$v_i \in N(A)$, for all $i = 1, \ldots, r$, we have $Av_i = 0$ for all $i = 1, \ldots, r$.
Now we can drop the zero vectors from $\{ Av_1, \ldots, Av_r, Az_1, \ldots, Az_t \}$
so that we have $\{Az_1, \ldots, Az_2\}$, which also spans $R(A \cdot B)$.

\textbf{Claim:} $\{Az_1, \ldots, Az_t\}$ is linearly independent.

Let $\alpha_1, \ldots, \alpha_t$ be scalars such that

\begin{align*}
\alpha_1 A z_1 + \cdots + \alpha_t A z_t = A (\alpha_1z_1 + \cdots + \alpha_t z_t) = 0.
\end{align*}

Now define $y = (\alpha_1 z_1 + \cdots + \alpha_t z_t)$. From above we see that $y \in N(A)$
and $y \in R(B)$ and so $y \in N(A) \cap R(B)$. Since $\{v_1, \ldots, v_r\}$ is a basis
for $N(A) \cap R(B)$ and $y \in N(A) \cap R(B)$ then there exists scalars
$\beta_1, \ldots, \beta_r$ such that

\[y = \beta_1 v_1 + \cdots + \beta_r v_r \].

Now since $y = \alpha_1 z_1 + \cdots + \alpha_t v_t$ and 
$y = \beta_1 v_1 + \cdots + \beta_r v_r$ then 

\[ \alpha_1 z_1 + \cdots + \alpha_t v_t = \beta_1 v_1 + \cdots + \beta_r v_r \]

and then subtracting $\beta_1 v_1 + \cdots + \beta_r v_r$ from both sides we have

\[ \alpha_1 z_1 + \cdots + \alpha_t v_t + (-\beta_1) v_1 + \cdots + (-\beta_r) v_r = 0. \]

Since $\{ v_1, \ldots, v_r, z_1, \ldots, z_t \}$ is a basis for $R(B)$ then
$\{ v_1, \ldots, v_r, z_1, \ldots, z_t \}$ is linearly independent. Therefore
$\beta_1 = 0 = \cdots = \beta_r = 0$ and $-\alpha_1 = 0 = \cdots = -\alpha_t = 0$ so
$\alpha_1 = 0 = \cdots = \alpha_t = 0$. Thus, $\{Az_1, \ldots, Az_t\}$ is linearly
independent.

Then $\{Az_1, \ldots, Az_t\}$ is a basis for $R(A \cdot B)$ and so $dim(R(AB)) = t$ and
then $rank(AB) = t$. Therefore, $rank(B) = dim(R(B)) = r + t$ and $dim(N(A) \cap R(B)) = r$
and so

\[ rank(AB) = t = r + t - r = rank(B) - dim(N(A) \cap R(B)). \]
\end{enumerate}

\end{proof}

\begin{example} $ $
\begin{enumerate}[label = (\arabic*)]
\item Let $A: m \times n$ and $P: m \times n$ invertible matrix.


\item Rank of $A \cdot A^T$ and $A^T \cdot A$ where $A \in \mathbb{R}^{m \times n}$.

\begin{theorem}
$rank(A \cdot A^T) = rank(A)$ and $rank(A^T \cdot A) = rank(A)$.
\end{theorem}

\begin{proof}
Note that $rank(A^T \cdot A) = rank(A) - dim(N(A^T) \cap R(A))$.

\textbf{Claim:} $N(A^T) \cap R(A) = \{0\}$.

Let $z \in N(A^T) \cap R(A)$. Then $z \in N(A^T)$ and $z \in R(A)$. Therefore,
$A^Tz = 0$ and $z = Ax$ for some $x$. Then $A^T z = A^T Ax = 0$. Then left 
multiply by $x^T$,


\begin{align*}
&\Rightarrow x^T A^T A x = x^T \\
&\Rightarrow (A \cdot x)^T A x = 0 \in \mathbb{R} \\
&\Rightarrow z^Tz = 0 \in \mathbb{R} \\
&\Rightarrow z = 0.
\end{align*}

Therefore, $N(A^T) \cap R(A) = \{0\}$. Then $rank(A^T \cdot A) = rank(A)$.

To show $rank(A \cdot A^T) = rank(A)$, we see that

\begin{align*}
rank(A \cdot A^T) &= rank((A^T)^T A^T) \\
&= rank(A^T) \\
&= rank(A).
\end{align*}
\end{proof}

\end{enumerate}
\end{example}


\begin{theorem}
Let $A$ be an $m \times n$ matrix. Then

\begin{enumerate}[label = (\arabic*)]
\item $R(A^T \cdot A) = R(A^T)$ and $R(A \cdot A^T) = R(A)$.

\item $N(A^T \cdot A) = N(A)$ and $N(A \cdot A^T) = N(A^T)$.
\end{enumerate}
\end{theorem}

\begin{proof} $ $
\begin{enumerate}[label = (\arabic*)]
\item Prove $R(A^T \cdot A) = R(A)$ only. Note that $R(A \cdot B)$ is a subspace
of $R(A)$ and therefore $R(A^T \cdot A)$ is a subspace of $R(A^T)$. Then it
suffices to show that $dim(R(A^T \cdot A)) = dim(R(A^T)) \iff rank(A^T \cdot A) = rank(A^T)$.
Since $rank(A^T \cdot A) = rank(A^T)$ and so $R(A^T \cdot A) = R(A^T)$.

\item Show $N(A^T \cdot A) = N(A)$ only.

Clearly, $N(A)$ is a subspace of $N(A^T \cdot A)$. This is because $x \in N(A) \iff Ax = 0
\Rightarrow A^T Ax = 0 \Rightarrow N(A^T \cdot A)$.

Is suffices to show that

\[ dim(N(A)) = dim(N(A^T \cdot A)) \].

Recall from the rank-nullity theorem that $rank(A) + dim(N(A)) = n$ and so
$dim(N(A)) = n - rank(A)$. Likewise, since $A^T A$ is an $n \times n$ matrix,
then $dim(A^T \cdot A) = n - rank(A^T \cdot A)$. Since 
$rank(A^T \cdot A) = rank(A)$ and $dim(N(A)) = dim(N(A^T \cdot A))$. Therefore,
$N(A^T \cdot A) = N(A)$.
\end{enumerate}
\end{proof}

\begin{remark} The following implications of the above theorems.

Let $A \in \mathbb{R}^{m \times n}$ and $b \in \mathbb{R}^m$.

\begin{enumerate}[label = (\arabic*)]
\item Consider the linear equation

\[ Ax = b. \]

In general, this equation may \textbf{not} have a solution.

\item Consider the \textbf{normal} equation

\[A^T \cdot A x = A^T b. \]

\textbf{Fact:} $A^T \cdot A x = A^T b$ always has a solution (or is always
\textbf{consistent}).

\begin{proof}
Since $R(A^T A) = R(A^T)$ and $A^T b \in R(A^T)$ then $A^Tb \in R(A^T A)$. Then
there exists a $p \in \mathbb{R}^n$ such that $A^T A p = A^T b$. Therefore,
$p$ is a solution to $A^T \cdot A \cdot x = A^T \cdot b$.
\end{proof}


\item If $Ax = b$ has a solution or is consistent, then $Ax = b$ and $A^T A x = A^T b$ have
the same solution set.

\item Suppose $Ax = b$ has a \textbf{unique} solution. Then $A^T \cdot A$ is invertible.

\begin{proof}
Since $Ax = b$ has a unique solution. Therefore $N(A) = \{0\}$ because otherwise
there exists a $0 \neq z \in N(A)$. Then $Ap = b$ and since $A$ is linear then
$A(p + z) = Ap + Az = b = 0$. However, this is a contradiction. Therefore,
$N(A) = \{0\}$.

Since $N(A^T \cdot A) = N(A)$ and $N(A) = \{0\}$ then $N(A^T A) = \{0\}$. Thus,
$A^T A$ is invertible. Then the unique solution to $A^T A x = A^T b$ is

\[x_{*} = (A^T A)^{-1} \cdot A^T \cdot b,\]

which is the unique solution to $Ax = b$.
\end{proof}
\end{enumerate}
\end{remark}

\section{Application to the least squares problem}

Let $A \in \mathbb{R}^{m \times n}$ and $b \in \mathbb{R}^n$. The least-squares (LS)
problem is to minimize 

\begin{align*}
F(x) &= || Ax - b ||_{2}^{2} \\
&= (Ax - b)^T (Ax - b) \\
&\geq 0.
\end{align*}

An optimal solution $x_*$ to the LS problem satisfies:

\[ \nabla F(x_*) = 0 \]

where $\nabla F(x) = 2(A^T A x = A^T b)$. Since $x_*$ is such that
$A^T A x = A^T b$, i.e., $x_*$ is solution to normal equation. Therefore,
the LS problem always has a solution.

\vspace{0.2cm}

\noindent \textbf{Claim:} $x_*$ satisfying the normal equation is an optimal solution to the
LS problem.

\begin{proof}
Let $x = x_* + u$, where $u \in \mathbb{R}^n$ is arbitrary. Then

\begin{align*}
F(x) &= (Ax - b)^T (Ax - b) \\
&= (Ax_* - b + Au)^T (Ax_* - b + Au) \\
&= (Ax_* - b)^T (Ax_* - b) + 2(Ax_* - b)^T Au + (Au)^T(Au).
\end{align*}

\noindent Note that the first term on the RHS above is $F(x_*)$, the second term $2(Ax_* - b)^T Au$ 
goes to zero, and the final term is a sum of squares and so is $\geq 0$.

\vspace{0.2cm}

Note that

\begin{align*}
2[(Ax_* - b)^T A]u = 2[A^T (Ax_* - b)]^T \cdot u = 0
\end{align*}

because term in brackets is equal to $0$ because $x_*$ satisfies the normal equation.

\vspace{0.2cm}

\noindent Therefore, $F(x) \geq F(x_*)$, for all $u \in \mathbb{R}^n$ or any $x \in \mathbb{R}^n$.
Thus, $x_*$ is an optimal solution to the LS problem.
\end{proof}


\section{Coordinate of a vector}

\begin{definition}
Let $V$ be an n-dimensional vector space, and $B = \{ v_1, \ldots, v_n \}$ be
a basis for $V$.

For each $x \in V$, let

\[ [x]_B =
\begin{bmatrix}
\alpha_1 \\
\alpha_2 \\
\vdots \\
\alpha_n
\end{bmatrix}
\in \mathbb{R}^n \text{ or } \mathbb{C}^n
\]

where $\alpha_1, \ldots, \alpha_n$ satisfy 
$x = \alpha_1 v_1 + \alpha_2 v_2 + \cdots + \alpha_n v_n$. Then $[x]_B$ is the
\textbf{coordinate vector} for $x$.
\end{definition}

\noindent \textbf{Fact:}

\begin{enumerate}[label = (\arabic*)]
\item $x \neq 0$ if and only if $[x]_B \neq 0$.

\item $x = 0$ if and and only $[x]_B = 0$.

\item For each $x \in V$, $[x]_B$ is unique.
\end{enumerate}


\begin{remark}
Recall that a linear function $T:U \rightarrow V$ where $U, V$ are finite
dimensional vector spaces, satisfies

\begin{enumerate}[label = (\arabic*)]
\item $T(x + y) = T(x) + T(y)$, for all $x, y \in U$

\item $T(\alpha x) = \alpha \cdot T(x)$, for all $x, y \in U$ and scalar $\alpha$.
\end{enumerate}

Then for any $x, y \in U$ and scalars $\alpha, \beta$

\[ T(\alpha x + \beta y) = \alpha T(x) + \beta T(Y). \]

\end{remark}


\noindent \textbf{Generalization}

\[ T \left( \sum^{n}_{i = 1} \beta_i u_i \right) = \sum^n_{i = 1} \beta_i \cdot T(u_i)
\text{ for all } u_i \in V,i = 1, \ldots, n. \]

\noindent \textbf{Special Case:}

If $T: V \rightarrow V$ is a linear function (i.e., $U = V$), then $T$ is call a
\textbf{linear operator}.

\begin{example} Linear Operator

\begin{enumerate}[label = (\arabic*)]

\item $T : \mathbb{R}^n \rightarrow \mathbb{R}^n$ given by $T(x) = A \cdot x$, where
$A$ is an $n \times n$ real matrix.

\item General linear operators: e.g., identity operator $I : V \rightarrow V$ given by
$I(x) = x$ for all $x \in V$.

\item Some other operators are projection operators and reflection operators.

\end{enumerate}

\end{example}

\section{Coordinate Matrix}

Let linear function $T: U \rightarrow V$ where $U, V$ are finite dimensional vector spaces.
Let $B = \{u_1, \ldots, u_n\}$ be a basis for $U$ and $B^{\prime} = \{v_1, \ldots, v_m \}$
be a basis for $V$.


\vspace{0.2cm}

\noindent \textbf{Question:} How to characterize the relation between $[x]_B$ and 
$[T(x)]_{B^{\prime}}$? 

\vspace{0.2cm}

We can use the \textbf{coordinate matrix} of $T$ from $B$ to $B^{\prime}$.

\vspace{0.2cm}

For each $u_i \in B$, $T(u_i) \in V$, since
$T(u_i) = \alpha_{1i} v_1 + \alpha_{2i} v_2 + \cdots + \alpha_{mi} v_m$ then
for weights $\alpha_{1i}, \ldots, \alpha_{mi}$

\[ T(u_i) =
\begin{bmatrix}
v_1 & \ldots & v_m
\end{bmatrix}
\begin{bmatrix}
\alpha_{1i} \\
\alpha_{2i} \\
\vdots \\
\alpha_{mi}
\end{bmatrix}
\]

\noindent Define the following $m \times n$ matrix,

\begin{align*}
[T]_{BB^{\prime}} &= \left[ [T(u_1)]_{B^{\prime}}, [T(u_2)]_{B^{\prime}}, \cdots [T(u_n)]_{B^{\prime}} \right] 
&= 
\begin{bmatrix}
\alpha_{11} & \alpha_{12} & \cdots & \alpha_{1n} \\
\alpha_{21} & \alpha_{22} & \cdots & \alpha_{2n} \\
\vdots & \vdots & & \vdots \\
\alpha_{m1} & \alpha_{m2} & \cdots & \alpha_{mn}
\end{bmatrix}
\in \mathbb{R}^{m \times n}.
\end{align*} 

Let $[T]_{BB^{\prime}}$ denote this matrix, which is called the \textbf{coordinate matrix} of $T$ 
from $B$ to $B^{\prime}$.

\begin{proposition}
$[T(x)]_{B^{\prime}} = [T]_{BB^{\prime}} \cdot [x]_B$ for all $x \in U$.
\end{proposition}


\begin{proof}
Let $x$ be an arbitrary vector in $U$. Since $\{u_1, \ldots, u_n \}$ is a basis for $U$
then we can write $x$ as a linear combination of the vectors $B$

\[ x = \beta_1 u_1 + \cdots + \beta_n u_n. \]

Then the coordinate vector of $x$ with respect to $B$ is
\[ [x]_B =
\begin{bmatrix}
\beta_1 \\
\beta_2 \\
\vdots \\
\beta_n
\end{bmatrix}
\in \mathbb{R}^n.
\]


Then we have

\begin{align*}
T(x) &= T(\beta_1 u_1 + \cdots + \beta_n u_n) \\
&= \beta_1 T(u_1) + \cdots + \beta_n T(u_n) \\
&= \begin{bmatrix} T(u_1) & T(u_2) & \cdots & T(u_n) \end{bmatrix}
\begin{bmatrix}
\beta_1 \\
\beta_2 \\
\vdots \\
\beta_n
\end{bmatrix} \\
&= \begin{bmatrix} v_1 + \cdots + v_m \end{bmatrix} [T]_{BB^{\prime}} \cdot [x]_B.
\end{align*}

Therefore $[T(x)]_{B^{\prime}} = [T]_{BB^{\prime}} \cdot [x]_B$.
\end{proof}

\begin{remark}
For a linear operator $T: V \rightarrow V$, if $B$ is a basis for $V$, the coordinate
matrix of $T$ from $B$ to $B$ is $[T]_{BB}$, which can be written as
$([T]_{BB}) = [T]_B$. (Because same basis for domain and codomain.)

If $V$ is an n-dimensional vector space, then $[T]_B$ is an $n \times n$ matrix for
any basis $B$ for $V$.
\end{remark}

\section{Change of basis}

\noindent \textbf{Question} Let $T$ be a linear operator and $B$ and $B^{\prime}$ be two
bases for $V$. What is the relation between $[T]_B$ and $[T]_{B^{\prime}}$?

\subsection{Change of basis}

Let $B = \{ u_1, \ldots, u_n \}$ and $B^{\prime} = \{v_1, \ldots, v_n\}$ be two bases
for an n-dimensional vector space $V$. Then the coordinate matrix of the identity
operator $I: V \rightarrow V$ from $B$ to $B^{\prime}$ is $[I]_{BB^{\prime}}$.

Then by construction of $[I]_{BB^{\prime}}$, we have

\begin{align*}
[I]_{BB^{\prime}} &= \begin{bmatrix} [I(u_1)]_{B^{\prime}} & [I(u_2)]_{B^{\prime}} &
\cdots & [I(u_n)]_{B^{\prime}} \end{bmatrix} \\
&= \begin{bmatrix} [u_1]_{B^{\prime}} & [u_2]_{B^{\prime}} & \cdots & [u_n]_{B^{\prime}}
\end{bmatrix}_{n \times n} \\
&= P.
\end{align*} 

\begin{remark} $ $
\begin{enumerate}[label = (\arabic*)]
\item For any $x \in V$, $[x]_{B^{\prime}} = [I]_{BB^{\prime}} \cdot [x]_B = P \cdot
[x]_B.$

\item $P = [I]_{BB^{\prime}}$ is invertible.

\begin{proof}
Recall $[T(x)]_{B^{\prime}} = [T]_{BB^{\prime}} \cdot [x]_B.$ Let $T = I$ the identity
operator. Then $[x]_{B^{\prime}} = [I]_{BB^{\prime}} \cdot [x]_B$ for all $x \in V$.

Suppose $P$ was not invertible. Then there exists $b \neq 0$ in $\mathbb{R}^n$
such that $P \cdot b = 0$. Then there exists an $x \in V$ such that $[x]_B = b$. Since
$b \neq 0$ then $x \neq 0$. Then $P \cdot b = [I]_{BB^{\prime}} \cdot [x]_B = 
[x]_{B^{\prime}}$ (by 1). Then $[x]_{B^{\prime}} = 0$ and so $x = 0$. However this
contradicts our assumption that if $b \neq 0$ then $x \neq 0$. Thus $P$ is invertible.
\end{proof}
\end{enumerate}
\end{remark}

\begin{theorem}
Let $P = [I]_{BB^{\prime}}$, the change of basis matrix. Then

\begin{enumerate}[label = (\arabic*)]
\item $[T]_B = P^{-1} \cdot [T]_{B^{\prime}} \cdot P$,
\item $[T]_{B^{\prime}} = P \cdot [T]_B \cdot P^{-1}$.
\end{enumerate}
\end{theorem}

\begin{proof}
It suffices to show $(1)$, which is equivalent to $P \cdot [T]_B = [T]_{B^{\prime}} \cdot P$.
To show $P \cdot [T]_B = [T]_{B^{\prime}} \cdot P$,

\begin{align*}
P \cdot [T]_B &= P \begin{bmatrix} [T(u_1)]_B & \cdots & [T(u_n)]_B \end{bmatrix} \\
&= \begin{bmatrix} P[T(u_1)]_B & \cdots & P[T(u_n)]_B \end{bmatrix} \\
&= \begin{bmatrix} [T(u_1)]_{B^{\prime}} & \cdots & [T(u_n)]_{B^{\prime}} \end{bmatrix} \\
&= \begin{bmatrix} [T]_{B^{\prime}} \cdot [u_1]_{B^{\prime}} & \cdots & [T]_{B^{\prime}} \cdot
[u_n]_{B^{\prime}} \end{bmatrix} \quad (1) \\
&= [T]_{B^{\prime}} \begin{bmatrix} [u_1]_{B^{\prime}} & \cdots & [u_n]_{B^{\prime}}
\end{bmatrix} \\
&= [T]_{B^{\prime}} \cdot P. \\
\end{align*}

where (1) follows because $[T(x)]_{B^{\prime}} = [T]_{B^{\prime}} \cdot [x]_{B^{\prime}}$.

Therefore, $P \cdot [T]_B = [T]_{B^{\prime}}$.
\end{proof}

\section{Similarity}

\begin{definition}
Let $A$ and $B$ be two $n \times n$ matrices. If there exists an invertible matrix $P$ such
that $A = P^{-1} B P$ then $A$ is \textbf{similar} to $B$, and we write $A \simeq B$.
\end{definition}


\subsection{Properties of similar relations}

\begin{enumerate}[label =  (\arabic*)]
\item Any $n \times n$ matrix is similar to itself. (By letting $P = I$).
\item If $A$ is similar to $B$, then $B$ is similar to $A$.

Since $A$ is similar to $B$ then $A = P^{-1} B P$ and so $ B= PAP^{-1} = (P^{-1})^{-1} A P^{-1}$.

\item If $A \simeq B$ and $B \simeq C$ then $A \simeq C$.

\item If $A$ is similar to $B$ then $A^k$ is similar to $B^k$ for any $k = 1, 2, \ldots$.
\item If $A$ is similar to $B$ then $rank(A) = rank(B)$ and $trace(A) = trace(B)$.
Also, $det(A) = det(B)$ and $A$ and $B$ have the same eigenvalues.

\begin{proof}
To show $trace(A) = trace(B)$. We have

\begin{align*}
trace(A) &= trace(P^{-1} B P) \\
&= trace((B \cdot P) \cdot P^{-1}) \\
&= trace(B \cdot I) \\
&= trace(B).
\end{align*}
\end{proof}
\end{enumerate}

\begin{remark}
In the above proof we used the fact that $trace(C \cdot D) = trace(D \cdot C)$, if $C$ and
$D$ are square.
\end{remark}


\section{Inner products, norms, and orthogonality}

\subsection{Inner products on $\mathbb{R}^n$ and $\mathbb{C}^n$}

Let $x = \begin{bmatrix} x_1 \\ \vdots \\ x_n \end{bmatrix}$ and
$y = \begin{bmatrix} y_1 \\ \vdots \\ y_n \end{bmatrix}$ be two vectors in $\mathbb{C}^n$
or $\mathbb{R}^n$.

\noindent If $x, y \in \mathbb{R}^n$, then the (standard) inner product is

\[
x^T y = \begin{bmatrix} x_1 & x_2 & \cdots & x_n \end{bmatrix}
\begin{bmatrix} y_1 \\ y_2 \\ \vdots \\ y_n \end{bmatrix} =
x_1 y_1 + x_2 y_2 + \cdots + x_n y_n \in \mathbb{R}^n
\]

\noindent If $x, y \in \mathbb{C}^n$ then the (standard) inner product of $x$ and $y$ is

\[
x^* \cdot y = \begin{bmatrix} \overbar{x_1} & \overbar{x_2} & \cdots & \overbar{x_n} \end{bmatrix}
\begin{bmatrix} y_1 \\ y_2 \\ \vdots \\ y_n \end{bmatrix} = \overbar{x_1} \cdot y + \cdots +
\overbar{x_n} \cdot y_n \in \mathbb{C}^n
\]

\subsection{Properties of (standard) inner products on $\mathbb{R}^n$ or $\mathbb{C}^n$}

Let $x, y \in \mathbb{C}^n$. Then the following hold:

\begin{enumerate}[label = (\arabic*)]
\item $y^* \cdot x = \overbar{x^* \cdot y}$
(When $x, y \in \mathbb{R}^n$, then $x^T y = y^T x$)

\item For any $\alpha \in \mathbb{C}$, $x^* (\alpha \cdot y) = \alpha (x^* \cdot y)$, and
$(\alpha \cdot x)^* \cdot y = \overbar{\alpha}(x^* \cdot y)$.

\item For any $z \in \mathbb{C}^n$, $x^* \cdot (y + z) = x^* \cdot y + x^* \cdot z$.

\item $x^* x \geq 0$ and $x^* \cdot x = 0$ if and only $x = 0$. 
(Note that $x^* \cdot x \in \mathbb{R}$).
\end{enumerate}

\begin{proof}
\begin{enumerate}[label = (\arabic*)]
\item We have that $y^* \cdot x = \overbar{y_1} \cdot x_1 + \overbar{y_2} \cdot x_2 +
\cdots + \overbar{y_n} \cdot x_n$ and 

\begin{align*}
\overbar{x^* \cdot y} &= \overbar{\overbar{x_1} \cdot y_1 + \cdots + \overbar{x_n} \cdot y_n} \\
&= \overbar{\overbar{x_1} \cdot y_1} + \cdots + \overbar{\overbar{x_n} \cdot y_n} \\
&= \overbar{\overbar{x_1}} \cdot \overbar{y_1} + \cdots + \overbar{\overbar{x_n}} \cdot \overbar{y_n} \\
&= x_1 \cdot \overbar{y_1} + \cdots + x_n \overbar{y_n} \\
&= \overbar{y_1} \cdot x_1 + \cdots + \overbar{y_n} \cdot x_n.
\end{align*}

Therefore, $y^* \cdot x = x^* \cdot y$.

\item Let $\alpha \in \mathbb{C}$. Then $x^* \cdot (\alpha \cdot y) = \alpha (x* \cdot y)$ (by
definition of $x^* \cdot y$).

To show $(\alpha x)^* \cdot y = \overbar{\alpha} (x^* \cdot y)$ we have

\begin{align*}
(\alpha \cdot x)^* \cdot y &= \overbar{y^* \cdot (\alpha x)} \\
&= \overbar{\alpha \cdot (y^* \cdot x} \\
&= \overbar{\alpha} \cdot \overbar{y^* \cdot x} \\
&= \overbar{\alpha} \cdot \overbar{\overbar{x^* y}} \\
&= \overbar{\alpha} \cdot x^* y.
\end{align*}

\item skipped

\item Given $x = \begin{bmatrix} x_1 \\ \vdots \\ x_n \end{bmatrix} \in \mathbb{C}^n$,
$x^* \cdot x = \overbar{x_1} \cdot x_1 + \cdots + \overbar{x_n} \cdot x_n$.

Since each $x_i \cdot \overbar{x_i} \in \mathbb{R}$ and $x_i \cdot \overbar{x_i} \geq 0$ for
all $i = 1, \ldots, n$ then $x_1 \cdot \overbar{x_1} + \cdots + x_n \cdot \overbar{x_n} =
\cdot x \geq 0$.

Clearly, if $x = 0$, then $x^* \cdot x = 0$. Conversely, if $x^* \cdot x = 0$, then
since $x^* \cdot x = |x_1|^2 + |x_2|^2 + \cdots + |x_n|^2 = 0$ then $|x_i| = 0$ for
all $i = 1, 2, \ldots, n$. Therefore, $x_i = 0$ for all $i = 1, 2, \ldots, n$ and so
$x = 0$.
\end{enumerate}
\end{proof}

\begin{remark} $ $
Recall that if $z = a + bi \in \mathbb{C}$ where $a, b \in \mathbb{R}$ then
$z \cdot \overbar{z} = (a + bi) (a - bi) = a^2 + b^2 \in \mathbb{R}$ and
$|z|^2 = a^2 + b^2 \geq 0$ where $|z| = \sqrt{a^2 + b^2}$.
\end{remark}

\section{Norms on $\mathbb{R}^n$ and $\mathbb{C}^n$}

\begin{definition}
Let $x \in \mathbb{C}^n$. Then it's (induced) norm is

\[ ||x|| = \sqrt{x^* x} \]

If $x \in \mathbb{R}^n$, then $||x|| = \sqrt{x^T \cdot x}$.
\end{definition}

\subsection{Properties of the induced norms}

Let $x \in \mathbb{C}^n$.

\begin{enumerate}[label = (\arabic*)]
\item $||x|| \geq 0$ and $||x|| = 0 \iff x = 0$.
\item For any $\alpha \in \mathbb{C}$, $||\alpha \cdot x|| = |\alpha| \cdot ||x||$
where $|\alpha|$ is the modulus of $\alpha \in \mathbb{C}$ for 
$\alpha = a + bi \in \mathbb{C}$.

If $x \in \mathbb{R}$, then $|\alpha|$ is the absolute value of $\alpha$.

\item Cauchy-Schwartz Inequality.

\[ |x^* \cdot y| \leq ||x|| \cdot ||y||, \quad \forall x, y \in \mathbb{C}^n \]

\item Triangle Inequality for $|| \cdot ||$.

\[||x + y || \leq ||x|| + ||y||.\]

\end{enumerate}

\begin{proof}

\begin{enumerate}[label = (\arabic*)]
\item Since $x^* \cdot x \geq 0$ and $x^* \cdot x \in \mathbb{R}$ for all
$x \in \mathbb{C}^n$ then $||x|| = \sqrt{x^* \cdot x} \geq 0$. Further,

\begin{align*}
||x|| = 0 &\iff \sqrt{x^* \cdot x} - 0 \\
&\iff x^* \cdot x = 0 \\
&\iff x = 0.
\end{align*}

where the last $\iff$ follows from properties of the inner product.


\item Let $\alpha \in \mathbb{C}$. Then

\begin{align*}
||\alpha \cdot x ||^2 &= (\alpha \cdot x)^* \cdot (\alpha \cdot x) \\
&= \alpha (\alpha x)^* \cdot (x) \\
&= \alpha \cdot \overbar{\alpha} (x^* \cdot x) \\
&= |\alpha|^2 \cdot ||x||^2.
\end{align*}

Therefore, taking square roots we have $||\alpha \cdot x || = |\alpha| \cdot ||x||$.


\begin{remark}
Before proving the Cauchy-Schwartz inequality we need a couple of facts. Let $z = a + bi$
where $a, b \ in \mathbb{R}$.

\begin{enumerate}[label = (\roman*)]
\item $z \cdot \overbar{z} = (a + bi)(a - bi) = a^2 + b^2 = |z|^2$.
\item $z + \overbar{z} = (a + bi) + (a - bi) = 2a \leq 2 \cdot |z|$, which implies
that $z + \overbar{z} \leq 2|z|$.
\end{enumerate}
\end{remark}

\item Proof of Cauchy-Schwartz Inequality.

When $x = 0$, then $|x^* y| = 0$ and $||x|| \cdot ||y|| = 0$. Then the C-S inequality
holds.

Now consider the case when $x \neq 0$ and therefore $||x|| > 0$. 

Define $\alpha$ as

\[
\alpha = \frac{x^* \cdot y}{|| x ||^2} 
\]

for a given $y \in \mathbb{C}$. Note that since $|| x || \geq 0$ then $\alpha$ is well defined.
Then

\begin{align*}
x^* \cdot (\alpha x - y) &= \alpha \cdot x^* \cdot x - x^* \cdot y \\
&= \alpha \cdot || x ||^2 - x^* \cdot y \\
&= \frac{x^* \cdot y}{|| x ||^2} ||x||^2 - x^* \cdot y \\
&= 0.
\end{align*}

Note that $|| \alpha \cdot x - y ||^2 \geq 0$ and

\begin{align*}
||\alpha \cdot x - y||^2 &= (\alpha \cdot x - y)^* \cdot (\alpha x - y) \\
&= [(\alpha x)^* - y^*] \cdot (\alpha x - y) \\
&= [\overbar{\alpha} x^* - y^*] \cdot (\alpha x - y) \\
&= \overbar{\alpha} \cdot x^* \cdot (\alpha x - y) - y^* \cdot (\alpha x - y) \\
&= y^* \cdot y - \alpha y^* \cdot x
\end{align*}

where the last equality follows because we know from earlier that $x^* \cdot (\alpha x - y) = 0$.
Therefore

\[y^* \cdot y - \alpha \cdot y^* \cdot x = || \alpha x - y ||^2 \geq 0 \]

and so

\[||y||^2 \geq \alpha \cdot y^* \cdot x = \frac{x^* \cdot y}{||x||^2} y^* \cdot x.\]

Multiplying by $||x||^2$ on both sides the the previous inequality we have

\begin{align*}
||x||^2 \cdot ||y||^2 &\geq (x^* \cdot y)(y^* \cdot x) \\
&= (x^* \cdot y) (x^* \cdot y) \\
&= (x^* \cdot y) \cdot \overbar{x^* \cdot y} \\
&= ||z||^2 \\
&= |x^* y|^2.
\end{align*}

Therefore, $||x||^2 ||y||^2 \geq |x^* \cdot y|^2$ and taking square roots on both sides we have

\[ ||x|| \cdot ||y|| \geq |x^* \cdot y|. \]


\item Proof of Triangle Inequality

\begin{align*}
||x + y||^2 &= (x + y)^* (x + y) \\
&= (x^* + y^*) \cdot (x + y) \\
&= (x^* \cdot x) + (x^* \cdot y) + (y^* \cdot x) + (y^* \cdot y) \\
&= ||x||^2 + x^* \cdot y + x^* \cdot y + ||y||^2 \\
&= ||x||^2 + 2 \cdot Re(x^* \cdot y) + ||y||^2 \\
&\leq ||x||^2 + 2 \cdot |x^* \cdot y| + ||y||^2 \\
&\leq ||x||^2 + 2||x|| \cdot ||y|| + ||y||^2 \\
&\leq (||x|| + ||y||)^2.
\end{align*}

Therefore, $||x + y||^2 \leq (||x|| + ||y||)^2$ and taking the square root of both sides
we have

\[||x+y|| \leq ||x|| + ||y||.\]

\end{enumerate}

\end{proof}


\section{General Inner Product Space}

\begin{definition}
An \textbf{inner product} on a real or complex vector space $V$ is a function

\[\langle \cdot, \cdot \rangle : V \times V \rightarrow \mathbb{R} \text{ or } \mathbb{C} \]

that satisfies the following four properties:

\begin{enumerate}[label = (\arabic*)]

\item $\langle x, x \rangle \geq 0$ for $x \in V$, and $\langle x, x \rangle \iff x = 0$.
\item $\langle x, y+z \rangle = \langle x,y \rangle + \langle x,z \rangle$ for all $x, y, z \in V$.
\item $\langle x, \alpha \cdot y \rangle = \alpha \langle x, y \rangle$ for all $x,y \in V$ and scalar $\alpha$ in
$\mathbb{R}$ or $\mathbb{C}$.
\item $\langle y,x \rangle = \overbar{\langle x, y \rangle}$ for all $x, y \in V$.

\end{enumerate}

Then $V$, along with the inner product $\langle \cdot, \cdot \rangle$ is an \textbf{inner product space}. (IPS)
\end{definition}

\subsection{Implications of Inner Products}

\begin{enumerate}[label = (\roman*)]
\item It can be shown that

\[\langle x, \alpha_1 v_1 + \cdots + \alpha_p v_p \rangle = \alpha_1 \langle x, v_1 \rangle +
\alpha_2 \langle x, v_2 \rangle + \cdots + \alpha_p \langle x, v_p \rangle.
\]
\item The induced norm is
\[ ||x|| = \sqrt{\langle x, x \rangle}\]
for all $x \in V$.

It can be shown that:
\begin{itemize}
\item $||x|| \geq 0$, and $||x|| = 0 \iff x = 0$
\item $||\alpha x|| = |\alpha| \cdot ||x||$ for all $x \in V$ and scalar $\alpha$.
\item $|\langle x, y \rangle | \leq ||x|| \cdot ||y||$ (Cauchy Schwartz)
\item $||x + y|| \leq ||x|| + ||y||$ (Triangle Inequality)
\end{itemize}

\item $\langle x, 0 \rangle$ for all $x \in V$.
\begin{proof}
Note that
\begin{align*}
\langle x, 0 \rangle &= \langle x, y + (-y) \rangle \\
&= \langle x, y \rangle + \langle x, (-1) \cdot y \rangle \\
&= \langle x, y \rangle + (-1) \langle x, y \rangle \\
&= 0.
\end{align*}
\end{proof}

\end{enumerate}

\begin{example} Inner Products
\begin{enumerate}[label = (\arabic*)]
\item Let $V = \mathbb{R}$ and let $A \in \mathbb{R}^{n \times n}$ be invertible.
For any $x, y \in \mathbb{R}^n$
\[ \langle x, y \rangle_{A} = (Ax)^T(Ay) \]
Note that if $A = I$, the identity matrix, then the above is the standard inner product.

\textbf{Claim:} $\langle \cdot, \cdot \rangle_{A}$ is an inner product on $V$.
\begin{proof}
It's easy to show $\langle \cdot, \cdot \rangle_{A}$ satisfies conditions $2) - 4)$. Only show
condition $1)$.

Since $\langle x, x \rangle_A = (Ax)^T (Ax) = z^Tz$ where $z = Ax \in \mathbb{R}^n$ and
$z^Tz \geq 0$, then $\langle x, x \rangle_A \geq 0$. Further, if $x = 0$, then 
$\langle x, x \rangle_A = 0$. 

Conversely, if $\langle x, x \rangle_A = 0$, then $z^Tz = 0$, where $z = A \cdot x$, which
implies that $z = 0$ and so $Ax = 0$. Since $A$ is invertible, then $x = 0$, then
$\langle x, x \rangle_A = 0 \iff x = 0$. Therefore, $\langle \cdot , \cdot \rangle_A$ is
an inner product.
\end{proof}

\item Let $V = \mathbb{R}^{n \times n}$, the space of all real matrices of size $n \times n$.
For any $A, B \in V$, let

\[\langle A, B \rangle = trace(A^T \cdot B). \].

\textbf{Claim:} $\langle \cdot, \cdot \rangle$ is an inner product.
\begin{proof} Fill in.
\end{proof}

The induced norm is

\[ ||A||_F = \sqrt{ \langle A, A \rangle } = \sqrt{ trace(A^T \cdot A) }. \]

Here, $|| \cdot ||_F$ is the Frobenius norm for a matrix.

\end{enumerate}
\end{example}

\begin{definition}
Let $|| \cdot ||$ be a norm on $V$. Then a vector $x \in V$ is a \textbf{unit vector} if $|| x || = 1$.

\textbf{Fact:} A unit vector must be a non-zero vector.
\end{definition}

\subsection{Normalization of a Vector}

Given any non-zero vector $z \in V$,
\[ \frac{z}{||z||} \]
is a unit vector.

\begin{proof}
\begin{align*}
\left|\left| \frac{z}{||z||} \right|\right| &= \left|\left| \frac{1}{||z||} \cdot z \right|\right| \\
&= |\alpha| \cdot ||z|| \\
&= \alpha \cdot ||z|| \\
&= \frac{1}{||z||} \cdot ||z|| \\
&= 1
\end{align*}

where $\displaystyle \alpha = \frac{1}{||z||} > 0$. Therefore $\displaystyle \frac{z}{||z||}$ is a
unit vector.
\end{proof}

\section{Orthogonal Vectors}

\begin{definition}
Let $(V, \langle \cdot, \cdot \rangle )$ be a real/complex inner product space. Given $x, y \in V$,
$x$ is \textbf{orthogonal} to $y$ (i.e., $x \perp y$) if $\langle x, y \rangle = 0$.
\end{definition}

\begin{remark} Facts about Orthogonal Vectors

\begin{enumerate}[label = (\arabic*)]
\item If $x \perp y$, then $y \perp x$?

\textbf{Claim:} $\langle x, y \rangle = 0 \implies \langle y, x \rangle = 0$.
\begin{proof}
Since $\langle y, x \rangle = \overbar{\langle x, y \rangle} = \overbar{0} = 0$, then
$y \perp x$.
\end{proof}
\item Is the zero vector orthogonal to any vector in $V$?
Answer: Yes because $\langle x, 0 \rangle = 0$ for all $x \in V$. Thus, $0 \perp x$ for all
$x \in V$.
\end{enumerate}
\end{remark}

\begin{definition}
Let $S = \{v_1, \ldots, v_p \}$ be a finite set in an inner product space $V$. We say $S$ is
an \textbf{orthogonal set} if

\[
\langle v_i, v_j \rangle = 0 \quad \quad \forall i \neq j
\]

or 

\[
v_i \perp v_j \quad \quad \forall v_i, v_j \in S \text{ with } i \neq j.
\]

\noindent If in addition, each $v_i \in S$ is a unit vector, then $S$ is an \textbf{orthonormal} set (or
simply an O.N. set).

\end{definition}

\begin{remark}
Note that $||v_i|| = 1 \iff ||v_i||^2 = 1$. Therefore, $S$ is orthonormal if and only if:

\[
\langle v_i, v_j \rangle =
\begin{cases}
0, & i \neq j \\
1, & i = j
\end{cases}
= \delta_{ij}
\]

where $\delta_{ij}$ is the delta function.
\end{remark}

\begin{example} $ $
Let $V = \mathbb{R}^3$ and $B = \{e_1, e_2, e_3 \}$ be the standard basis for $\mathbb{R}^3$. Show that
$B$ is an O.N. set.

\begin{proof}
For any $i, j$ from $\{1, 2, 3 \}$, we have

\[
\langle e_i, e_j \rangle = e_i^T \cdot e_j = 
\begin{cases}
0, & i \neq j, \\
1, & i = j
\end{cases}.
\]

Therefore, $B$ is orthonormal.
\end{proof}
\end{example}


\section{Properties of Orthogonal Sets}

\begin{lemma}
An orthogonal set of non-zero vectors is linearly independent.
\end{lemma}

\begin{proof}
Let $\{u_1, \dots, u_p\}$ be an orthogonal set where each $u_i \neq 0$.

Suppose there are scalars $\alpha_1, \dots, \alpha_p$ such that

\[ \alpha_1 u_1 + \cdots + \alpha_p u_p = 0. \]

We want to show $\alpha_1 = 0, \alpha_2 = 0, \ldots, \alpha_p = 0$.

\vspace{0.2cm}

Note that for each $u_i$ $\langle u_i, 0 \rangle = 0$ and so

\[ \langle u_i, \alpha_1 u_1 + \cdots + \alpha_i u_i + \cdots + \alpha_p u_p \rangle = 0. \]

The left hand side is:

\[ \alpha_1 \langle u_i , u_1 \rangle + \cdot + \alpha_i \langle u_i, u_i \rangle + \cdots + \alpha_p 
\langle u_i, u_p \rangle. \]

Since $\{ u_1, \ldots, u_p \}$ is an orthogonal set then $\langle u_i, u_j \rangle = 0$ if $i \neq j$.
Therefore, the left hand side equals $\alpha_i \langle u_i, u_i \rangle = \alpha_i ||u_i||^2$. (Since
$u_i \neq 0$ then $\langle u_i, u_i \rangle > 0$). Since $u_i \neq 0$ then $||u_i||^2 > 0$ and
then since $\alpha_i \cdot ||u_i||^2 = 0$ then $\alpha_i = 0$.

Therefore, $\alpha_1 = 0, \alpha_2 = 0, \ldots, \alpha_p = 0$ and so 
$\{u_1, \ldots, u_p \}$ is linearly independent.
\end{proof}

\begin{remark}
In the above proof we can repeat the process for each $i$ to show that $\alpha_i = 0$.
\end{remark}

\begin{corollary} $ $
An O.N. set is linearly independent.
\end{corollary}
\begin{proof}
This is because each vector in an O.N. set is a unit vector and must be non-zero. Then
by the preceding lemma, an O.N. set is linearly independent.
\end{proof}

\section{Orthonormal Basis}

\begin{definition}
Let $B$ be a basis for a finite dimensional inner product space (IPS). If $B$ is
an orthonormal set, then $B$ is an \textbf{orthonormal basis}.
\end{definition}

\begin{corollary}
Let $V$ be an n-dimensional IPS. An O.N. set of $n$ vectors in $V$ is an O.N. basis
for $V$.
\end{corollary}

\begin{proof}
Let $B$ be an O.N. set of $n$ vectors. By the above corollary, $B$ is linearly
independent. If $V$ is of $\dim{n}$ and $B$ has $n$ vectors then $B$ is a maximally
linearly independent set in $V$. Therefore, $B$ is a basis for $V$ and so $B$ is
an orthonormal basis for $V$.
\end{proof}

\begin{proposition}
Let $B = \{u_i, \ldots, u_n\}$ be an O.N. basis for $V$. Given a vector $x \in V$,
$x$ can be written as

\[x = \langle u_1, x \rangle u_1 + \cdots + \langle u_n, x \rangle u_n \]

where $\langle u_i, x \rangle$ are the Fourier coefficients and writing $x$ this
way is a special case of the Fourier Expansion.

\begin{remark}
The coordinate vector of $x$ with respect to basis $B$ can be written as

\[ [x]_B = 
\begin{bmatrix}
\langle u_1, x \rangle \\
\langle u_2, x \rangle \\
\vdots \\
\langle u_n, x \rangle
\end{bmatrix}.
\]
\end{remark}
\end{proposition}

\begin{proof}
Since $B = \{u_1, \ldots, u_p \}$ is an O.N. basis for $V$ then for any given vector
$x \in V$, there exists unique scalars $\alpha_1, \ldots, \alpha_n$ such that

\[ x = \alpha_1 u_1 + \alpha_2 u_2 + \cdots + \alpha_n u_n. \]

For each $u_i$ (i.e., fix $i$),

\begin{align*}
\langle u_i, x \rangle &= \langle u_i, \alpha_1 u_1 + \cdots + \alpha_i u_i + \cdots
+ \alpha_n u_n \rangle \\
&= \alpha_1 \langle u_i, u_1 \rangle + \cdots + \alpha_i \langle u_i, u_i \rangle +
\cdots + \alpha_n \langle u_i, u_n \rangle.
\end{align*}

Since $B$ is O.N. then

\[
\langle u_i, u_j \rangle =
\begin{cases}
0, & \text{ if } i \neq j \\
1, & \text{ if } i = j
\end{cases}.
\]

Therefore, $\langle u_i, x \rangle = \alpha_i$ for all $i = 1, \ldots, n$. (Repeat the
above process for each fixed $i$).

\end{proof}

\section{Idea for Gram-Schmidt Procedure/Process}

\textbf{Goal:} Let $B = \{ v_1, \ldots, v_n \}$ be a basis for an n-dimensional IPS $V$.
Construct an O.N. basis from $B$.

\begin{remark}
In general, there could be multiple O.N. bases and the Gram-Schmidt process gives one
O.N. basis that depends on what basis you start from.
\end{remark}

We will use the following preliminary result below:

\begin{lemma}
Any subset of a linearly independent set is also linearly independent.
\end{lemma}

\noindent \textbf{Gram-Schmidt Process:} To find an orthonormal basis for an n-dimensional 
IPS $V$ from $B = \{ v_1, \ldots, v_n \}$, the Gram-Schmidt process uses the following process:

\begin{itemize}
\item [\textbf{Step 1}] Let $W_1 = span\{v_1\}$.

Clearly, $\{v_1\}$ spans $W_1$, and is linearly independent because it is a subset
of $B$. Since $\{ v_1 \}$ is a basis for $W_1$ then $\dim W_1 = 1$ and therefore
$W_1$ is a one-dimensional subspace of $V$.

\item [\textbf{Step 2}] Let $W_2 = span\{v_1, v_2 \}$.

Likewise, $\{v_1, v_2\}$ is a basis for $W_2$ and therefore $\dim W_2 = 2$ and so
$W_2$ is a two dimensional subspace of $V$. Also, $W_1$ is a one-dimensional
subspace of $W_2$ (i.e., $W_1 \subseteq W_2$).

\item [\textbf{Steps $3, \ldots, n$}] Similarly, we can define $W_2, W_3,
\ldots, W_k, \ldots, W_n$, i.e., $W_k = span\{v_1, v_2, \ldots, v_k\}$ is an
k-dimensional subspace of $V$. Finally, $W_n = span\{v_1, \ldots, v_n\} = V$.

\end{itemize}

Note that $W_1 \subseteq W_2 \subseteq W_3 \subseteq \cdots \subseteq W_k \subseteq \cdots
\subseteq W_n = V$.

\begin{remark}
The idea of the GS process is to construct on O.N. basis for $W_1$ and use that to
construct O.N. basis for $W_2$, $\ldots$ O.N. basis for $W_n = V$.
\end{remark}

\begin{definition} Consider the subspace $W_k$. Suppose $\{u_1, \ldots, u_k \}$ is
an O.N. basis for $W_k$. Consider an arbitrary vector $x \in V$ (not necessarily in
$W_k$). We call

\[ p = \langle u_1, x \rangle u_1 + \cdots + \langle u_k, x \rangle u_k \]

the \textbf{orthogonal projection} of $x$ onto $W_k$ denoted by $\vctproj[W_k]{x}$.

\end{definition}

\noindent Properties of $\vctproj[W_k]{x}$.
\begin{enumerate}[label = (\arabic*)] 
\item $\vctproj[W_k]{x} \in W_k$ because $x$ is a linear combination of $u_1, \ldots, u_k$,
which are basis vectors for $W_k$.
\item $(x - \vctproj[W_k]{x})$ is orthogonal to each $u_i$, $i = 1, \ldots, k$.
\begin{proof}
For each $u_i$, $i = 1, \ldots, k$ compute

\[ \langle u_i, x - \vctproj[W_k]{x} \rangle = \langle u_i, x \rangle - \langle u_i, p \rangle \]

where

\begin{align*}
\langle u_i , p \rangle &= \langle u_i, \alpha_1 u_1 + \cdots + \alpha_k u_k \rangle \\
&= \alpha_i \langle u_i, u_i \rangle \\
&= \alpha_i \\
&= \langle u_i, x \rangle.
\end{align*}

with $\alpha_k = \langle u_i, x \rangle$ and $\langle u_i, u_i \rangle = 1$. Therefore,
$\langle u_i, x - \vctproj[W_k]{x} \rangle = 0$ and so $x - \vctproj[W_k]{x} \perp u_i$
for all $i = 1, \ldots, k$.

Therefore $x - \vctproj[W_k]{x}$ is orthogonal to any vector in $W_k$.

\end{proof}
\end{enumerate}

\section{Gram-Schmidt Procedure/Process}

\noindent \textbf{Goal:} Construct on O.N. basis from a basis $B = \{v_1, \ldots, v_n\}$
for an n-dimensional IPS $V$.

\subsection{Preliminary Results}

We will use the following preliminary results:

\begin{enumerate}[label = (\arabic*)]
\item\begin{align*}
W_1 &= span\{v_1\} \\
W_2 &= span\{v_1, v_2\} \\
&\vdots \\
W_k &= span\{v_1, \ldots, v_k \} \\
&\vdots \\
W_n &= span\{v_1, \ldots, v_n\} = V
\end{align*}

\item Projection of $x \in V$ onto $W_k$, assuming $\{u_1, \ldots, u_k\}$ is an O.N.
basis for $W_k$ is

\[ \vctproj[W_k]{x} = \langle u_1, x \rangle u_1 + \cdots + \langle u_k, x \rangle u_k. \]

Then,
\begin{enumerate}[label = (\arabic*)]
\item $\vctproj[W_k]{x} \in W_k$
\item $(x - \vctproj[W_k]{x}) \perp (u_i)$ for all $i = 1, \ldots, k$.
\end{enumerate}

\end{enumerate}

\subsection{Gram-Schmidt Procedure (Inductive Process)}

\begin{itemize}
\item [\textbf{Step 1}] Local goal in this step is to construct on O.N. basis for $W_1$.
Since $W_1 = span\{v_1\}$ and $v_1 \neq 0$ then $\{v_1\}$ is a basis for $W_1$. Using the
normalization of $v_1$ we have

\[u_1 = \frac{v_1}{||v_1||} \in W_1 \]

and $u_1$ is a unit vector. Therefore, $\{u_1\}$ is an orthonormal basis for $W_1$.

\item [\textbf{Step 2}] Local goal is to construct O.N. basis for 
$W_2 = span \{v_1, v_2 \}$, where $W_1$ is a subspace of $W_2$.

From step 1, $\{u_1\}$ is an O.N. basis for $W_1 \subseteq W_2$. To generate a vector 
in $W_2$ that is orthogonal to $u_1$ , we consider

\begin{align*}
\tilde{u_2} &= v_2 - \vctproj[W_1]{v_2} \\
&= v_2 - \langle v_2, u_1 \rangle \cdot u_1 \in W_1.
\end{align*}

Properties of $\tilde{u_2}$:
\begin{enumerate}[label = (\arabic*)]
\item $\widetilde{u}_2 \in W_2$. Since $\widetilde{u_2}$ is a linear combination of $v_1$
and $v_2$ then $\widetilde{u}_2 \in W_2$.

\item $\widetilde{u}_{2} \neq 0$. This is because otherwise, 
$v_w = \langle v_2, u_1 \rangle \cdot u_1 = \text{ a multiple of } v_1$. 
(as $\displaystyle u_1 = \frac{v_1}{||v_1||}$).

\item $\widetilde{u}_2 \perp u_1$. Since $\{u_1, \widetilde{u}_2 \}$ is an
orthogonal set, and both $\widetilde{u}_2 \neq 0$ and $u_1 = 0$.
\end{enumerate}

Using the normalization of $\widetilde{u}_2$ we have

\[ u_2 = \frac{\widetilde{u}_2}{||\widetilde{u}_2||} =
\frac{v_2 - \langle u_1, u_2 \rangle u_1}{||v_2 - \langle u_1, v_2 \rangle u_1 ||} \in W_2. \]

Since $\{u_1, u_2\} \subset W_2$, $u_1 \perp u_2$, and $||u_1|| = ||u_2|| = 1$, then
$\{u_1, u_2\}$ is an orthonormal set in $W_2$ and therefore linearly independent. Thus,
$\{u_2, u_2\}$ is an O.N. basis $W_2$ where

\[ u_1 = \frac{v_1}{||v_1||}, \quad
u_2 = \frac{v_2 - \langle u_1, u_2 \rangle u_1}{||v_2 - \langle u_1, v_2 \rangle u_1 ||}. \]
\item[\textbf{Step $k + 1$:}] Where $1 \leq k \leq n-1$.
Local goal is to construct on O.N. basis for $W_{k+1}$.

\noindent \textbf{Induction Hypothesis:} $W_k$ has an orthonormal basis given by
$\{ u_1, \ldots, u_k \}$.

Now define

\begin{align*}
\widetilde{u}_{k+1} &= v_{k+1} - \vctproj[W_k]{v_{k+1}} \\
&= v_{k+1} - ( \langle u_1, v_{k+1} \rangle u_1 + \cdots + \langle u_k, v_{k+1} \rangle u_k) \in W_{k+1}
\end{align*}

Properties of $\widetilde{u}_{k+1}$
\begin{enumerate}[label = (\arabic*)]
\item $\widetilde{u}_{k_1} \in W_{k+1}$. This is because $u_{k+1} \in W_{k+1}$ and $\vctproj[W_k]{v_{k+1}} \in W_{k+1}$
such that $v_{k+1} - \vctproj[W_k]{v_{k+1}} = \widetilde{u}_{k+1}$ and $v_{k+1} - \vctproj[W_k]{v_{k+1}} \in W_{k+1}$.

\item $\widetilde{u}_{k+1} \neq 0$. Using a result from HW \# 8.

\item $\widetilde{u}_{k+1} \perp u_i$ for all $i = 1, \ldots, k$

Using the normalization of $\widetilde{u}_{k+1}$ we have

\[
u_{k+1} = \frac{\widetilde{u}_{k+1}}{|| \widetilde{u}_{k+1} ||} =
\frac{v_{k+1} - (\langle u_1, v_{k+1} \rangle u_1 + \cdots + \langle u_k, v_{k+1} \rangle u_k )}
{|| v_{k+1} - (\langle u_1, v_{k+1} \rangle u_1 + \cdots + \langle u_k, v_{k+1} \rangle u_k ) ||}
\].

Therefore, $\{u_1, \ldots, u_k, u_{k+1}\}$ is an O.N. set in $W_{k+1}$, which are all
orthogonal by assumption and 3) above. Since $\{u_1, \ldots, u_k, u_{k+1}\}$ is
an O.N. then it is linearly independent and therefore $\{u_1, \ldots, u_k, u_{k+1}\}$
is an O.N. basis for $W_{k+1}$.
\end{enumerate}

By the induction principle, we obtain an orthonormal basis $\{u_1, \ldots, u_n\}$ for $W_n = V$.
\end{itemize}

\subsection{Summary of Gram-Schmidt Procedure}

\begin{enumerate}[label = (\arabic*)]
\item $\displaystyle u_1 = \frac{v_1}{||v_1||}$

\item $\displaystyle u_2 = \frac{\widetilde{u}_2}{||\widetilde{u}_2||} =
\frac{v_2 - \langle u_1, u_2 \rangle u_1}{||v_2 - \langle u_1, v_2 \rangle u_1 ||}$

\end{enumerate}

\begin{enumerate}[label = (\alph*)]
\setcounter{enumi}{10}
\item $\displaystyle u_k = \frac{\displaystyle v_k - \sum^{k+1}_{i = 1} \langle u_i, v_k \rangle u_i}
{\displaystyle \left|\left| v_k - \sum^{k+1}_{i = 1} \langle u_i, v_k \rangle u_i \right|\right|}$


\setcounter{enumi}{13}
\item $\displaystyle u_n = \frac{\displaystyle v_n - \sum^{n-1}_{i = 1} \langle u_i, v_n \rangle u_i}
{\displaystyle \left|\left| v_n - \sum^{n-1}_{i = 1} \langle u_i, v_n \rangle u_i \right|\right|}$
\end{enumerate}

\begin{example}
Let $V = \mathbb{R}^3$ and $\displaystyle B = \left\{ \begin{bmatrix} 1 \\ 0 \\ 0 \end{bmatrix},
\begin{bmatrix} 1 \\ 1 \\ 0 \end{bmatrix}, \begin{bmatrix} 1 \\ 1 \\ 1 \end{bmatrix} \right\}$.
Find an O.N. basis for $\mathbb{R}^3$ from $B$.

\vspace{0.2cm}

\noindent \textit{Solution:}

\[ u_1 = \frac{v_1}{|| v_1 ||} = v_1 = \begin{bmatrix} 1 \\ 0 \\ 0 \end{bmatrix} \text{ because } || v_1 || = 1.\]


\[
\text{Then } v_2 - \langle u_1, v_2 \rangle \cdot u_1 = v_2 - u_1 = v_2 - v_1 = \begin{bmatrix} 0 \\ 1 \\ 0 \end{bmatrix}.
\]

\[
\text{ Normalizing we have } u_2 = \frac{v_2 - \langle u_1, v_2 \rangle u_1}{||v_2 - \langle u_1, v_2 \rangle u_1||}
= \begin{bmatrix} 0 \\ 1 \\ 0 \end{bmatrix}. 
\]

\[
\text{Then } v_3 - \langle u_1, v_3 \rangle u_1 - \langle u_2, v_3 \rangle u_2 = v_3 - u_1 - u_2 =
\begin{bmatrix} 0 \\ 0 \\ 1 \end{bmatrix}.
\]

\[\text{Therefore, } u_3 = \begin{bmatrix} 0 \\ 0 \\ 1 \end{bmatrix}.
\]

\[\text{Thus, } \{u_1, u_2, u_3\} \text{ is an O.N. basis for } \mathbb{R}^3\]

\end{example}

\section{Orthogonal and Unitary Matrices}

\begin{definition} $ $

A matrix $P \in \mathbb{R}^{n \times n}$ is \textbf{orthogonal} if its columns form an O.N.
basis for $\mathbb{R}^n$.

A matrix $U \in \mathbb{C}^{n \times n}$ is \textbf{unitary} if its columns form an O.N.
basis for $\mathbb{C}^n$.
\end{definition}

\begin{proposition}
Let $P \in \mathbb{R}^{n \times n}$. Then $P$ is orthogonal if and only if the columns of $P$,
$\{p_1, p_2, \ldots, p_n\}$, is an O.N. basis for $\mathbb{R}^n$.
\end{proposition}

\begin{proof}
$\Leftarrow$ Let $P = \begin{bmatrix} p_1 & p_2 & \cdots & p_n \end{bmatrix}$, where $p_k$ is the $k^{th}$
column of $P$. The condition that $\{p_1, p_2, \ldots, p_n\}$, is an O.N. basis for $\mathbb{R}^n$
is equivalent to 

\[\langle p_i, p_j \rangle = p_i^T p_j =
\begin{cases}
0, & \text{ if } i \neq j, \\
1, & \text{ if } i = j
\end{cases}.
\]

Then if we multiply $P^T$ by $P$ we have

\begin{align*}
P^T \cdot P &=
\begin{bmatrix}
p_1^T \\
p_2^T \\
\vdots \\
p_n^T
\end{bmatrix}
\begin{bmatrix}
p_1 & p_2 & \cdots & p_n
\end{bmatrix} \\
&=
\begin{bmatrix}
p_1^T & p_1^T p_2 & \cdots & p_1^T p_n \\
p_2^T & p_2^T p_2 & \cdots & p_2^T p_1 \\
\vdots & \vdots & \ddots & \vdots \\
p_n^T p_1 & p_n^T & \cdots & p_n^T p_n
\end{bmatrix} \\
&=
\begin{bmatrix}
1 & 0 & \cdots & 0 \\
0 & 1 & \cdots & 0 \\
\vdots & \vdots & \ddots & \vdots \\
0 & 0 & \cdots & 1
\end{bmatrix}
\end{align*}

By definition, $P$ is orthogonal if $P^TP = I_n$, thus if $\{p_1, p_2, \ldots, p_n\}$, 
is an O.N. basis for $\mathbb{R}^n$, then $P$ is orthogonal.

$(\Rightarrow)$ Conversely, if $P^TP = I_n$ then 

\[\langle p_i, p_j \rangle = p_i^T p_j =
\begin{cases}
0, & \text{ if } i \neq j, \\
1, & \text{ if } i = j
\end{cases}.
\]

Therefore, $\{p_1, p_2, \ldots, p_n\}$ is an O.N. basis for $\mathbb{R}^n$ and so
$P$ is orthogonal matrix.

In summary, $P$ is an orthogonal matrix $\iff$ $P^T \cdot P = I_n$.
\end{proof}

\begin{theorem}
Let $P \in \mathbb{R}^{n \times n}$. Then $P$ is an orthogonal matrix if and only if
$||P \cdot x|| = ||x||$ for all $x \in \mathbb{R}^n$, where $||x|| = \sqrt{x^Tx}$ is
the induced norm (or the 2-norm/Euclidean norm).

Here, $P$ is an \textbf{isometry}. (This means if does not change the length of the
norm).
\end{theorem}

\begin{proof}
''only if``: Suppose $P$ is orthogonal. Then $P^T P = I_n$. Then for a given
$x \in \mathbb{R}^n$, 
\begin{align*}
||Px||^2 &= (Px)^T(Px) \\
&= x^T \cdot (P^T \cdot P) \dot x\\
&= x^T \cdot x \\
&= ||x||^2.
\end{align*}

Then, taking square roots we have $||Px|| = ||x||$.

''If``: Suppose $||Px|| = ||x||$ for all $x \in \mathbb{R}^n$. Then

\begin{align*}
||Px||^2 = ||x||^2 &\iff (Px)^T (Px) = x^T x \\
&\iff x^T P^T \cdot Px - x^T I x = 0 \\
&\iff x^T \cdot (P^T P - I) \cdot x = 0.
\end{align*}

In the above equality we cannot conclude that $A = 0$. To see this, suppose
$x^T \cdot A \cdot x = 0$ for all $x \in \mathbb{R}^n$. Let 
$\displaystyle \begin{bmatrix} 0 & 1 \\ -1 & 0 \end{bmatrix}$. Then $A$ is
skew symmetric (i.e., $A^T = A$). Then for any $x \in \mathbb{R}^2$. Then

\[ x^TAx = \begin{bmatrix} x_1 & x_2 \end{bmatrix} 
\begin{bmatrix} x_2 \\ -x_1 \end{bmatrix} = x_1 x_2 - x_1 x_2 = 0.
\]

Now let $x = e_i$, $i = 1, \ldots, n$ where $e_i$ is the $i^{th}$ standard basis
vector of $\mathbb{R}^n$.

Then $Px = P \cdot e_i = p_i$, where $p_i$ is the $i^th$ column of $P$ and therefore
$||Px||^2 - ||p_i||^2$ and $||x||^2 = ||e_i||^2 = 1$. Since $||p_i||^2 = 1$ then
$p_i^T p - 1$ for all $i = 1, \ldots, n$.

Let $x = e_i + e_j$ where $i \neq j$. Then $e_i^T e_j = 0$ and
$P \cdot x - P(e_i + e_j) = Pe_i + Pe_j = P_i + P_j$. Further

\begin{align*}
||Px||^2 &= (p_i + p_j)^T \cdot (p_i + p_j) \\
&= P_i^T P_i + P_i^T P_j + P_j^T P_i + P_j^T P_j \\
&= 2 + 2P_i^T P_j.
\end{align*}

Also,
\begin{align*}
||x||^2 &= (e_i + e_j)^T (e_i + e_j) \\
&= e_i^T e_i + e_i^T + e_j^T e_i + e_j^T e_j \\
&= 2.
\end{align*}

Since $||Px||^2 = ||x||^2$ then $2 + 2 \cdot P_i^T P_j = 2$ and subtracting 2 and
dividing by 2 on both side we have $P_i^T P_j = 0$ for all $i \neq j$. Thus,
$P$ is an orthogonal matrix.
\end{proof}

The above theorem also holds for unitary matrices.

\begin{example} (Isometries)
\begin{enumerate}[label = (\arabic*)]
\item Let $P = I_n$, the identity matrix.
\item Let $P = \begin{bmatrix} \cos \theta & -\sin \theta \\ \sin\theta & \cos\theta \end{bmatrix}_{2 \times 2}$,
where $\theta \in (-\pi, \pi]$ in $\mathbb{R}^2$.

Then 

\begin{align*}
P^T \cdot P &= 
\begin{bmatrix}
\cos\theta & \sin\theta \\
-\sin\theta & \cos\theta
\end{bmatrix}
\begin{bmatrix}
\cos\theta & -\sin\theta \\
\sin\theta & \cos\theta
\end{bmatrix}
&=
\begin{bmatrix}
1 & 0 \\
0 & 1
\end{bmatrix}
&= I_2.
\end{align*}

Thus, $P$ is orthogonal.
\end{enumerate}
\end{example}

\begin{remark}
In (2) above the matrix $P = P(\theta)$ is a rotation matrix and $P(\theta)$ is commutative
in the sense that $P(\theta_1) - P(\theta_2) = P(\theta_1 + \theta_2)$.
\end{remark}

\section{Global Properties of Orthogonal (or Unitary) Matrices}

Let $O(n)$ be the set of all $n \times n$ orthogonal matrices. 
($O(n) \subseteq \mathbb{R}^{n \times n}$)

\vspace{0.5cm}

\noindent Properties of $O(n)$ under the matrix product ``$\cdot$''
\begin{enumerate}[label = (\arabic*)] 
\item Let $A, B \in O(n)$. \textbf{Claim:} $A \cdot B \in O(n)$.
\begin{proof}
Since $A, B \in O(n)$ then $A^TA = B^TB = I$. Then 

\[ (A \cdot B)^T \cdot (A \cdot B) = B^T \cdot A^T \cdot A \cdot B = I_n.\]

Therefore, $A \cdot B$ is orthogonal and so $A \cdot B \in O(n)$. Thus,
$O(n)$ is closed under ``$\cdot$''.
\end{proof}
\item The matrix product ``$\cdot$'' is associative, i.e., $(AB) \cdot C = A \cdot (BC)$.
\item The identity matrix $I_n \in O(n)$.
\item For any $A \in O(n)$, $A^{-1} \in O(n)$.

\begin{proof}
Since $A \in O(n)$ then $A^T \cdot A = I_n$ and so $A^{-1} = A^T$. Therefore,

\[(A^{-1})^T (A^{-1}) = (A^T)^T \cdot A^{-1} = A \cdot A^{-1} = I_n. \]

Therefore, $A^{-1} \in O(n)$.
\end{proof}
\end{enumerate}

\begin{remark}
From the above properties, $(O(n)$, ``$\cdot$'') is a group, more specifically a Lie group.

For any $P \in O(n)$, $P^T \cdot P = I_n$ and $\det (P^T \cdot P) = (\det P)^2$ and
$\det I_n = 1$. Therefore, $\det P = \pm 1$.

The Special $O(n)$ group is the set $SO(n) = \{P \in O(n) : \det P = 1\}$. This is the
Special Orthogonal Group of Order $n$. When $n = 2$, we get the space of all rotation
matrices. $SU(n)$ is the special unitary group of order $n$.
\end{remark}

\section{Projection}

Let $V$ be a vector space, and $X$ and $Y$ be two subspaces of $V$.

\begin{remark}
Recall that $X + Y = \{x + y : x \in X \text{ and } y \in Y\}$ and
$X \cap Y = \{ v : v \in X \text{ and } v \in Y\}$ are subspaces of $V$ and
$Y \subseteq X + Y$ and $x \subseteq X + Y$.

If $X$ and $Y$ are finite dimensional, then 
$\dim(X + Y) = \dim(X) + \dim(Y) - \dim(X \cap Y)$
\end{remark}

\begin{definition}
The subspaces $X$ and $Y$ are \textbf{complementary} subspaces of $V$ if

\begin{enumerate}[label = (\arabic*)]
\item $V = X + Y$
\item $X \cap Y = \{0\}$
\end{enumerate}

In this case, $V$ is the \textbf{direct sum} of $X$ and $Y$, and it is
written as

\[ V = X \oplus Y.\]
\end{definition}


\begin{theorem}
Let $X$ and $Y$ be subspaces of $V$ with bases $B_x = \{u_1, \ldots, u_p\}$ and
$B_y = \{v_1, \ldots, v_r\}$ for $X$ and $Y$, respectively. Then the following are
equivalent:

\begin{enumerate}
\item $V = X \oplus Y$
\item For any $v \in V$, there exists ``unique'' vectors $x \in X$ and $y \in Y$ such that
$V = X + Y$.
\item $B_x \cap B_y = \varnothing$ and $B_x \cup B_y$ is a basis for $V$.
\end{enumerate}
\end{theorem}

\begin{proof}
We will prove the following chain of implications: $1 \Rightarrow 2 \Rightarrow 3 \Rightarrow 1$.

\begin{itemize}
\item[($1 \Rightarrow 2$)] Since $V = X \oplus Y$, then $V = X + Y$, and $X \cap Y = \{0\}$.
For an arbitrary vector $v \in V$, $v \in V = X + Y$ and therefore, there exists an $x \in X$ and
a $y \in Y$ such that $v = x + y$. To prove uniqueness, suppose that there exists an $x^{\prime} \in X$
and a $y^{\prime}$ such that $x^{\prime} + y^{\prime} = v$. Since $x + y = v$, then 
$x + y = v = x^{\prime} + y^{\prime}$. Therefore, $x - x^{\prime} = y^{\prime} - y$. Since $x \in X$
and $x^{\prime} \in X$ and $X$ is a subspace, then $x - x^{\prime} \in X$ (because $X$ is
closed under ``$+$'' and ``$\cdot$''). Likewise, $y - y^{\prime} \in Y$ because $y \in Y$
and $y^{\prime} \in Y$ and $Y$ is a subspace. Since $x - x^{\prime} = y - y^{\prime}$ and
$X \cap Y = \{0\}$, then $x - x^{\prime} = y - y^{\prime} = 0$, which implies that
$x^{\prime} = x$ and $y^{\prime} = y$. This proves uniqueness.

\item[($2 \Rightarrow 3$)] First we show that $B_X \cap B_Y = \varnothing$. Suppose not. Then
there exists a $v \in B_X \cap B_Y$ such that $v \neq 0$ and $v \in B_X$ and $v \in B_Y$.
Also, $v \in X \cap Y$. However, $v = v + 0 = 0 + v$ where $v \in X$ and $0 \in Y$ in the
first inequality and $0 \in X$ and $v \in Y$ in the second equality. However, this contradicts
that unique decomposition in (2).


Now we show that $B_X \cup B_Y$ is a basis for $V$. Since $B_X \cap B_Y = \varnothing$ then
$B_X \cup B_Y = \{u_1, \ldots, u_p, v_1, \ldots, v_r \}$. To show that $B_X \cup B_Y$ spans
$V$, we see that for all $v \in V$, $v = x + y$, where $x \in X$ and $y \in Y$. Since
$B_X$ is a basis for $X$ then $x = \alpha_1 u_1 + \cdots + \alpha_p u_p$ for some
scalars $\alpha_i$ for $i = 1, \ldots, p$. Likewise, 
$y = \beta_1 v_1 + \cdots + \beta_r v_r$ for some scalars $\beta_i$, $i = 1, \ldots, r$.
Then
\begin{align*}
v &= x + y \\
&= \alpha_1 u_1 + \cdots + \alpha_p u_p + \beta_1 v_1 + \cdots + \beta_r v_r.
\end{align*}

Therefore $v \in span(B_X \cup B_Y)$ and since $v \in V$ was arbitrary, this shows
that $V = span(B_X \cup B_Y)$. This proves the spanning property. To show that $B_X \cup B_Y$
is linearly independent, let $\alpha_i$ and $\beta_i$ be such that


\[\alpha_1 u_1 + \cdots + \alpha_p u_p + \beta_1 v_1 + \cdots + \beta_r v_r = 0\]

where $(\alpha_1 u_1 + \cdots + \alpha_p u_p) \in X$ and 
$(\beta_1 v_1 + \cdots + \beta_r v_r) \in Y$. By part (2) we can write the zero vector in $V$
as the unique composition $0 = 0 + 0$ where the first zero is in $X$ and the second zero is in
$Y$. Therefore $\alpha_1 u_1 + \cdots + \alpha_p u_p = 0$ and 
$\beta_1 v_1 + \cdots + \beta_r v_r = 0$.  Since $B_X$ and $B_Y$ are linearly independent, then
$\alpha_i = 0$ for all $i$ and $beta_j = 0$ for all $j$. Thus, $B_X \cup B_Y$ is linearly
independent, which implies that $B_X \cup B_Y$ is a basis for $V$.

\item[($3 \Rightarrow 1$)] Assume $B_X \cap B_Y = \varnothing$ and $B_X \cup B_Y$ is
a basis for $V$. First we show that $V = X + Y$. Since $B_X \cup B_Y$ is a basis for $V$,
then for any $v \in V$, we can write $v$ as linear combination of 
$u_1, \ldots, u_p, v_1, \ldots, v_r$. Therefore

\[v = (\alpha_1 v_1 + \cdots + \alpha_p u_p) + (\beta_1 v_1 + \cdots + \beta_r v_r) \]

where $(\alpha_1 v_1 + \cdots + \alpha_p u_p) \in X$ and
$(\beta_1 v_1 + \cdots + \beta_r v_r) \in Y$. Thus, $v \in X + Y$ and since $v \in V$
was arbitrary, then $V = X + Y$.

To show that $X \cap Y = \{0\}$, let $z \in X \cap Y$. Then $z \in Z$ and $z \in Y$. Since
$B_X = \{u_1, \ldots, u_p\}$ is a basis for $X$, then $z = \alpha_1 u_1 + \cdots + \alpha_p u_p$.
Likewise, $z = \beta_1 v_1 + \cdots + \beta_r v_r$. Then 

\[\alpha_1 u_1 + \cdots + \alpha_p u_p = \beta_1 v_1 + \cdots + \beta_r v_r\]

and subtracting all terms from the right hand side we have

\[\alpha_1 u_1 + \cdots + \alpha_p u_p - \beta_1 v_1 - \cdots - \beta_r v_r = 0.\]

Since $B_X \cup B_Y$ is basis for $V$, then $B_X \cup B_Y$ is linearly independent. Therefore,
since the above expression is a linear combination of vectors from $B_X \cup B_Y$, then
$\alpha_i = 0$ for all $i$ and $-\beta_j = 0$ for $j$, which implies that $\beta_j = 0$ for
all $j$. This implies that $z = \alpha_1 u_1 + \cdots + \alpha_p u_p = 0$ and so $X \cap Y = \{0\}$.

Thus, $V = X \oplus Y$.
\end{itemize}
\end{proof}

\begin{definition}
\noindent Suppose $V = X \oplus Y$. Since for any $v \in V$ there exists a unique $x \in X$ and $y \in Y$
such that $v = x + y$, then $v \rightarrow x$ defines a function and $v \rightarrow y$ defines a
function.

Here we call $x$ the \textbf{projection of $v \in V$ onto $X$} (along $Y$) and we call $y$ the
\textbf{projection of $v \in V$ onto $Y$} (along $X$). 
\end{definition}

\begin{definition}
A linear operator $P : V \rightarrow V$ is a \textbf{projector} if there exists complementary
subspaces $X$ and $Y$ of $V$, such that $Pv = X$ for all $v \in V$, where $v = x + y$ for
$x \in X$ and $y \in Y$.
\end{definition}

\begin{remark} $ $
In the above definition, $v = x + y$ is a key decomposition that is used in solving problems.
\end{remark}

\begin{remark} Facts about a projector:

\begin{enumerate}[label = (\arabic*)]
\item For any $x \in X$, $P \cdot x = x$ because $x = x + 0$, where $x \in X$ and $0 \in Y$.
\item for any $y \in Y$, $P \cdot y = 0$ because $y = 0 + y$, where $0 \in X$ and $y \in Y$.
\item For any two linear operators $R$ and $Q$, $R = Q$ if and only if $Rv = Qv$ for all
$v \in V$. (if given any input, the output is the same).
\end{enumerate}
\end{remark}

\begin{theorem}
Let $P: V \rightarrow V$ be a projector matrix onto the subspace $X$ along $Y$. Then

\begin{enumerate}[label = (\arabic*)]
\item $P^2 (= P \circ P) = P$ (idempotent property)
\item $I - P$ is the projector onto $Y$ along $X$.
\item $R(P) = \{v \in V : Pv = v \} (=S) = X$.
\end{enumerate}
\end{theorem}

\begin{proof}
\begin{enumerate}[label = (\arabic*)]
\item We want to show that $P^2 v = Pv$ for all $v \in V$. Since $P$ is a projector matrix,
then $X$ are complementary subspaces. By the above theorem, we can uniquely write any $v \in V$
as $v = x + y$, where $x \in X$ and $y \in Y$. Since $Pv = x \in X$, then 
$P^2 v = P(Pv) = P \cdot x = x = Pv$. Therefore, since $P^2 v = P v$ for all $v \in V$,
then $P^2 = P$.

\item For any $v \in V$ we have the unique decomposition of $v$ as $v = x + y$ where $x \in X$
and $y \in Y$, with $Pv = x$. Now

\begin{align*}
(I - P)v &= Iv - Pv \\
&= v - Pv \\
&= v - x \\
&= x + y - x \\
&= y.
\end{align*}

Therefore, for any $v \in V$, $(I - P)(v) = y$, which is the $y$ part of the unique decomposition
$v = x + y$. Thus, $I - P$ is the projector onto $Y$ along $X$.

\item First we show that $R(P) = S$. Clearly, for any $v \in S$, $Pv = v$ where $Pv \in R(P)$.
This implies that $v \in R(P)$. Therefore, $S \subseteq R(P)$.

Conversely, let $z \in R(P)$. Then $z = P \cdot v$ for some $v \in V$ and so
$Pz = P(Pv) = P^2 (v)$. Now since $P^2 = P$ (idempotent property) then $P^2 v = Pv = z$, which
implies that $Pz = z$. Thus, $z \in S$. Since $S \subseteq R(P)$ and $R(P) \subseteq S$, then
$R(P) = S$.

Now we show that $X = S$. First, $X \subseteq S$ because for any $x \in X$, $Px = x \in S$,
which implies that $X \subseteq S$. To show that $S \subseteq X$, let $z \in S$. Then $Pz = z$.
Since $Pv \in X$ for $v \in V$ then $Pz \in X$ and so $z \in X$, which implies that $S \subseteq X$.
Since we have shown that $X \subseteq S$ and $S \subseteq X$, then $X = S$.
\end{enumerate}
\end{proof}

In fact, $N(P) = Y$, which we show in the homework.

\section{Matrix representation of a projector $P$ on $\mathbb{R}^n$}

Let $V = \mathbb{R}^n$ and $X, Y$ be complementary subspaces. Let $P$ be the projector onto $X$
along $Y$. Then $R(P) = X$ and $N(P) = Y$.

Let $B_X = \{x_1, \ldots, x_r\}$ and $B_Y = \{y_1, \ldots, y_{n-r} \}$. Since $B_X \cap B_Y = \varnothing$
and $B = B_X \cup B_Y = \{x_1, \ldots, x_r, y_1, \ldots, y_{n-r} \}$ is a basis for $\mathbb{R}^n$, then

\[ [P]_B = \begin{bmatrix} [Px_1]_B & & [Px_r]_B & [Py_1]_B & & [By_{n-r}]_B \end{bmatrix} \] 

where $Px_i = x_i$ ($x_i \in X$) for all $i = 1, \ldots, r$ and $Py_i = 0$ ($y_i \in Y$) for all
$i = 1, \ldots, n-r$. Then

\[ [P]_B = \begin{bmatrix} [x_1]_B & \cdots & [x_r]_B & 0 & \cdots & 0 \end{bmatrix} =
\begin{bmatrix}
I_r & 0 \\
0 & 0
\end{bmatrix}
\]
where $[x_1]_B = e_1, \ldots, [x_r]_B = e_r$. Let $\tilde{B} = \{e_1, e_2, \ldots, e_n\}$ be the
standard basis for $\mathbb{R}^n$. Then

\[ [P]_{\tilde{B}} = [I]_{B \tilde{B}} \cdot [P]_B \cdot \left([I]_{B \tilde{B}} \right)^{-1} \]

where


\begin{align*}
[I]_{B \tilde{B}} &= \begin{bmatrix} [Ix_1]_{\tilde{B}} & \cdots & [Ix_r]_{\tilde{B}} & [Iy_1]_{\tilde{B}}
& \cdots & [Iy_{n-r}]_{\tilde{B}} \end{bmatrix} \\
&= \begin{bmatrix} x_1 & \cdots & x_r & y_1 & \cdots & y_{n-r} \end{bmatrix}_{n \times n}.
\end{align*}

Therefore

\begin{align*}
[P]_{\tilde{B}} &= \begin{bmatrix} x_1 & \cdots & x_r & y_1 & \cdots & y_{n-r} \end{bmatrix}
\begin{bmatrix}
I_r & 0 \\
0 & 0
\end{bmatrix}
\left(
\begin{bmatrix}
x_1 & \cdots & x_r & y_1 & \cdots & y_{n-r}
\end{bmatrix}^{-1}
\right) \\
&= \begin{bmatrix} B_X & 0 \end{bmatrix} \begin{bmatrix} B_X & B_Y \end{bmatrix}^{-1}.
\end{align*}

Therefore, the matrix representation of the projector $P$ is

\[ [P]_{\tilde{B}} = \begin{bmatrix} B_X & 0 \end{bmatrix} \cdot \begin{bmatrix} B_X & B_Y 
\end{bmatrix}^{-1}. \]

\begin{example}
Let $V = \mathbb{R}^3$. Let $X = span \left\{ \begin{bmatrix} 1 \\ 0 \\ 0 \end{bmatrix},
\begin{bmatrix} 0 \\ 1 \\ 0 \end{bmatrix} \right\}$ and
$Y = span \left\{ \begin{bmatrix} 1 \\ 1 \\ 1 \end{bmatrix} \right\}$.

Since $B_X \cap B_Y = \varnothing$ and $B_X \cup B_Y$ is a basis for $\mathbb{R}^3$ then
$V = X \oplus Y$.

To find the matrix representation of the projector $P$ onto the $X$ space,

\begin{align*}
[P]_{\tilde{B}} &= \begin{bmatrix} B_X & 0 \end{bmatrix} \cdot 
\begin{bmatrix} B_X & B_Y \end{bmatrix}^{-1} \\
&=
\begin{bmatrix}
1 & 0 & 0 \\
0 & 1 & 0 \\
0 & 0 & 0 
\end{bmatrix}
\cdot
\begin{bmatrix}
1 & 0 & 1 \\
0 & 1 & 1 \\
0 & 0 & 1
\end{bmatrix}^{-1} \\
&= 
\begin{bmatrix}
1 & 0 & -1 \\
0 & 1 & -1 \\
0 & 0 & 0
\end{bmatrix}.
\end{align*}
\end{example}

\begin{theorem}
Let $P : V \rightarrow V$ be a linear operator. Then $P$ is a projector if and only if
$P^2 = P$, i.e., $P$, is idempotent.
\end{theorem}

\begin{proof}
$(\rightarrow)$ If $P$ is a projector then $P^2 = P$, from theorem above.

$(\leftarrow)$ Suppose $P^2 = P$. We will show that $V = R(P) \oplus N(P)$, i.e.,
$R(P)$ and $N(P)$ are complementary subspaces of $V$.

First we show that $V = R(P) + N(P)$. For any $v \in V$, $v = Pv + (v - Pv) = Pv + (I - P)v$.
Clearly, $Pv \in R(P)$. Since $P(I-P)v = Pv - P^2v$ and $P^2 = P$ then $P(I - P)v = 0$ and
so $(I - P)v \in N(P)$. Therefore, $v \in R(P) + N(P)$. Since $v \in V$ was arbitrary this
shows that $V = R(P) + N(P)$.

Now we show that $R(P) \cap N(P) = \{0\}$. Let $z \in R(P) \cap N(P)$. Then $z \in R(P)$ and
$z \in N(P)$, which means $z = Pv$ for some $v \in V$ and $Pz = 0$. 
Therefore $0 = P \cdot z = P(P \cdot v) = P^2v$ and since $P$ is idempotent then
$Pv = z$. Thus, $z = 0$. Since $z$ was arbitrary then this shows that
$R(P) \cap N(P) = \{0\}$ and so $V = R(P) \oplus N(P)$.

Therefore, for any $v \in V$, $v$ is uniquely decomposed into the sum

\[v = x + y\]

where $x \in R(P)$ and $y \in N(P)$. Therefore, $P$ is a projector onto $R(P)$ along $N(P)$.
\end{proof}

\section{Orthogonal Complement}

Let $(V, \langle \cdot, \cdot \rangle)$ be an inner product space. This is a general assumption in
this section.

\begin{definition}
A vector $v \in V$ is orthogonal to a non-empty set $S$ in $V$ if it is orthogonal to each
vector in $S$, i.e., $v \perp z$ for all $z \in S$. In this case, $v \perp S$.
\end{definition}

\begin{proposition}
Let $M$ be a subspace of $V$ with a basis $B_M = \{u_1, \ldots, u_p\}$. Then a vector $v \perp M$ if
and only if $v \perp u_i$ for all $i = 1, \ldots, p$.
\end{proposition}

\begin{proof}
$(\Rightarrow)$ Suppose $v \perp M$. Then $v \perp z$ for all $z \in M$. Since $u_i \in M$ for all 
$i = 1, \ldots, p$ then $v \perp u_i$ for all $i = 1, \ldots, p$.

$(\Leftarrow)$ Suppose $v \perp u_i$ for all $i = 1, \ldots, p$. Since $B_M$ is a basis for
$M$, then for any $z \in M$ we can write as

\[z = \alpha_1 u_1 + \cdots + \alpha_p u_p\]

for some scalars $\alpha_1, \ldots, \alpha_p$. Now consider the inner product of $v$ and $z$,
which is

\begin{align*}
\langle v, z \rangle &= \langle v, \alpha_1 u_1 + \alpha_2 u_2 + \cdots + \alpha_p u_p \rangle \\
&= \alpha_1 \langle v, u_1 \rangle + \cdots + \alpha_p \langle v, u_p \rangle \\
&= 0.
\end{align*}

Therefore, $v \perp z$ for all $z \in M$ and so $v \perp M$.
\end{proof}

\begin{definition}
Given a non-empty set $S$ in $V$, the orthogonal complement of $S$, denoted by $S^{\perp}$,
is the collection of all vectors in $V$ that are orthogonal to $S$, i.e.,

\[S^{\perp} = \{v \in V : v \perp S \} = \{v \in V : v \perp z, \forall z \in S\}\]
\end{definition}

\begin{proposition}
For any non-empty set $S$, $S^{\perp}$ is a subspace.
\end{proposition}

\begin{proof}
We proved in a homework exercise.
\end{proof}

\begin{theorem}
Let $V$ be a finite dimensional inner product space, and $M$ be a subspace of $V$. Then

\[V = M \oplus M^{\perp} \]

i.e., $M$ and its orthogonal complement $M^{\perp}$ are complementary subspaces of $V$.
\end{theorem}

\begin{proof}
First we show that $M \cap M^{\perp} = \{ 0 \}$. Let $z \in M \cap M^{\perp}$. Then 
$z \in M^{\perp}$ and $z \in M$. Since $z \in M^{\perp}$ then $z \perp u$ for all
$u \in M$. Then since $u \in M$, $z \perp z$ and so $\langle z, z \rangle = 0$. Thus,
$z = 0$. Since $z$ was arbitrary, this shows that $M \cap M^{\perp} = \{ 0 \}$.

Now we show that $V = M + M^{\perp}$ . $V$ is a finite dimensional inner product space.
Therefore $M$ and $M^{\perp}$ are finite dimensional product spaces such that $M$ has
an orthonormal basis $B_M$, and $M^{\perp}$ has an orthonormal basis $B_{M^{\perp}}$.
Since $B \subseteq M$, $B_{M^{\perp}} \subseteq M^{\perp}$, and
$M \cap M^{\perp} = \{ 0 \}$, then $B_M \cap B_{M^{\perp}} = \varnothing$ and
$B_M \cup B_{M^{\perp}}$ is an orthonormal set in $M$. Therefore, $B_M \cup B_{M^{\perp}}$
is linearly independent, which implies that $B_M \cup B_{M^{\perp}}$ spans $M + M^{\perp}$.
Therefore, $B_M \cup B_{M^{\perp}}$ is a basis for $M + M^{\perp}$.

\textbf{Claim:} $B_M \cup B_{M^{\perp}}$ spans $V$.

Suppose not. Then there exists a $u \in V$ such that $u \notin span(B_M \cup B_{M^{\perp}}) =
M + M^{\perp}$. Therefore, by the Gram-Schmidt procedure,

\[
w = u - \vctproj[M + M^{\perp}]{u} \neq 0
\text{ and }
w \perp M + M^{\perp}\]

where $\vctproj[M + M^{\perp}]{u}$ is the orthogonal projection of $u$ onto $M + M^{\perp}$.
Note that $M \subseteq M + M^{\perp}$ (because we can write $v = v + 0$ where $v \in M$ and
$0 \in M^{\perp}$.) Since $w \perp M + M^{\perp}$ then $w \perp M$, which implies that
$w \in M^{\perp}$. On the other hand, $M^{\perp} \subseteq M + M^{\perp}$. Therefore,
$w \in M^{\perp} \subseteq M + M^{\perp}$ and so $w \in M + M^{\perp}$. Therefore,
$w \perp w$ which implies that $w = 0$. However, this contradict the fact above that
$w \neq 0$. Therefore, $B_M \cup B_{M^{\perp}}$ spans $V$, i.e., $V = M + M^{\perp}$.
Thus, $V = M \oplus M^{\perp}$.
\end{proof}

\begin{proposition}
Let $V$ be a finite dimensional inner product space and $M$ is a subspace of $V$. Then,
$(M^{\perp})^{\perp} = M$.
\end{proposition}

\begin{proof}
First we prove that $M \subseteq (M^{\perp})^{\perp}$. Let $v \in M$. Therefore, for any
$v \in M^{\perp}$, $w \perp v$. Because $v \perp w$ for all $w \in M^{\perp}$, then
$v \in (M^{\perp})^{\perp}$, which implies that $M \subseteq (M^{\perp})^{\perp}$.

Now we prove that $(M^{\perp})^{\perp} \subseteq M$. Let $v \in (M^{\perp})^{\perp}
\subseteq V$. Since $V = M \oplus M^{\perp}$, then we can write $v$ as
$v = u + w$ where $u \in M$ and $w \in M^{\perp}$. Since $v \in (M^{\perp})^{\perp}$
and $w \in M^{\perp}$, then $v \perp w$ or equivalently, $\langle w, v \rangle = 0$.
Since $v = u + w$, then $\langle w, v \rangle = \langle w, u + w \rangle = langle w, u 
\rangle + langle w , w \rangle$. Because $u \in M$ and $w \in M^{\perp}$, then 
$w \perp u$ or $langle w, u \rangle = 0$. Therefore, $0 = 0 + \langle w, w \rangle$,
which implies that $\langle w, w \rangle = 0$ and so $w = 0$. Thus, $v = u \in M$.
Since $v \in M$ then $(M^{\perp})^{\perp} \subseteq M$.
\end{proof}

\section{Orthogonal Complements of Fundamental Subspaces of a Matrix}

Let $A$ be an $m \times m$ real matrix, i.e., $A \in \mathbb{R}^{m \times m}$. Then

\begin{itemize}
\item $R(A)$ and $N(A^T)$ are subspaces of $\mathbb{R}^m$
\item $R(A^T)$ and $N(A)$ are subspaces of $\mathbb{R}^n$
\end{itemize}

\begin{proposition}
$[R(A)]^{\perp} = N(A^T) \text{ and } [R(A^T)]^{\perp} = N(A)$.
\end{proposition}

\begin{proof}
\begin{align*}
x \in [R(A)]^{\perp} &\iff x \perp z, \forall z \in R(A) \\
&\iff z^T x = 0, \forall z \in R(A) \\
&\iff (Ay)^T \cdot x = 0, \forall y \in \mathbb{R}^n \\
&\iff y^T \cdot (A^T \cdot x) = 0, \forall y \in \mathbb{R}^n \\
&\iff A^T x = 0 \\
&\iff x \in N(A^T).
\end{align*}

Therefore, $[R(A)]^T = N(A^T)$.
\end{proof}

\begin{remark}
It follows that $\mathbb{R}^m = R(A) \oplus N(A^T)$ and $\mathbb{R}^n = R(A^T) \oplus N(A)$.
\end{remark}

\section{Normal Matrix}

\begin{definition}
An $n \times n$ complex matrix $A$ is normal if $A^* A = A A^*$.
\end{definition}

A special case is when $A \in \mathbb{R}^{n \times n}$, then $A$ is normal $\iff$ 
$A^T A - A A^T$.

\begin{example}
\begin{enumerate}[label = (\arabic*)]
\item $A$ is real symmetric/skewed symmetric.
\item a is hermitian/skewed hermitian.
\end{enumerate}
\end{example}

\begin{proposition}
Let $A \in \mathbb{R}^{n \times n}$ be a normal matrix. Then $[R(A)]^{\perp} = N(A)$ and
$\mathbb{R}^n = R(A) \oplus N(A)$.
\end{proposition}

\begin{proof}
We use the fact that $N(A^T) = N(A \cdot A^T)$ and $N(A) = N(A^T A)$. Note that
$[R(A)]^T = N(A^T) = N(A \cdot A^T) = N(A^T \cdot A)$. (Since $A$ is normal.) Then
$N(A^T A) = N(A)$. Then $[R(A)]^{\perp} = N(A)$. Thus, $\mathbb{R}^n = R(A) \oplus N(A)$.
\end{proof}

\section{Orthogonal Projection}

Let $V$ be a finite-dimensional inner product space. Suppose $V = M \oplus M^{\perp}$,
where $M$ is a subspace of $V$ and $M^{\perp}$ is the orthogonal complement of $M$. Then
for any $v \in V$, $v = m + n$, where $m \ M$, $n \in M^{\perp}$ (unique decomposition).
This gives the linear operator $P_M : V \rightarrow V$, given by $P_M (v) = m \in M$. Here
$P_M$ is \textbf{orthogonal projector} onto $M$ space (We showed this in HW\#10, Question 4).

\subsection{Orthogonal Projectors onto $M$}

Let $B_M = \{ u_1, \ldots, u_p \}$ be an orthonormal basis for $M$ and 
$B_{M^{\perp}} = \{ w_1, \ldots, w_r \}$ be an orthonormal basis for $M^{\perp}$. Then,
$B_M \cap B_{M^{\perp}} = \varnothing$ and $B_M \cup B_{M^{\perp}}$ is an
orthonormal basis for $V = M \oplus M^{\perp}$. Then let 
$B = B_M \cup B_{M^{\perp}} = \{ u_1, \ldots , u_p, w_1, \ldots, w_r\}$. Then for all
$v \in V$, $v = \langle v, u_1 \rangle u_1 + \cdots + \langle v, u_p \rangle u_p +
\langle v, w_1 \rangle w_1 + \cdots + \langle v, w_r \rangle w_r $ where
$\langle v, u_1 \rangle u_1 + \cdots + \langle v, u_p \rangle u_p = u \in M$ and
$\langle v, w_1 \rangle w_1 + \cdots + \langle v, w_r \rangle w_r = w \in M^{\perp}$.
Then the projection of $v$ onto $M$ is,

\[ \vctproj[M]{v} = u = \langle v, u_1 \rangle u_1 + \cdots + \langle v, u_p \rangle u_p.
\]

\subsection{Matrix Representation of an Orthogonal Projector $P_M$ on $V = \mathbb{R}^n$}

Let $B_M = \{ u_1, \ldots, u_p \}$ be an arbitrary basis for $M$, and 
$B_{M^{\perp}} = \{ w_1, \ldots, w_{n - p} \}$ be an orthonormal basis for $M^{\perp}$.
Then let the matrices $S$ and $Q$ be

\[ S = \begin{bmatrix} u_1, \ldots, u_p \end{bmatrix} \text{ and }
Q = \begin{bmatrix} w_1, \ldots, w_{n - p} \end{bmatrix}.
\]

\noindent Facts about $S$ and $Q$:
\begin{enumerate}[label = (\arabic*)]
\item The columns of $S$ and $Q$ are linearly independent. Then $Sx = 0$ implies that
$x = 0$. Therefore, $S^{\perp} S$ is invertible because

\[S^{\perp} Sx = 0 \implies x^T S^T S x = 0 \implies (Sx)^T (Sx) = 0 \implies Sx = 
0 \implies x = 0 \]

\item $Q^T Q = I_{n - p}$.

\item 
\[S^T Q = 
\begin{bmatrix}
u_1^{T} \\
\vdots \\
u_p^{T}
\end{bmatrix}
\begin{bmatrix}
w_1, \ldots, w_{n - p}
\end{bmatrix}
=
\begin{bmatrix}
u_1^T w_1 & \cdots & u_1^T w_{n - p} \\
\vdots & & \vdots \\
u_p^T w_1 & \cdots & u_p^T w_{n - p}
\end{bmatrix}
=
0.
\]
Since $\langle u_i, w_i \rangle = 0$ where $u_i \in M$ and $w_j \in M^{\perp}$.

Similarly, $Q^T S = 0$ because $Q^T S = (S^T Q )^T$.
\end{enumerate}


\begin{proposition}
The matrix representation of $P_M$ given by $W = S(S^T \cdot S)^{-1} S^T$, where 
$W \in \mathbb{R}^{n \times n}$ and
$S = \begin{bmatrix} u_1, \ldots, u_p \end{bmatrix} \in \mathbb{R}^{n \times p}$,
\end{proposition}

\begin{proof}
Recall that $W = \begin{bmatrix} S & 0 \end{bmatrix} \cdot 
\begin{bmatrix} S & Q \end{bmatrix}^{-1}$, by results for general projection.

We claim that 

\[\begin{bmatrix} S & Q \end{bmatrix}^{-1}
=
\begin{bmatrix}
(S^T S)^{-1} S^T \\
Q^T
\end{bmatrix}.
\]

Since

\begin{align*}
\begin{bmatrix}
(S^T S)^{-1} S^T \\
Q^T
\end{bmatrix}
\begin{bmatrix}
S & Q
\end{bmatrix}
=
\begin{bmatrix}
(S^T S)^{-1} S^T S & (S^T S)^{-1} S^T Q \\
Q^T S & Q^T Q
\end{bmatrix}
=
\begin{bmatrix}
I_p & 0 \\
0 & I_{n - p}
\end{bmatrix}
\end{align*}

Thus,

\[
\begin{bmatrix}
S & Q
\end{bmatrix}^{-1}
=
\begin{bmatrix}
(S^T S)^{-1} S^T \\
Q^T
\end{bmatrix}.
\]

Then,
\[
W = 
\begin{bmatrix}
S & 0
\end{bmatrix}
\begin{bmatrix}
(S^T S)^{-1} S^T \\
Q^T
\end{bmatrix}
=
S(S^T S)^{-1} S^T.
\]
\end{proof}

\begin{remark} $ $
\begin{enumerate}[label = (\arabic*)]
\item The matrix representation of $P_M$ is independent of the basis for $M$.
\item $W = W^T$ (i.e., $W$ is symmetric).

To see this, note that

\begin{align*}
W^T = \begin{bmatrix} S(S^T S)^{-1} S^T \end{bmatrix} &=
(S^T)^T \cdot \begin{bmatrix} (S^T S)^{-1} \end{bmatrix}^T \cdot S^T \\
&= S \cdot \begin{bmatrix} (S^T S)^T \end{bmatrix}^{-1} \cdot S^T \\
&= S \cdot \begin{bmatrix} S^T S \end{bmatrix}^{-1} S^T \\
&= W
\end{align*}
Therefore, $W^T = W$ and so $W$ is symmetric.
\end{enumerate}
\end{remark}


\begin{theorem} Suppose $P : \mathbb{R}^n \rightarrow \mathbb{R}^n$ is a projector
($P^2 = P$). Let $W$ be its matrix representation. Then $P$ is an orthogonal projector
if and only if $W = W^T$.
\end{theorem}

\begin{proof}
$(\Rightarrow)$ We already covered.

$(\Leftarrow)$ Suppose $W$ is symmetric, i.e., $W = W^T$. Since $P$ is a projector and $W$ 
is the matrix representation, then $\mathbb{R}^n = R(P) \oplus N(P)$ and $R(P) = R(W)$, 
$N(P) = R(W)$. To show that $P$ is orthogonal projector, we want to show $R(P) \perp N(P)$
(or equivalently $R(W) \perp N(W)$). Recall that $\begin{bmatrix} R(W) \end{bmatrix}^{\perp}
= N(W^T)$. Since $W$ is symmetric then $W^T = W$ and so 
$\begin{bmatrix} R(W) \end{bmatrix}^{\perp} = N(W)$. Then $\mathbb{R}^n = R(W) \oplus N(W)$
and $R(W) \perp N(W)$. Thus $P$ is an orthogonal projector.
\end{proof}

\begin{theorem}
Let $P_M$ be an orthogonal projection onto the subspace $M$. Then, for all $z \in V$, 
$P_M(z) \in M$ is the closest point in $M$ to $z$, i.e., for all $u \in M$,
$|| z - \vctproj[M]{z} || \leq ||u - z||$.
\end{theorem}

\begin{proof}
We will use the facts:

\begin{enumerate}[label = (\arabic*)]
\item IF $x \perp y$, then $||x + y||^2 = ||x||^2 + ||y||^2$.
\item $(z - \vctproj[M]{z}) \perp M$, which implies $z - \vctproj[M]{z} \in M^{\perp}$.
\end{enumerate}

For any $u \in M$, $||u - z||^2 = ||u - P_M z + P_M z - z ||^2$ where
$u - P_M z \in M$ and $P_M z - z \in M^{\perp}$.

Thus, $||u - z||^2 = || u - P_m z ||^2 + ||P_M z - z ||^2$. Therefore, 
$||u - \vctproj[M]{z}||^2 > 0$. Then $||u = z ||^2 \geq ||\vctproj[M]{z}||^2$
implies $||u - z|| \geq ||\vctproj[M]{z} - z || = || z - \vctproj[M]{z} ||$.
This implies that $|| u - z || \geq ||z - \vctproj[M]{z}||$ for all $u \in M$.
Thus, $P_M (z)$ is the closest point in $M$ to $z$. 
\end{proof}

\section{Determinant}

\begin{definition}
Let $A$ be an $n \times n$ matrix, i.e., $A \in \mathbb{R}^{n \times n}$ or 
$A \in \mathbb{C}^{n \times n}$. Then the determinant of $A$, using co-factor
expansion is an inductive process on $n$. When $n = 1$ then $A$ is a scalar and
$\det(A) = A$. Now consider the general case. Then $A = [a_{ij}]$ where $a_{ij}$ is the
$(i, j)$ element of $A$. In this case, the cofactor expansion across the $i^{th}$ row is

\[ \det A = \sum_{j = 1}^n (-1)^{i + j} \cdot a_{ij} \cdot \det A_{ij} \]

where $A_{ij}$ is the submatrix of $A$ after removing the $i^{th}$ row and the $j^{th}$
column of $A$. Therefore, $A_{ij}$ is an $(n - 1) \times (n - 1)$ matrix. By the inductive
hypothesis, $\det A_{ij}$ is known. Similarly, co-factor expansion down the $j^{th}$
column is

\[ \det A = \sum_{i = 1}^n (-1)^{i + j} \cdot a_{ij} \cdot \det A_{ij}\]

where $j$ is fixed and $i \in \{1, \ldots, n\}$.
\end{definition}

\begin{example} $ $
\begin{enumerate}[label = (\arabic*)]
\item Let $n = 2$ and let 
$\displaystyle A = \begin{bmatrix} a_{11} & a_{12} \\ a_{21} & a_{22} \end{bmatrix}$.
Using co-factor expansion across the first row we get,
\[\det A = (-1)^{1 + 1} \cdot a_{11} \cdot \det (A_{11}) + (-1)^{1 + 2} \cdot a_{12} \cdot \det A_{12}\]

where $A_{11} = [a_{22}]$ and $A_{12} = [a_{21}]$. Then 
$\det A = a_{11} \cdot a_{22} - a_{12} \cdot a_{21}$.

In this case, $\det A$ has 2 terms, each of which is a product of 2 elements of $A$.

\item Let $n = 3$ and $\displaystyle A = \begin{bmatrix} a_{11} & a_{12} & a_{13} \\
a_{21} & a_{22} & a_{23} \\ a_{31} & a_{32} & a_{33} \end{bmatrix}$. Using co-factor
expansion across the first row we have

\begin{align*}
\det(A) &= a_{11} \cdot \det A_{11} - a_{12} \cdot \det A_{12} + a_{13} \cdot \det A_{13} \\
&= a_{11} \cdot \det \begin{bmatrix} a_{22} & a_{23} \\ a_{32} & a_{33} \end{bmatrix} -
a_{12} \cdot \det \begin{bmatrix} a_{21} & a_{23} \\ a_{31} & a_{33} \end{bmatrix} +
a_{13} \cdot \det \begin{bmatrix} a_{21} & a_{22} \\ a_{31} & a_{32} \end{bmatrix} \\
&= a_{11}(a_{22}a_{33}- a_{23}a_{32}) - a_{12}(a_{21}a_{23} - a_{31}a_{23}) +
a_{13}(a_{21}a_{32}- a_{31}a_{22}).
\end{align*}

In this case, $\det(A)$ has 6 terms, each of which is a product of 3 elements of $A$.
\end{enumerate}
\end{example}

\begin{remark}
For an $n \times n$ matrix, $\det A$ contains $n!$ terms, each of which is a product
of $n$ elements of $A$.
\end{remark}

\begin{example}
Let $A$ be an $n \times n$ lower triangular matrix. Then

\begin{align*}
\det A &= 
\begin{bmatrix}
a_{11} & 0 & 0 & 0 \\
a_{21} & a_{22} & 0 & 0\\
\vdots & & \ddots & 0 \\
a_{n1} & a_{n2} & \cdots & a_{nn}
\end{bmatrix} \\
&= a_{11} \cdot \det A_{11} \\
&= a_{11} \cdot \det
\begin{bmatrix}
a_{22} & 0 & 0 & 0 \\
a_{32} & a_{33} & 0 & 0 \\
\vdots & & \ddots & 0 \\
a_{n1} & a_{n2} & \cdots & a_{nn}
\end{bmatrix} \\
&= a_{11}a_{22} \cdots a_{nn}.
\end{align*}

Similarly, the determinant of an upper triangular matrix si the product of the
diagonal entries.

\end{example}

\section{Properties of Determinants}

\begin{enumerate}[label = (\arabic*)]
\item The determinant is linear in each row or column when the other column/row is fixed.
\item Determinant under elementary row operations. There are three elementary row operations:
\begin{enumerate}[label = (\arabic*)]

\item row scaling: If we multiply a row of $A$ by a scalar $\alpha$ to get $B$, then the
 $\alpha \cdot \det A = \det B$.
\item If we switch a row in $A$ to get $B$ then $\det B = - \det A$.
\item If we replace a row in $A$ to get $B$ then $\det B = \det A$.  
\end{enumerate}
\item \begin{theorem}
A square matrix $A$ is invertible if and only if $\det A \neq 0$.
\end{theorem}

\begin{proof}
We will use the following facts in the proof:

\begin{enumerate}[label = (\alph*)]
\item An echelon form of $A$ is an upper triangular matrix and each
pivot is a diagonal entry of the echelon form.
\item $A$ is invertible if and only if each column/row has a pivot.
\end{enumerate}

Let $U$ be an echelon form of $A$. Then $A$ is invertible if and only
if $\det U \neq 0$. To get $U$, one only needs row switching and row
replacement,

\[A \sim U_1 \sim U_2 \sim \cdots \sim U \]

where each $\sim$ represents a row switch or row replacement. Therefore,

\[\det U = (-1)^r \cdot \det A \]

where $r$ is the number of row switchings used in this process. Then 

\[ \det U \neq 0 \iff \det A \neq 0.\]

Thus, $A$ is invertible if and only if $\det A \neq 0$.
\end{proof}

\item $\det (A^T) = \det A$

\item $\det (A \cdot B) = \det A \cdot \det B$

\item $\det \begin{bmatrix} A & B \\ 0 & D \end{bmatrix} = \det A \cdot \det B$

\item Suppose $A$ is invertible. Then $\displaystyle \det (A^{-1}) = \frac{1}{\det A}$.

\begin{proof}
Because $A^{-1} \cdot A = I$ then $\det ( A^{-1} \cdot A ) = \det I = 1$. Therefore,
$\det A \cdot \det A^{-1} = 1$, which implies $\displaystyle \det (A^{-1}) = \frac{1}{\det A}$.
\end{proof}

\item Suppose $A$ and $B$ are similar. Then $\det A = \det B$.

\begin{proof}
We did in homework.
\end{proof}
\end{enumerate}

\section{Eigenvalues and Eigenvectors}

Let $A$ be an $n \times n$ real/complex matrix.

\begin{definition}
A scalar $\lambda$ (which is possibly complex number) is an \textbf{eigenvalue} of $A$ if and only
if there is a nonzero vector $v$ such that

\[Av = \lambda \cdot v\]

where $v$ is an \textbf{eigenvector} of $A$ associated with the eigenvalue $\lambda$.

Here $(\lambda, v)$ is called an eigen-pair.
\end{definition}


\begin{definition}
The set of all distinct eigenvalues of $A$ is the \textbf{spectrum} of $A$, denoted by $\sigma (A)$.
\end{definition}


\noindent To determine all of the eignevalues of $A$:

\begin{align*}
\lambda \text{ is an eigenvalue of } A &\iff Av = \lambda v \text{ for some } v \neq 0 \\
&\iff Av - \lambda v = 0 \text{ for some } v \neq 0 \\
&\iff (A - \lambda I) v = 0 \text{ for some } v \neq 0 \\
&\iff A - \lambda I \text{ is singular or nonivertible} \\
&\iff \det(A - \lambda I) = 0.
\end{align*}

Therefore, $\lambda$ is an eigenvalue of $A$ if and only if $\det(A - \lambda I) = 0$.

Let $p(\lambda) = \det(A - \lambda I)$, where $\lambda \in \mathbb{C}$ is a scalar
variable.

It can be shown that $p(\lambda)$ is a polynomial function of degree $n$ and $p(\lambda)$
is the \textbf{characteristic polynomial} of $A$. Also,

\[p(\lambda) = 0\]

is the characteristic equation of $A$.

This equation has $n$ (possibly repeated) roots/solutions. Therefore, $A$ has at most $n$
distinct eigenvalues.

\begin{remark}
\begin{enumerate}[label = (\alph*)]
\item Even if $A$ is a real matrix, its eigenvalues maybe complex. As an example, consider
the matrix $A = \begin{bmatrix} 0 & -1 \\ 1 & 0 \end{bmatrix}$. This matrix is a skew
symmetric matrix. 

Then $p(\lambda) = \det(A - \lambda I) = 0$. Because

\begin{align*}
\det(A - \lambda I) &= \det \begin{bmatrix} -\lambda & -1 \\ 1 & -\lambda \end{bmatrix} \\
&=(-\lambda)^2 + 1 \\
&= \lambda^2 + 1 \\
&= p(\lambda)
\end{align*}

then

\begin{align*}
p(\lambda) &\iff \lambda^2 + 1 = 0 \\
&\iff \lambda = \pm \sqrt{-1}
\end{align*}

where $i = \sqrt{-1}$.

\item Let $A$ be an $n \times n$ hermitian matrix; i.e., $A^* = A$ (or $A^T = A$). Then any
eigenvalue of $A$ is a real number.

\begin{proof}
Let $\lambda$ be an eigenvalue of $A$. Then there exists a vector $v \neq 0$ such that
$Av = \lambda v$. Subtracting $\lambda v$ from both sides and taking the conjugate
transpose then we have $(Av)^* = (\lambda v)^*$. By properties of the conjugate transpose,
then $v^* \cdot A^* = \overbar{\lambda} \cdot v^*$. Since $A$ is hermitian, then
$A^* = A$. Therefore $v^* A = \overbar{\lambda} \cdot v^*$. Right multiplying both
sides by $v$ we have $v^* (Av) = \overbar{\lambda} \cdot v^* \cdot v$ and then because
$\lambda$ is an eigenvalue of $A$ we can replace $Av$ with $\lambda v$, 
$v^* ( \lambda \cdot v ) = \overbar{\lambda} \cdot v^* \cdot v$. Then moving the scalar
$\lambda$ to the left side we have $\lambda \cdot (v^* \cdot v) = \overbar{\lambda} (v^* \cdot v)$.
Because $v$ is an eigenvector we know $v \neq 0$ and so $v^* \cdot v = \langle v, v \rangle > 0$.
Therefore, it must be the case that $\lambda = \overbar{\lambda}$. Now let $\lambda = a + bi$.
Then $\lambda = \overbar{\lambda}$ implies that $a + bi = a - bi$ and so $bi = -bi$, and
therefore, $b = -b$. Since $b \in \mathbb{R}$, then this means that $b = 0$. Thus, 
$\lambda = a \in \mathbb{R}$.
\end{proof}

\begin{corollary}
If $A$ is a real symmetric matrix, i.e., $A = A^T \in \mathbb{R}^{n \times n}$, then each
eigenvalue of $A$ is real.
\end{corollary}


\item If $A$ is skew Hermitian, i.e., $A^* = -A$, then each eignevalue of $A$ is imaginary.

\begin{proof} Use similar technique as above proof.
\end{proof}

\item $A$ is singular (i.e., non-invertible) if and only if $A$ has a zero eigenvalue (i.e.,
$\lambda = 0$ is an eigenvalue of $A$).

\begin{proof}
\begin{align*}
\text{Suppose } \lambda = 0 \text{ is an eigenvalue of } A \text{ and so } \det(A - \lambda I) = 0 &\iff
\det A = 0 \\
&\iff A \text{ is singular }. 
\end{align*}
\end{proof}
\end{enumerate}
\end{remark}

\begin{definition}
Let $p(\lambda)$ be the characteristic polynomial of an $n \times n$ matrix $A$ and let $\sigma(A) =
\{\lambda_1, \ldots, \lambda_S \}$ be the distinct eigenvalues of $A$. Therefore,

\[ p(\lambda) = (-1)^n \cdot (\lambda - \lambda_1)^{a_1} \cdot (\lambda - \lambda_2)^{a_2} \cdots
(\lambda - \lambda_S)^{a_S} \]

where $a_i$, $i = 1, \ldots, S$, is the number of times that $\lambda_i$ is repeated. Therefore,
$1 \leq a_i \leq n$ and $a_1 + a_2 + \cdots + a_S = n$. Each $a_i$ is the \textbf{algebraic 
multiplicity} of $\lambda_i$.
\end{definition}

\noindent To determine the eigenvectors associated with an eigenvalue $\lambda$ and matrix $A$.

\begin{align*}
\text{Let } \lambda \in \mathbb{C} \text{ be an eigenvalue of an } n \times n \text{ matrix } A &\iff
\exists v \neq 0 \text{ such that } (A - \lambda I)v = 0 \\
&\iff \exists v \neq 0 \text{ such that } v \in N(A - \lambda I). 
\end{align*}

Then $v \neq 0$ is an eigenvector associated with $\lambda$ if and only if $v \neq 0$ and
$v \in N(A - \lambda I)$.

Here $N(A - \lambda I)$ is called the \textbf{eigenspace} associated with $\lambda$, and 
$N(A - \lambda I)$ is a non-trivial subspace for any eigenvalue $\lambda$ of $A$.

\begin{remark} $ $
\begin{enumerate}[label = (\arabic*)]
\item $dim(N(A - \lambda I)) \geq 1$ for all $\lambda \in \sigma(A)$ where $\sigma(A)$ is the
spectrum of $A$, i.e., the set of all distinct eigenvalues of $A$.

\item $dim(N(A - \lambda_i )) \leq a_i$, where $a_i$ is the algebraic multiplicity of $\lambda_i$.
Therefore,

\[ \sum_{i = 1}^{S} dim(N(A - \lambda_i I)) \leq \sum_{i = 1}^{S} a_i = n \]

where $S$ is the total number of distinct eigenvalues of $A$.
\end{enumerate}
\end{remark} 

\section{Diagonal Matrix and Diagonalization}

Recall that an $n \times n$ matrix $D$ is diagonal if

\[
D =
\begin{bmatrix}
d_{11} & 0 & 0 & 0\\
0 & d_{22} & 0 & 0\\
0 & 0 & \ddots & 0 \\
0& 0 & 0 & d_{nn}
\end{bmatrix}.
\]


Some useful properties of diagonal matrices:

\begin{enumerate}[label = (\arabic*)]
\item 

\[ D^2 = D \cdot D = 
\begin{bmatrix}
d_{11}^2 & 0 & 0 & 0\\
0 & d_{22}^2 & 0 & 0\\
0 & 0 & \ddots & 0 \\
0& 0 & 0 & d_{nn}^2
\end{bmatrix}
\]

and

\[ D^k = D \cdot D \cdots D = 
\begin{bmatrix}
d_{11}^k & 0 & 0 & 0\\
0 & d_{22}^k & 0 & 0\\
0 & 0 & \ddots & 0 \\
0& 0 & 0 & d_{nn}^k
\end{bmatrix}
\]

for all $k = 1, 2, \ldots$.


\item $D \cdot \tilde{D} = \tilde{D} \cdot D$ where $D, \tilde{D}$ are diagonal.
\end{enumerate}

\begin{definition}
An $n \times n$ matrix $A$ is called \textbf{diagonalizable} if $A$ is similar to
a diagonal matrix $D$, i.e., there exists an invertible matrix $P$ such that 
$A = P \cdot D \cdot P^{-1}$.
\end{definition}

\begin{theorem}
An $n \times n$ matrix $A$ is diagonalizable if and only if $A$ has $n$ linearly independent
eigenvectors. Further, the diagonal entries of the diagonal matrix $D$ are the eigenvalues
of $A$.
\end{theorem}

\begin{proof}
$(\Leftarrow)$ Suppose $A$ has $n$ linearly independent eigenvectors, $p_1, p_2, \ldots, p_n$.
Since each $p_i$ is an eigenvector associated with an eigenvalue $\lambda_i$, then
$A p_i = \lambda_i \cdot p_i$.

Let the matrix $P = \begin{bmatrix} p_1 & p_2 & \cdots & p_n \end{bmatrix}$ and because
$\{p_1, p_2, \ldots, p_n\}$ are linearly independent, then $P$ is invertible. Further,

\begin{align*}
A \cdot P &= A \cdot \begin{bmatrix} p_1 & p_2 & \cdots & p_n \end{bmatrix}\\
&= \begin{bmatrix} Ap_1 & Ap_2 & \cdots & Ap_n \end{bmatrix} \\
&= \begin{bmatrix} \lambda_1 p_1 & \lambda_2 p_2 & \cdots & \lambda_n p_n \end{bmatrix} \\
&= \begin{bmatrix} p_1 & p_2 & \cdots & p_n \end{bmatrix}
\begin{bmatrix}
\lambda_1 & 0 & 0 & 0\\
0 & \lambda_2 & 0 & 0\\
0 & 0 & \ddots & 0 \\
0& 0 & 0 & \lambda_n
\end{bmatrix} \\
&= P \cdot D.
\end{align*}

Therefore, $A \cdot P = P \cdot D$ where $D$ is diagonal and $P$ is invertible. Also,
$A = P \cdot D \cdot P^{-1}$ and so $A$ is similar to $D$ and therefore $A$ is
diagonalizable.

$(\Rightarrow)$ Suppose $A$ is diagonalizable. Then $A = P \cdot D \cdot P^{-1}$ where

\[ D =
\begin{bmatrix}
d_{11} & 0 & 0 & 0\\
0 & d_{22} & 0 & 0\\
0 & 0 & \ddots & 0 \\
0& 0 & 0 & d_{nn}
\end{bmatrix}
\]

and $P = \begin{bmatrix} p_1 & p_2 & \cdots & p_n \end{bmatrix}$, which is invertible.
Therefore $\{p_1, p_2, \ldots, p_n\}$ is linearly independent.

Also, $A \cdot P = P \cdot D$. Therefore,

\[ A \begin{bmatrix} p_1 & p_2 & \cdots p_n \end{bmatrix} =
\begin{bmatrix} p_1 & p_2 & \cdots & p_n \end{bmatrix} 
\begin{bmatrix}
d_{11} & 0 & 0 & 0\\
0 & d_{22} & 0 & 0\\
0 & 0 & \ddots & 0 \\
0& 0 & 0 & d_{nn}
\end{bmatrix} 
\]

and so

\[
\begin{bmatrix} Ap_1 & Ap_2 & \cdots & Ap_n \end{bmatrix} = 
\begin{bmatrix} d_{11}p_1 & d_{22}p_2 & \cdots & d_{nn}p_n \end{bmatrix}
\]

because if two matrices are the same then the columns are the same.

Therefore, $Ap_i = d_{ii} \cdot p_i$ for all $i = 1, \ldots, n$. Note that
$p_i$ cannot be zero because $\{p_1, p_2, \dots, p_n\}$ is linearly independent.

Therefore each $d_{ii}$ is an eigenvalue of $A$, and $p_i$ is the eigenvector associated
with $d_{ii}$. Thus, $A$ has $n$ linearly independent eigenvectors.
\end{proof}

\section{Eigenbasis and Diagonalizable Matrices}

\begin{definition}
Given an $n \times n$ matrix $A$, a basis $B = \{v_1, \ldots, v_n\}$ is an
\textbf{eigenbasis} for $\mathbb{R}^n$ or $\mathbb{C}^n$ if each $v_i$ is an
eigenvector.
\end{definition}

\begin{corollary}
$A$ is diagonlizable if and only if $A$ has an eigenbasis.
\end{corollary}

\begin{proposition}
Let $S = \{v_1, \ldots, v_p\}$ of eigenvectors of $A$, where each $v_i$ is associated
with a ``distinct'' eigenvalue $\lambda_i$. Then $S$ is linearly independent.
\end{proposition}

\begin{proof}
First consider the case where $p > 1$. For sake of contradiction suppose $S$ is
linearly independent (where $p \geq 2$). Let $\{v_1, \ldots, v_r\}$ be a minimal
spanning set of $span(S)$, where $r < p$.

Therefore, $\{v_1, \ldots, v_r\}$ is linearly independent and $v_{r+1}$ is a linear
combination of $v_1, \ldots, v_r$, i.e., 

\[v_{r+1} = \alpha_1 v_1 + \cdots + \alpha_r v_r\]

for some scalars $\alpha_1, \ldots, \alpha_r$. Then multiplying both sides by $A$ we have

\[A v_{r + 1} = \alpha_1 A v_1 + \cdots + \alpha_r A v_r. \]

Then we can rewrite the $v_{r + 1}$ as 

\[ \lambda_{r + 1} v_{r + 1} = \alpha_1 \lambda_1 v_1 + \cdots \alpha_r \lambda_r v_r.\]

Then plugging in $v_{r + 1} = \alpha_1 v_1 + \cdots + \alpha_r v_r$ in the above equation,
we have

\[\lambda_{r + 1} \cdot (\alpha_1 v_1 + \cdots + \alpha_r v_r) = \alpha_1 \lambda_1 v_1 + \cdots +
\alpha_n \lambda_n v_n\]

and multiplying through on the right hand side by $\lambda_{r + 1}$ we have

\[
\alpha_1 \lambda_{r + 1} \cdot v_1 + \cdots + \alpha_r \lambda_{r + 1} v_r = \alpha_1
\lambda_1 v_1 + \cdots + \alpha_n \lambda_n v_n 
\]

and moving all terms on the right to left side of the equation and we get

\[ 
\alpha_1 (\lambda - \lambda_{r + 1}) \cdot v_1 + \cdots + 
\alpha_r (\lambda_r - \lambda_{r + 1}) v_r = 0.
\]

Since $\{v_1, \ldots, v_r\}$ is linearly independent then
$\alpha_1 (\lambda_1 - \lambda_{r + 1}) = 0, \ldots, \alpha_r (\lambda_r - \lambda_{r + 1}) = 0$.

Then since $\lambda_i \neq \lambda_j$ for $i \neq j$ then
$\alpha_1 = 0, \alpha_2 = 0, \ldots, \alpha_r = 0$.

This shows that $v_{r + 1} = \alpha_1 v_1 + \cdots + \alpha_r v_r = 0$, which is a 
contradiction because $v_{r + 1}$ is an eigenvector and cannot be equal to 0.

Thus, $S$ is linearly independent.
\end{proof}

\begin{remark} $ $
\noindent The following is an implication of the above proposition:

Suppose an $n \times n$ matrix $A$ has $n$ distinct eigenvalues $\lambda_1, \ldots, \lambda_n$.
In this case, each $\lambda_j$ has its eigenvector $v_i$, where $v_i \neq 0$.

Then by the preceeding proposition, $\{v_1, v_2, \ldots, v_n \}$ is linearly
independent. Then by a preceding theorem for necessary and sufficient condition
for diagonalizability, $A$ is diagonalizable.
\end{remark}

\begin{remark}

\noindent Recall that $\sigma(A) = \{\lambda_1, \ldots, \lambda_S\}$, i.e., the set of distinct
eigenvalues of $A$.

For each $\lambda_i \in \sigma(A)$ its eigenspace is $N(A - \lambda_i I)$. Let $B_i$ be a
basis for the eigenspace $N(A - \lambda_i I)$.

Then for any $\lambda_i, \lambda_j \in \sigma(A) = \{\lambda_1, \ldots, \lambda_S\}$ with $i \neq j$, 
$B_i \cap B_j = \varnothing$.
\end{remark}

\begin{proposition}
In the above setting, where $\sigma(A) = \{\lambda_1, \ldots, \lambda_S\}$,
$B = B_1 \cup B_2 \cup \cdots \cup B_S$ is linearly independent.
\end{proposition}

\begin{proof}
Let $G_k = B_1 \cup B_2 \cup \cdots \cup B_k$, for $k = 1, \ldots, S$. We use induction on $k$.

For the base case, assume $k = 1$. In this case, $G_1 = B_1$. Since $B_1$ is a basis for
$N(A - \lambda_1 I)$ then $B_1$ is linearly independent. Thus, $G_1$ is linearly independent.

Now suppose that $G_k$ is linearly independent and consider $G_{k + 1}$. Note that 
$G_{k + 1} = G_k \cup B_{k + 1}$ where $G_k = \{ v_1, \ldots, v_p \}$ and $B_{k + 1} = \{u_1, \ldots, u_r\}$
is a basis for $N(A - \lambda_{k + 1} I)$. Note that the $v_i$ in $G_k$ are eigenvectors associated with
other eigenvalues.

Suppose scalars $\alpha_1, \ldots, \alpha_r$ and $\beta_1, \ldots, \beta_p$ are such that

\[\alpha_1 u_1 + \cdots + \alpha_r u_r + \beta_1 v_1 + \cdots + \beta_p v_p = 0.\]

We want to show that $\alpha_1 = \cdots = \alpha_r = \beta_1 = \cdots = \beta_p = 0$.

From the above equation we have

\[
(A - \lambda_{k + 1} I) (\sum_{i = 1}^{r} \alpha_i u_i + \sum_{j = 1}^{p} \beta_j v_j) = 0.
\]

Distributing the $(A - \lambda_{k + 1} I)$ term we get,

\[
(A - \lambda_{k + 1} I) \sum_{i = 1}^{r} \alpha_i u_i + \sum_{j = 1}^{p} \beta_j (A - \lambda_{k + 1} I) \cdot v_j = 0.
\]

Since each $u_i \in N(A - \lambda_{k + 1} I)$ then $\sum_{i = 1}^{r} \alpha_i u_i \in N(A - \lambda_{k + 1}I)$ because
the null space is a subspace. Therefore $(A - \lambda_{k + 1} I) \sum_{i = 1}^{r} \alpha_i u_i = 0$. Further,

\begin{align*}
(A - \lambda_{k + 1} I ) v_j &= Av_j - \lambda_{k + 1}v_j \\
&= A q_j v_j - \lambda_{k + 1} v_j \\
&= (\lambda_{q_j} - \lambda_{k + 1})v_j
\end{align*} 

where $(\lambda_{q_j} - \lambda_{k + 1}) \neq 0$ because $\lambda_{q_j} \neq \lambda_{k + 1}$.

Then since $\sum_{j = 1}^p \beta_j (\lambda_{q_j} - \lambda_{k + 1}) v_j = 0$ then
$G_k = \{v_1, \ldots, v_k\}$ are linearly independent. Therefore 
$B_j \cdot (\lambda_{q_j} - \lambda_{k + 1}) = 0$ for all $j = 1, \ldots, p$. This implies
that $\beta_j = 0$ for $j = 1, \ldots, p$.

This shows that $\alpha_1 u_1 + \cdots + \alpha_r u_r = 0$ and therefore
$B_{k + 1} = \{u_1, \ldots, u_r\}$ is a basis for $N(A - \lambda_{k + 1}I)$ and so
$\{u_1, \ldots, u_r\}$ is linearly independent. Thus, $\alpha_1 = \cdots = \alpha_r = 0$ and
so $G_{k + 1} = G_k \cup B_{k + 1}$ is linearly independent. By induction principle,
each $G_k$ is linearly independent. Thus $G_S$ is linearly independent.
\end{proof}


\begin{theorem}
An $n \times n$ matrix $A$ is diagonalizable if and only if for each $\lambda_i \in \sigma(A)$, its
geometric multiplicity, i.e., $dim(N(A - \lambda_i I))$, equals its algebraic multiplicity.
\end{theorem}

\begin{proof}
$(\Leftarrow)$ Suppose $dim[N(A - \lambda_j I)] = a_i$ for all $\lambda_i \in \sigma(A)$
where $\{ \lambda_1, \ldots, \lambda_S\}$. Since $\displaystyle \sum_{i=1}^{S} a_i = n$,
then $\displaystyle \sum_{i = 1}^{S} [N(A - \lambda_j I)] = n$.

Let $B_i$ be a basis for each eigenspace $N(A - \lambda_i I)$, for all 
$\lambda_i \in \sigma(A)$. Therefore, $\# (B_i) = dim [N(A - \lambda_i I)]$ for all
$\lambda_i \in \sigma(A)$, where $\#(\cdot)$ is number of vectors in a finite set. Therefore,
$\displaystyle \sum^{S}{i = 1} \#(B_i) = n$.

Recall that $B_i \cap B_j = \varnothing$ for all $i \neq j$. Therefore,
$\#(B_1 \cup B_2 \cup \cdots \cup B_S) = n$, where $B = B_1 \cup B_2 \cup \cdots \cup B_S$.

Also, from the preceding proposition, we know $B = B_1 \cup B_2 \cup \cdots B_S$ is 
linearly independent. Therefore, $B$ is a set of $n$ linearly independent eigenvectors
and so $A$ is diagonalizable.

$(\Rightarrow)$ Suppose $A$ is diagonalizable. Then there exists an invertible matrix
$P$ such that $A = PDP^{-1}$ where $D$ is a diagonal matrix. Without loss of generality,
$D$ can be written as

\[
D =
\begin{bmatrix}
\lambda_1 & & & & & & & & & & \\
& \ddots & & & & & & & & & \\
& & \lambda_1 & & & & & & & & \\
& & & \lambda_2 & & & & & & & \\
& & & & \ddots & & & & & \\
& & & & & \lambda_2 & & & & & \\
& & & & & & \ddots & & & & \\
& & & & & & & \lambda_S & & & \\
& & & & & & & & \ddots & & \\
& & & & & & & & & \lambda_S & \\
\end{bmatrix}.
\]

Here $a_1$ is the algebraic multiplicity and $a_1 + a_2 + \cdots + a_S = n$.

Consider the eigenvalue $\lambda_1$. Then

\begin{align*}
A - \lambda_1 I &= P \cdot D \cdot P^{-1} - P (\lambda_1 I ) \cdot P^{-1} \\
&= P \cdot ( D - \lambda_1 I) P^{-1} \\
&=
P \cdot
\begin{bmatrix}
0 & & & & \\
& \ddots & & & \\
& & 0 & & & \\
& & & \ddots & \\
& & & & \ddots 
\end{bmatrix}
\cdot P^{-1} 
\end{align*}

where all the first $a_1$ eigenvalues are set to zero. Then the number of pivots in
previous matrix (all $a_1$ entries are \textbf{not} pivots but all others are). Therefore,

\[ rank(A - \lambda_1 I) = n - a_1. \]

Since $A - \lambda_1 I$ is similar to $D - \lambda_1 I$, then

\begin{align*}
rank(A - \lambda_1 I) &= rank( D - \lambda_1 I) \\
&= n - a_1.
\end{align*}

From the rank-nullity theorem, we have

\begin{align*}
dim[N(A - \lambda I)] &= n - rank(A - \lambda_1 I) \\
&= n - (n - a_1) \\
&= a_1. 
\end{align*}

Therefore, $dim[N(A - \lambda_1 I)] = a_1$.

Likewise, for each $\lambda_2, \ldots, \lambda_S \in \sigma(A)$ then
$dim[N(A - \lambda_i I)] = a_i$ for all $i = 2, \ldots, S$. 
\end{proof}

\begin{remark} $ $
The above proof shows that $A$ is diagonalizable if and only if

\[B = B_1 \cup B_2 \cup \cdots \cup B_S \]

has $n$ linearly independent eigenvectors

\begin{align*}
&\iff B \text{ is an eigenbasis for } \mathbb{R}^n \text{ or } \mathbb{C}^n \\
&\iff \mathbb{R}^n (\text{ or } \mathbb{C}^n) = N(A - \lambda_1 I) \oplus \cdots
\oplus N(A - \lambda_S I)
\end{align*}

Then any vector can be written as a unique combination of vectors from

\[ N(A - \lambda_1 I) \oplus N(A - \lambda_2 I) \oplus \cdots \oplus N(A - \lambda_S I). \]

\end{remark}

\begin{example} $ $

\begin{enumerate}[label = (\arabic*)]

\item Let $A = \begin{bmatrix} 4 & 0 & 0 \\ 1 & 3 & 0 \\ 0 & 0 & 3 \end{bmatrix}_{3 \times 3}$


where $A$ is lower triangular. Therefore, $A - \lambda I$ is also lower triangular. Then
$\det (A - \lambda I) = (4 - \lambda)(3 - \lambda)^2$. Then the two distinct eigenvalues
are $\lambda_1 = 4$ with algebraic multiplicity $a_1 = 1$ and $\lambda_2 = 3$ with
algebraic multiplicity $a_2 = 2$.

Now we find the geometric multiplicity of both eigenvalues. Consider $\lambda_1 = 4$. Then

\[A - \lambda_1 I =
\begin{bmatrix}
0 & 0 & 0 \\
-1 & -1 & 0 \\
0 & 0 & -1
\end{bmatrix}
\]

and so the $rank(A - \lambda I) = 2$. Therefore, $dim[N(A - \lambda_1 I)] = 3 - 2 = 1 = a_1$.

Now consider $\lambda_2 = 3$. Then

\[
A - \lambda_2 I =
\begin{bmatrix}
1 & 0 & 0 \\
1 & 0 & 0 \\
0 & 0 & 0
\end{bmatrix}
\]

and so $rank(A - \lambda_2 I) = 1$. Therefore, $dim[N(A - \lambda_2 I)] = 3 - 1 = 2 = a_2$.

Therefore, $A$ is diagonalizable.

\item Now consider

\[A =
\begin{bmatrix}
4 & 0 & 0 \\
0 & 3 & 1 \\
0 & 0 & 3
\end{bmatrix}
\]
 
which is an upper triangular matrix. Therefore, $A$ has two distinct eigenvalues $\lambda_1 = 4$
with algebraic multiplicity $a_1 = 1$ and $\lambda_2 = 3$ with algebraic multiplicity
$a_2 = 2$.

For $\lambda_1 = 4$, $dim[(N - \lambda_1 I)] = 1 = a_1$.

For $\lambda_2 = 3$

\[A - \lambda_2 I =
\begin{bmatrix}
1 & 0 & 0 \\
0 & 0 & 1 \\
0 & 0 & 0
\end{bmatrix}.
\]

Therefore, $rank(A - \lambda_2 I) = 2$ and so $dim[N(A - \lambda_2 I)] = 3 - 2 = 1 < a_2 = 2$.

Thus, $A$ is \textbf{not} diagonalizable.
\end{enumerate}
\end{example}

\section{Unitary Diagonalization and Spectral Theorem}

\subsection{Facts about normal matrices}

An $n \times n$ real/complex matrix $A$ is \textbf{normal} if and only if 
$A \cdot A^* = A^* \cdot A$ (or $A A^T = A^T A$).

Facts about normal matrices:

\begin{enumerate}[label = (\arabic*)]
\item $N(A^*) = N(AA^*) = N(A^* A) = N(A)$.
\item $R(A) \perp N(A)$, $\Rightarrow$ $A$ represents an orthogonal projector. (if $A^2 = A$).
\item Let $\lambda_i$, $\lambda_j$ be two distinct eigenvalues of $A$. Then
$N(A - \lambda_i I) \perp N(A - \lambda_j I)$.
\end{enumerate}

\begin{theorem} (Unitary Diagonalization)
Let $A$ be an $n \times n$ complex matrix. Then $A = U \cdot D \cdot U^*$, where $U$ is a
unitary matrix (i.e., $U \cdot U^* = U^* \cdot U = I$) and $D$ is diagonal, if and only
if $A$ is normal.

In other words, $A$ has an orthonormal eigenbasis for $\mathbb{C}^n$ if and only if
$A$ is a normal matrix.
\end{theorem}

\begin{proof}
We will only prove $(\Rightarrow)$ here because $(\Leftarrow)$ is more involved.

$(\Rightarrow)$ Suppose $D$ is diagonal matrix. Then 

\[ D \cdot D^* = D^* \cdot D. \]

This is because $D \cdot D^*$ and $D^* \cdot D$ are diagonal, and 
$(D \cdot D^*)_{ii} = d_{i} \cdot \overbar{d_i}$ and 
$(D^* \cdot D)_{ii} = \overbar{d_i} \cdot d_i$.

Assume $A = U \cdot D \cdot U^*$, where $U$ is unitary and $D$ is diagonal. Therefore,

\begin{align*}
A \cdot A^* &= (U \cdot D \cdot U^*) \cdot (U \cdot D \cdot U^*)^* \\
&= U \cdot D \cdot D^* \cdot U^*
\end{align*}

and

\begin{align*}
A^* \cdot A &= (U \cdot D \cdot U^*)^* \cdot (U \cdot D \cdot U^*) \\
&= U \cdot D^* \cdot D \cdot U^*.
\end{align*}

Therefore, $AA^* = A^*A$ and so $A$ is normal.
\end{proof}

\begin{remark}

The following is a special case of the above theorem.

Let $A \in \mathbb{R}^{n \times n}$ be a real symmetric matrix, i.e., $A = A^T$. (This
implies $A$ is normal)

Then,
\begin{enumerate}[label = (\arabic*)]
\item Each eigenvalue of $A$ is real.
\item All eigenvectors are in $\mathbb{R}^n$.
\item By the past theorem, there exists an orthogonal matrix $P \in \mathbb{R}^{n \times n}$
(Therefore, $P \cdot P^T = P^T \cdot P = I$.) and a diagonal matrix $D$ such that
$A = P^T \cdot D \cdot P$. This is called the spectral decomposition of a real symmetric
matrix, i.e., $A$ has on O.N. eigenbasis for $\mathbb{R}^n$, and the diagonal entries
of $D$ are the real eigenvalues of $A$.
\end{enumerate}
\end{remark}

\section{Positive definite and semi-definite, and indefinite matrices}

\begin{definition}
Let $A \in \mathbb{R}^{n \times n}$ be a real symmetric matrix, i.e., $A = A^T$.

\begin{enumerate}[label = (\arabic*)]
\item We call $A$ \textbf{positive definite} (P.D.) if

\[ x^T \cdot a \cdot x > 0, \quad \forall 0 \neq x \in \mathbb{R}^n. \]

\item We call $A$ \textbf{positive semi-definite} (P.S.D.) if

\[ x^T \cdot A \cdot x \geq 0, \quad \forall x \in \mathbb{R}^n. \]

\item We call $A$ \textbf{indefinite} if there exists $x \in \mathbb{R}^{n \times n}$ and
$y \in \mathbb{R}^{n \times n}$ such that

\[ x^T \cdot A \cdot x > 0 \quad \text{ but } \quad y^T \cdot A \cdot y < 0. \] 

\end{enumerate}
\end{definition}

\begin{theorem}
Let $A \in \mathbb{R}^{n \times n}$ be a real symmetric matrix. The following are
equivalent:

\begin{enumerate}[label = (\arabic*)]
\item $A$ is positive definite.
\item Each eigenvalue of $A$ is positive.
\item There is an invertible matrix $B$ such that $A = B^T \cdot B$.
\end{enumerate}
\end{theorem}

\begin{proof}
$1) \Rightarrow 2)$ Suppose $A$ is P.D. Let $v \in \mathbb{R}^n$ be an eigenvector
associated with an eigenvalue $\lambda$. Then $A v = \lambda v$ and 
$v^T A v = v^T (\lambda \cdot v) = \lambda \cdot v^T v$ where $v^T \cdot A \cdot v > 0$
and $v^T v > 0$ because $v \neq 0$.

Since $A$ is positive-definite and $v \neq 0$ then $v^T A v > 0$ and so

\[\lambda = \frac{v^T A v} {v^T v} > 0 \]

where $v^T v > 0$. Therefore, each eigenvalue $\lambda$ of $A$ is positive.

$2) \Rightarrow 3)$: We will use the spectral decomposition. Suppose $A$ is real
symmetric and all eigenvalues are positive. Therefore, by the spectral decomposition
of $A$, $A = P^T \cdot D \cdot P$, where $P$ is orthogonal and

\[ D =
\begin{bmatrix}
\lambda_1 & & \\
& \ddots & \\
& & \lambda_n
\end{bmatrix}
\]

and $\lambda_i > 0$ for all $i$.


Define $D^{\frac{1}{2}}$ as

\[
D^{\frac{1}{2}} =\
\begin{bmatrix}
\sqrt{\lambda_1} & & & \\
& \sqrt{\lambda_2} & & \\
& & \ddots & \\
& & & \sqrt{\lambda_n}
\end{bmatrix}
\]

which is a diagonal matrix. Then $D^{\frac{1}{2}} \cdot D^{\frac{1}{2}} = D$ and
$D^{\frac{1}{2}}$ is invertible.

Therefore, 

\[ A = P^T \cdot D^{\frac{1}{2}} \cdot D^{\frac{1}{2}} \cdot P =
(P^T \cdot D^{\frac{1}{2}}) \cdot (D^{\frac{1}{2}} \cdot P).\]

Let $B = D^{\frac{1}{2}} \cdot P$. Since $D^{\frac{1}{2}}$ and $P$ are invertible, then
$B$ is invertible. Also, $B^T = (D^{\frac{1}{2}} \cdot P)^T = P^T (D^{\frac{1}{2}})^T = P^T \cdot D^{\frac{1}{2}}$,
therefore $A = B^T \cdot B$.

$3) \Rightarrow 1)$ Suppose $A = B^T \cdot B$, where $B$ is invertible. We must show $A$ is P.D. For any
$0 \neq x \in \mathbb{R}^n$,

\begin{align*}
x^T A x &= x^T \cdot B^T \cdot B \cdot x \\
&= (Bx)^T \cdot (Bx) = || B x ||^2.
\end{align*}

Since $x \neq 0$ and $B$ is invertible, then $Bx \neq 0 \Rightarrow || Bx ||^2 > 0$. Therefore,
$x^T \cdot A \cdot x > 0$ for all $0 \neq x \in \mathbb{R}^n$. Therefore, $A$ is positive
definite.
\end{proof}

If $A$ is positive semi-definite then each eigenvalue is non-negative, $A = B^T \cdot B$, where
$B \in \mathbb{R}^{n \times n}$, not necessarily invertible.

\end{document}

